\documentclass[a4paper,1pt]{report}
\usepackage[utf8]{inputenc}
\usepackage{amsfonts}
\usepackage{amsthm}
\usepackage{amssymb}

\newtheorem*{pbo}{Principio del Buen Ordenamiento}

\newtheorem*{pim}{Principio de Inducción Matemática}

\newtheorem*{teo}{Teorema}

\newtheorem*{cor}{Corolario}

\newtheorem*{dem}{Demostración}

\newtheorem*{dfn}{Definición}

\newtheorem*{lem}{Lema}

\newtheorem*{prp}{Propiedades}

% Title Page
\title{Conferencia 1 - Principios de la Teoría de Números}
\author{}



\begin{document}
\maketitle

%\begin{abstract}
%\end{abstract}

\begin{pbo}

Todo subconjunto no vacío de $\mathbb{Z}_{+}$ contiene un elemento mínimo. O sea,
$\exists(m)$ tal que $\forall(x) x\in A\wedge x\neq m$ se cumple que $m<x$
\end{pbo}

\begin{pim}
 Dada una proposición $P$, si se cumple $P(n_0)$ con $n_0\in \mathbb{Z}_{+}$ y, además, 
 $\forall(n)$ $n\geq n_0\wedge P(n) \Rightarrow P(n+1)$ entonces $\forall(n)$ $n\geq n_0 \wedge P(n)$
\end{pim}

\begin{teo}
 El Principio del Buen Ordenamiento es equivalente al Principio de Inducción Matemática
\end{teo}

\textbf{Demostración}

Sea $C$ el conjunto de los números naturales que no cumplen $P$ y asumamos que $P\neq \varnothing$. Entonces, por el \textbf{Principio del Buen Ordenamiento} existe $m\in C$ tal que $m$ es el mínimo elemento de $C$.

Ahora, asumamos a 1 como $n_0$, luego como $P(1)$ se cumple entonces $m>1$ por lo que $m-1\geq 1$.

Como $m-1<m$ entonces $m-1\notin C$ por lo que $P(m-1)$ se cumple. Por tanto, como para todo $n>1$ se tiene que $P(n)\Rightarrow P(n+1)$ entonces dado que $P(m-1)$ se cumple se tendría que $P(m)$ también se cumple ¡lo que es una contradicción!

\textbf{Ejemplo}
Demuestre, utilizando el \textbf{Principio del Buen Ordenamiento}, que para toda $n$, $n\in\mathbb{Z}$, $n\geq 1$ se cumple que $\sum^n_{k=1}(2k-1)=n^2$

Sea $C$ el conjunto de los números naturales que no cumplen $P$ y asumamos que $P\neq \varnothing$. Entonces, por el \textbf{Principio del Buen Ordenamiento} existe $m\in C$ tal que $m$ es el mínimo elemento de $C$.

$P(1)$ se cumple pues $\sum^1_{k=1}(2k-1)=2-1=1=1^2$, por tanto $m>1$ por lo que $m-1\geq1$. Ahora, como $m-1\geq m$ entonces $m-1\notin C$ por lo que $P(m-1)$ se cumple. Entonces $\sum^{m-1}_{k=1}(2k-1)=(m-1)^2$.

Ahora se tiene que\\ 
$\sum^m_{k=1}(2k-1)=\sum^{m-1}_{k=1}(2k-1)+(2m-1)$\\
$\sum^m_{k=1}(2k-1)=(m-1)^2+(2m-1)$\\
$\sum^m_{k=1}(2k-1)=(m^2-2m+1)+(2m-1)$\\
$\sum^m_{k=1}(2k-1)=m^2$\\
O sea, $P(m)$ se cumple, lo que es una ¡contradicción!

\begin{dfn}
 Sean $a,b$, $a\in\mathbb{Z}$, $b\in\mathbb{Z}$, $a\neq 0$, se dice que $a$ divide a $b$ o que $a$ es múltiplo de $b$, denotado $a|b$, si $\exists(q)$ $q\in\mathbb{Z}$ tal que $b=a*q$
\end{dfn}

\begin{lem}
 Todo número $a$, $a\in\mathbb{Z}$, es divisor de 0
\end{lem}

\begin{teo}
 Sean $a,b$, $a\in\mathbb{Z}$, $b\in\mathbb{Z}$, si $b|a$ y $a\neq 0$ entonces $a\geq b$
\end{teo}



\begin{teo}
 La relación \textbf{ser divisor de} es transitiva. O sea, si $a|b$ y $b|c$ entonces $a|c$
\end{teo}

\textbf{Demostración}

\begin{teo}
 \textbf{Algoritmo de la División}, sean $a,b$, $a\in\mathbb{Z}$, $b\in\mathbb{Z}$, $a> 0$, entonces existen $q,r$, $q\in\mathbb{Z}$, $r\in\mathbb{Z}$, únicos tales que $b=a*q+r$ donde $0\leq r < b$
\end{teo}

\textbf{Demostración}

\begin{dfn}
 Sea $a\in\mathbb{Z}$ tal que $n>1$, se dice que $n$ es un \textbf{número primo} si y solo sus únicos divisores positivos son $1$ y $n$, de lo contrario se dice que $n$ es un \textbf{número compuesto}
\end{dfn}

\begin{cor}
 $n$, $n\in\mathbb{Z}$, $n>1$, es un \textbf{número compuesto} si y solo si $n=a*b$ con $a\in\mathbb{Z}$, $b\in\mathbb{Z}$, $1<a\leq b < n$
\end{cor}

\begin{lem}
 Todo número entero mayor que 1 tiene un divisor primo
\end{lem}

\textbf{Demostración}

\begin{teo}
 Hay una infinita cantidad de números primos
\end{teo}

\textbf{Demostración}

\begin{dfn}
 Sean $a,b$, $a\in\mathbb{Z}$, $b\in\mathbb{Z}$, se dice que $c$, $c\in\mathbb{Z}$, es común divisor de a y b si $c|a$ y $c|b$
\end{dfn}


\begin{dfn}
 Sean $a,b$, $a\in\mathbb{Z}$, $b\in\mathbb{Z}$, $a\neq 0$ o $b\neq 0$, se denota\\ $mcd(a,b)=max\{d|   d\in\mathbb{Z}\wedge d|a\wedge d|b\}$ como el máximo común divisor de a y b
\end{dfn}

El $mcd(a,b)$ también suele denotarse $(a,b)$ 

\begin{prp}
 $mcd(a,b)=mcd(-a,b)=mcd(a,-b)=mcd(-a,-b)$
\end{prp}

\begin{teo}
 Sean $a,b$, $a\in\mathbb{Z}$, $b\in\mathbb{Z}$, si $a|b$ entonces $mcd(a,b)=|a|$
\end{teo}


\begin{dfn}
 Sean $a,b$, $a\in\mathbb{Z}$, $b\in\mathbb{Z}$, si el $mcd(a,b)=1$ entonces a y b son \textbf{primos relativos}
\end{dfn}

\begin{dfn}
 Un entero $c$ es combinación lineal de los enteros $a_1,a_2,\dots,a_n$ si exiten enteros $b_1,b_2,\dots,b_n$ tales que $c=a_1*b_1+a_2*b_2+\dots+a_b*b_n$
\end{dfn}

\begin{teo}
 El máximo común divisor de  $a_1,a_2,\dots,a_n$, números enteros, no todos iguales a 0, $mcd(a_1,a_2,\dots,a_n)$ es el menor entero positivo que puede ser expresado como combinación lineal de $a_1,a_2,\dots,a_n$
\end{teo}

\textbf{Demostración}

\begin{teo}
 Sean $a,b$, $a\in\mathbb{Z}$, $b\in\mathbb{Z}$, el conjunto de los divisores comunes de $a$ y $b$ coincide con el conjunto de los divisores del $mcd(a,b)$
\end{teo}

\begin{cor}
 Si $a_1,a_2,\dots,a_n$ son números enteros no todos iguales a 0 entonces $mcd(a_1,a_2,\dots,a_n)=mcd(a_!,mcd(a_2,a_3,\dots,a_n))$
\end{cor}

\begin{cor}
 Sean $a,b$, $a\in\mathbb{Z}$, $b\in\mathbb{Z}$, no simultáneamente nulos, entonces $\frac{a}{mcd(a.b)}$ y  $\frac{b}{mcd(a.b)}$ son \textbf{primos relativos}. O sea, $mcd(\frac{a}{(a.b)},\frac{b}{(a.b)})=1$
\end{cor}

\begin{teo}
 Sea $a$, $a\in\mathbb{Z}$, $a\neq 0$, $b_i\in\mathbb{Z}$, $1\leq i \leq n$, si $a|b_1*b_2*\dots *b_n$ y para  todo $j$, $1\leq j \leq n-1$, se cumple que $mcd(a,b_j)=1$ entonces $a|b_n$
\end{teo}

\begin{cor}
 Sean $a,b,q,r$ tales que $a\in\mathbb{Z}$, $b\in\mathbb{Z}$, $q\in\mathbb{Z}$, $r\in\mathbb{Z}$, $b\neq 0$, y $a=q*b+r$ entonces $mcd(a,b)=mcd(b,r)$
\end{cor}

\textbf{Demostración}

\begin{dfn}
 Sean $a,b,c$, $a\in\mathbb{Z}$, $b\in\mathbb{Z}$, $c\in\mathbb{Z}$, $a\neq 0$, $b\neq 0$ se dice que $ax+by=c$ es una ecuación lineal diofantina si esta es resuelta con $x\in\mathbb{Z}$ y $y\in\mathbb{Z}$
\end{dfn}

\begin{teo}
 La ecuación lineal $ax+by=c$ tiene solución si y solo si $mcd(a,b)|c$
\end{teo}

\textbf{Demostración}

\begin{teo}
 \textbf{Algoritmo de Euclides}. Sean $a,b$, $a\in\mathbb{Z}$, $b\in\mathbb{Z}$, $a>b$, si se realizan los siguientes cálculos:\\
 $a=q_1*b+r_1$ $0\leq r_1<b$\\
 $b=q_2*r_1+r_2$  $0\leq r_2<r_1$\\
 $r_1=q_3*r_2+r_3$ $0\leq r_3<r_2$\\
 $r_2=q_4*r_3+r_4$ $0\leq r_4<r_3$\\
 \dots\\
 \dots\\
 \dots\\
 $r_{k-2}=q_k*r_{k-1}+r_k$ $0\leq r_k<r_{k-1}$\\
 $r_{k-1}=q_{k+1}*r_{k}$ $0=r_{k+1}$\\
 donde $r_k$ es el último resto diferente de 0, entonces $r_k=mcd(a,b)$
\end{teo}

\textbf{Ejemplo}
Para calcular el máximo común divisor de 3088 y 456:\\
$3088=6*456+352$\\
$456=1*352+104$\\
$352=3*104+40$\\
$104=2*40+24$\\
$40=1*24+16$\\
$24=1*16+8$\\
$16=2*8 + 0$\\
Entonces 8 es el último resto distinto de 0. Por tanto $mcd(3088,456)=8$

A partir del \textbf{Algoritmo de Euclides} también se puede calcular la combinación lineal de la siguiente forma:\\
$A_1=1$ $B_1=-q_k$\\
$A_2=B_1$ $B_2=A_1-q_{k-1}*B_1$\\
\dots\\
$A_{i+1}=B_i$ $B_{i+1}=A_i-q_{k-i}*B_i$\\
\dots\\
$A_{k-1}=B_{k-2}$ $B_{k-1}=A_{k-2}-q_2*B_{k-2}$\\
$A_{k}=B_{k-1}$ $B_{k}=A_{k-1}-q_1*B_{k-1}$\\
Luego $r_k=a*A_k+b*B_k$ y, por lo tanto, $r_k=a*A_k+b*B_k=mcd(a,b)$

\textbf{Ejemplo}
Para calcular la combinación lineal de 3088 y 456 con la que se obtiene su $mcd$ se tiene:\\
$3088=6*456+352$ $A_1=1$ $B_1=-1$\\
$456=1*352+104$ $A_2=-1$ $B_2=1-1*(-1)=2$\\
$352=3*104+40$ $A_3=2$ $B_3=-1-2*2=-5$\\
$104=2*40+24$ $A_4=-5$ $B_4=2-3*(-5)=17$\\
$40=1*24+16$ $A_5=17$ $B_5=-5-1*17=-22$\\
$24=1*16+8$ $A_6=-22$ $B_6=17-6*(-22)=149$\\
$16=2*8 + 0$\\
Por tanto $8=mcd(3088,456)=3088*(-22)+456*149$

\begin{teo}
 Si $x_0,y_0$ son una solución de la ecuación diofantina $ax+by=c$ entonces 
 $x=x_0+k\frac{b}{(a,b)}$ y $y=y_0-k\frac{a}{(a,b)}$ con $k\in\mathbb{Z}$
\end{teo}

\textbf{Demostración}

\begin{dfn}
 Sean $a,b,c$, $a\in\mathbb{Z}$, $b\in\mathbb{Z}$, $c\in\mathbb{Z}$, los tres distintos de 0, se dice que $c$ es múltiplo común de $a$ y $b$ si $c$ es múltiplo de $a$ y $c$ es múltiplo de $b$. Se dice que $c$ es el mínimo común múltiplo de $a$ y $b$, si es el menor entero positivo múltiplo común de de $a$ y $b$, lo que se denota $mcm(a,b)$.
\end{dfn}

El $mcm(a,b)$ también suele denotarse $[a,b]$ 

\begin{teo}
 Sean $a,b$, $a\in\mathbb{Z}_+$, $b\in\mathbb{Z}_+$, todo múltiplo común de $a$ y $b$ se expresa como $k\frac{a*b}{(a,b)}$ donde $k\in\mathbb{Z}$
\end{teo}

\begin{cor}
 El $mcm(a,b)=\frac{|a*b|}{mcd(a,b)}$, lo que es lo mismo $(a,b)=\frac{|a*b|}{[a,b]}$
\end{cor}

\begin{cor}
 Todo múltiplo común de $a$ y $b$ es múltiplo común de $[a,b]$
\end{cor}














\end{document}          
