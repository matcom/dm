\documentclass[a4paper,12pt]{report}
\usepackage[utf8]{inputenc}
\usepackage{amsfonts}
\usepackage{amsthm}
\usepackage{amssymb}
\usepackage{amsmath} 

\newtheorem*{pbo}{Principio del Buen Ordenamiento}
\newtheorem*{pim}{Principio de Inducción Matemática}
\newtheorem*{psm}{Principio de la Suma}
\newtheorem*{ppr}{Principio del Producto}
\newtheorem*{pie}{Principio de Inclusión - Exclusión}

\newtheorem*{teo}{Teorema}

\newtheorem*{cor}{Corolario}

\newtheorem*{dem}{Demostración}

\newtheorem*{dfn}{Definición}

\newtheorem*{lem}{Lema}

\newtheorem*{prp}{Propiedades}

\newtheorem*{pro}{Proposición}

% Title Page
\title{Conferencia 8 - Combinatoria}
\author{}



\begin{document}
\maketitle

%\begin{abstract}
%\end{abstract}

\begin{teo}
    El n\'umero de palabras de tamaño $k$ en un alfabeto de $n$ letras es $n^k$.
\end{teo}

\begin{dem}
    Por principio de la multiplicación tenemos $n$ opciones para cada posición de la palabra, y son $k$ posiciones por tanto $n^k$.
\end{dem}

\begin{teo}
    Repartir $n$ objetos distintos en $k$ categor\'ias diferentes es $k^n$.
\end{teo}

\begin{dem}
    Para cada uno de los objetos se decide en cual de las $n$ categor\'ias va a estar, como para cada uno hay $n$ opciones, por principio de multiplicaci\'on es $k^n$.
\end{dem}


\begin{teo}
    Binomio de Newton.
    
    Sean $n$ un entero no negativo, $a, b$ números cualesquiera.
    $$(a+b)^n = \sum^n_{k=0} \binom{n}{k}  a^{n-k}b^k$$

    ${n}\choose{k}$ se conocen como coeficientes binomiales
\end{teo}

\begin{dem}
    Considere el producto $(a+b)*...*(a+b)$, cada sumando que resulta se puede ver como una palabra de tamaño $n$ que solo tiene $a$ y $b$. Nótese que todas las cadenas posibles son creadas ($2^n$), porque en cada $(a+b)$ se usa la $a$ o la $b$. Luego de tener todos estos sumandos se pueden agrupar por la misma cantidad de $a$ (de esta forma también se fija la misma cantidad de $b$), entonces tendremos una cantidad de sumandos con $n-k$ $a$ y $k$ $b$ igual a ${n}\choose{k}$
\end{dem}


\textbf{Propiedades de los Coeficientes}

\begin{enumerate}
 \item  $$\binom{n}{k}=\binom{n}{n-k}$$
 
 \textbf{Demostración}
 
 $\binom{n}{k}$ representa la cantidad de formas en que se pueden seleccionar $k$ elementos de un conjunto que posee $n$ elementos, esto pudiera verse 
 como de cu\'antas formas puede particionarse el conjunto de $n$ elementos en dos categor\'ias $A$ y $B$ de modo que la categor\'ia $A$ tenga $k$ elementos y $B$ los $n-k$ elementos restantes,
 de donde el n\'umero de particiones en los conjuntos $A$ y $B$ tambi\'en se pudieran contar a partir de conocer de c\'uantas formas podemos poner $n-k$ elementos en el conjunto $B$ y dejar los $k$ restantes en el conjunto $A$, valor que se obtiene 
 mediante la expresi\'on ${n}\choose{n-k}$.

 Luego ${n}\choose{k}$$=$${n}\choose{n-k}$
 
 \item $$\binom{n}{k}=\binom{n-1}{k}+\binom{n-1}{k-1}$$
 
 \textbf{Demostración}
 
 ${n}\choose{k}$ es la cantidad de subconjuntos de tamaño $k$ que pueden obtenerse de un conjunto con cardinalidad $n$.
 
 Esta cantidad es también igual a la cantidad de subconjuntos de tamaño $k$ en la que no aparece un elemento $a_i$ más la cantidad de conjuntos del mismo tamaño en los que sí aparece.
 
 La cantidad de conjuntos en los que no aparece $a_i$ es igual a ${n-1}\choose{k}$
 
 La cantidad de conjuntos en los que sí aparece $a_i$ es igual a ${n-1}\choose{k-1}$
 
 Luego ${n}\choose{k}$$=$${n-1}\choose{k}$$+$${n-1}\choose{k-1}$
 
 
 \item $$k\binom{n}{k}=(n-k+1)\binom{n}{k-1}=n\binom{n-1}{k-1}$$
 
 \textbf{Demostración}
 
 Las tres expresiones cuentan el n\'umero de formas de seleccionar un subgrupo de $k$ personas de un grupo de $n$ personas y asignar a uno como l\'ider.

 La primera expresi\'in cuenta de cu\'antas formas se puede seleccionar $k$ de $n$ y de los $k$ seleccionados escoger uno de ellos para ser el l\'ider, lo cual por principio de multiplicaci\'on queda $k$${n}\choose{k}$. 
 
 La segunda expresi\'on cuenta de cu\'antas formas se puede seleccionar $k-1$ personas del grupo y de las restantes seleccionar a una como l\'ider y agregarla a la selecci\'on anterior, teniendose finalmente cuantos subgrupos hay de $k$ personas y una de ellas l\'ider $(n-k+1)$${n}\choose{k-1}$.

 La tercera expresi\'on cuanta de cuantas formas se puede selccionar al l\'ider entre las $n$ personas del grupo y luego seleccionar a $k-1$ miembros de las $n-1$ personas restantes, contando de igual modo los subgrupos de $k$ personas donde una de ellas es l\'ider.

 \item $$\binom{k}{k}+\binom{k+1}{k}+\binom{k+2}{k}+\dots +\binom{n}{k}=\binom{n+1}{k+1}$$\\ 
 o lo que es lo mismo $\sum^n_{j=k}$${j}\choose{k}$$=$${n+1}\choose{k+1}$\\\\
 
 \textbf{Demostración}

 El miembro derecho de la igualdad represnta el n\'umero de formas en que se pueden seleccionar $k+1$ elementos de un conjunto de $n+1$ elementos. De igual modo 
 cada sumando de la parte izquierda de la igualdad representa de cuantas formas seleccionar $k+1$ elementos de un conjunto de $n$ elementos, pero en cada uno fijando que el $k+1$-\'esimo elemento es el que ocupa la posicion $k+i +1$ y que los restantes $k$ elementos se toman de la posicion $1$ a la $k+i$. 
 Es decir:
 
 ${k}\choose{k}$ cuenta los casos donde el se toman $k$ elementos desde el $1$ al $k$ y el elemento $k+1$ es el que toma la posicion $k+1$.

 ${k+1}\choose{k}$ cuenta los casos donde el se toman $k$ elemantos desde el $1$ al $k+1$ y el elemento $k+1$ es el que toma la posicion $k+2$.

 ${k+2}\choose{k}$ cuenta los casos donde el se toman $k$ elementos desde el $1$ al $k+2$ y el elemento $k+1$ es el que toma la posicion $k+3$.

 y as\'i sucesivamente hasta que  ${n}\choose{k}$ cuenta los casos donde el se toman $k$ elementos desde el $1$ al $n$ y el elemento $k+1$ es el que toma la posicion $n+1$.

 Al sumar todos estos t\'erminos se obtiene en general el n\'umero de formas de escoger $k+1$ elementos de un conjunto de $n+1$ elementos.

\end{enumerate}


\begin{dfn}
 Un multiconjunto es el  par $<A,m>$ donde $A$ es un conjunto y $m$ es una función $m:A\rightarrow \mathbb{N}$. 
\end{dfn}

Se dice que para cada $a$ de $A$ la multiplicidad de $a$ es el número $m(a)$.\\

Si el conjunto $A$ es finito entonces el tamaño o longitud del multiconjunto $<A,m>$ es la suma de todas las multiplicidades de los elementos de $A$,\\ o sea, $\sum_{a\in A}m(a)$\\

Un submulticonjunto $<B,n>$ del multiconjunto $<A,m>$ cumple que $B\subseteq A$ y $n:B\rightarrow \mathbb{N}$ tal que $n(x)\leq m(x)$ para todo $x\in B$ 

\begin{teo}
 Sea un multiconjunto $N$ con $n$ objetos donde hay $n_1$ objetos de tipo 1, $n_2$ objetos de tipo 2 y así hasta $n_k$ objetos de tipo $k$ donde $n=\sum^k_{i=1}a_i$\\
 Entonces el número de permutaciones distintas de $N$ es $$\frac{n!}{n_1! n_2! \dots n_k!}$$
\end{teo}

\begin{dem}
    Sea $P$ el n\'umero total de permutaciones de un conjunto de $n$ elementos, sabemos que $P = n!$, luego sea $X$ el n\'umero total de permutaciones sin repeticiones, 
    por cada permutaci\'on que se haya contado en $X$, si en dicha permutacion intercambiamos los $n_1$ objetos de tipo $1$ seguimos obteniendo permutaciones iguales, pero que en $P$ se cuentan como diferentes, lo mismo 
    con los $n_2$ objetos de tipo $2$ y en general con los $n_i$ objetos de tipo $i$; esto por principio de la multiplicaci\'on ser\'ia $Xn_1!n_2!...n_k!$ y aqu\'i como por cada permutaci\'on diferente estamos contando todas sus r\'eplicas que se generan al permutar los objetos iguales entre s\'i, en general estamos contando todas las permutaciones del conjunto de $n$ elementos,
    luego $P = Xn_1!n_2!...n_k!$ de donde $X = \frac{n!}{n_1!n_2!...n_k!}$
\end{dem}


\textbf{Ejemplo}

Pruebe que $k!^{(k-1)!}|(k!)!$\\

Si se tiene un conjunto de $k!$ elementos donde hay $(k-1)!$ tipos diferentes y de cada tipo hay $k$ elementos entonces la cantidad de permutaciones distintas de este conjunto es

$$\frac{(k!)!}{k!k!\dots k!}=\frac{(k!)!}{k!^{(k-1)!}}$$ \\

luego como el número de permutaciones es un número entero entonces $k!^{(k-1)!}|(k!)!$

\begin{teo}
 El número de formas de particionar un conjunto de $n$ elementos distintos en $k$ categorías diferentes de forma que haya $n_1$ objetos en la categoría 1, $n_2$ objetos en la categoría 2 y así hasta llegar a $n_k$ objetos en la categoría $k$, donde $n = \sum^k_{i=1}n_i$ es  
 $\frac{n!}{n_1!n_2!\dots n_k!}$
\end{teo}


\textbf{Demostración}
Podemos contar las palabras de tamaño $n$ donde hayan $n_i$ letras del tipo $i$ y que la suma de los $n_i$ sea $n$. Y asociar cada palabra a una distribución de los $n$ objetos asignando el objeto $j$ a la letra que le corresponde y tratando dicha letra como una categoría.


\begin{teo}
 El número de formas de particionar $n$ objetos iguales en $k$ categorías diferentes es $$\binom{n+k-1}{k-1}$$
\end{teo}


\textbf{Demostración}
Este problema es equivalente a tener una secuencia de $n+k-1$ elementos iguales y convertir a $k-1$ de estos elementos en separadores. Luego, la solución tenemos el conjuntos de todas las posibles combinaciones de $k-1$ posiciones que pueden ser seleccionadas como separadores y esto es ${n+k-1}\choose{k-1}$

\begin{teo}
    El número de formas de particionar $n$ objetos iguales en $k$ categorías diferentes de modo que ninguna categor\'ia quede vac\'ia es $$\binom{n-1}{k-1}$$
   \end{teo}
   
   \textbf{Demostración}
La idea de soluci\'on es igual que la anterior de tomar $n + k -1$ y de ellos seleccionar $k-1$ como de ellos como separadores, pero es necesario garantizar que ninguna categor\'ia quede vac\'ia, entonces de mis $n$ elementos iniciales tomo $k$, para garantizar que puedo asignar uno de ellos a cada categor\'ia y luego 
proceso a agregar la $k-1$ barritas teniendo  ${n - k + k-1}\choose{k-1}$$=$${n -1}\choose{k-1}$

\begin{dfn}
    Estos dos últimos problemas también son conocidos como composición y composición débil de $n$. Formalmente:  
    
    Una secuencia $(a_1, a_2, ..., a_k)$ de enteros tales que $a_i \geq 0$ y $\sum_{i=1}^{k}a_i = n$ es llamada una composición débil de $n$, además si $a_i > 0$ entonces es llamada una composición de $n$.
\end{dfn}

\begin{teo}
    El número de composiciones de $n$ es $2^{n-1}$.
\end{teo}

\begin{dem}
    Notemos que se puede tomar la suma de las composiciones de $n$ para todas las $k$. Pero una demostración menos algebráica consiste en una inducción, haciendo ver que existe una forma de construir las composiciones de $n$ a partir de las de $n-1$.
\end{dem}

\begin{dfn}
    una partición de $N_n$ es una colección de conjuntos no vacíos tales que cada elemento de $N_n$ pertenezca a exactamente un conjunto de la colección. El número de particiones de $N_n$ en $k$ conjuntos no vacíos se denota por $S(n,k)$, es llamado número de Stirling de segundo tipo.
\end{dfn}

\begin{dfn}
    El número de todas las particiones de $N_n$ es denotado como $B(n)$ y conocido como n-ésimo número de Bell. 
\end{dfn}

\end{document}          
