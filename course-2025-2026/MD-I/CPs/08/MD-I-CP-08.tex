\documentclass{article}
\usepackage{enumitem}
\usepackage{tikz}
\usetikzlibrary{trees}
\usepackage{amsmath} 
\usepackage{amssymb}
\usepackage{nopageno}
\usepackage{listings}
\usepackage{xcolor}

\lstset{
	basicstyle=\ttfamily,
    keywordstyle=\color{blue},
    commentstyle=\color{green},
    stringstyle=\color{red},
    numbers=left,
    numberstyle=\tiny,
    stepnumber=1,
    numbersep=5pt,
    backgroundcolor=\color{gray!10},
    frame=single,
    tabsize=4,
    breaklines=true
}

\title{Clase pr\'actica 8}

\begin{document}

\maketitle
\begin{enumerate}
    \item Se tiene $2n+1$ caramelos de un mismo tipo y se quieren repartir a $3$ niños. Sin embargo se quiere que ningún niño tenga más caramelos que la suma de los caramelos que tienen los otros dos. De cuántas formas se puede lograr?
	\item Calcule el n\'umero de soluciones $x_1 + x_2 + x_3 + x_4 = 50$ donde los $x_i$ son todos impares.
	\item Cuántos números de $10$ cifras son divisibles por $9$ y sus dígitos solo pueden ser $1,2$ o $3$, además tienen que tener exactamente dos $3$. 
	\item Demuestre que:
	\item[] 
	\begin{itemize}
		\item $\sum_{k=0}^{n} (-1)^k \binom{n}{k} = 0$
		\item $\sum_{k=0}^{n} \binom{n}{k}^2 = \binom{2n}{n}$
		\item $\sum_{k=0}^{n} k \binom{n}{k} =n2^{n-1}$
	\end{itemize}
    \item De una forma cerrada para las siguientes expresiones:
    \begin{itemize}
        \item $\sum_{k=1}^{n}\sum_{i=1}^{k}i$
        \item $1^2 + 2^2 + 3^2 + \cdots + n^2$
    \end{itemize} 
	\item Calcule el n\'umero de subconjuntos $X$ de tama\~no k del conjunto $\{1,2,\dots,n\}$ tal que $\forall a,b \in X$ se cumple que $|a-b| \geq 3$.
	\item Calcule el n\'umero de arreglos de tama\~no $n$ ordenados de menor a mayor donde cada elemento es un n\'umero de 1 a $n$.
	\item Hay 12 bombillos alineados en una fila, enumerados desde el 1 al 12 y ordenados seg\'un su n\'umero de menor a mayor. Se sabe que los bombillos con los n\'umeros 3, 7 y 11 est\'an encendidos. En cada paso se enciende un bombillo, pero solo si est\'a junto a uno ya encendido. Calcule de cu\'antas formas distintas se pueden encender todos.
	\item ¿De cu\'antas formas se puede particionar un conjunto de n elementos en $j_1$ subconjuntos de tama\~no 1, $j_2$ de tama\~no 2 y as\'i hasta $j_k$ de tama\~no k?
	\item Obtenga una expresi\'on para $(x_1 + ... + x_k) ^ n$
\end{enumerate}
\end{document}
