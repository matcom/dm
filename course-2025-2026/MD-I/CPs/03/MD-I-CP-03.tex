\documentclass{article}
\usepackage{enumitem}
\usepackage{tikz}
\usetikzlibrary{trees}
\usepackage{amsmath} 
\usepackage{nopageno}

\begin{document}

\title{Clase pr\'actica 3}
\maketitle

\begin{enumerate}
    \item Sean $a$ y $n$ enteros mayores que $1$.
    \begin{itemize}
        \item Si $a^n - 1$ es primo $\Rightarrow$ $a=2$ y $n$ es primo.
        \item Si $a^n + 1$ es pimo $\Rightarrow$ $a$ es par y $n$ es una potencia de 2
    \end{itemize}

    \item Sea $n \in \mathbf{Z} ~ n>4$. Demuestra que $n|(n-1)!$ si y solo si $n$ es compuesto.
    
    \item Encuentre la descomposición canónica de $20!$.
    
    \item Demustra que existe un bloque de 2022 enteros consecutivos donde exactamente 15 de ellos son primos.

    \item Demuestre que si $n \in \mathbf{Z^+}$ entonces $2^{2^n} - 1$ tiene al menos $n$ divisores primos distintos.
     
    \item Demuestre que si $p$ y $p^2+2$ son primos entonces $p^3+2$ es primo.
    
    \item Sea $p_n$ el n-\'esimo primo. Demuestre que $p_n \leq 2^{2^{n-1}}$.
    
    \item Demuestre que si $n > 2$, entonces existe un primo $p$ que satisface $n < p < n!$.
\end{enumerate}
\end{document}