\documentclass{article}
\usepackage{enumitem}
\usepackage{tikz}
\usetikzlibrary{trees}
\usepackage{amsmath} 
\usepackage{amssymb}
\usepackage{nopageno}

\title{Clase pr\'actica 61}

\begin{document}

\maketitle
\begin{enumerate}
    \item Diga si $712!$ + 1 es primo.
    \item Encuentre todos los primos $p$ tales que la suma de todos los divisores enteros positivos de $p^4$ es igual al cuadrado de un entero.
    \item Sea $n$ un entero impar que no es múltiplo de 5, entonces $n$ divide a un entero cuyos dígitos son todos iguales a 1.
    \item Sean $a, n$ enteros tales que $n > 1$ y $(a, n) = 1$. El orden de $a$ módulo $n$ es el menor entero positivo $k$ tal que $a^k \equiv 1 \mod n$. Se denota como $ord_n(a) = k$. Demuestre que si $ord_n(a) = k$, entonces $a^h \equiv 1 \mod n$ si y solo si $k | h$.
    \item Si $ord_n(a) = k$, entonces $a^i \equiv a^j \mod n$ si y solo si $i \equiv j \mod k$.
    \item Sea $p$ un primo mayor que 5. Demuestre que $(p-1)! + 1$ tiene al menos dos divisores primos diferentes.
    \item Sea $n$ un entero positivo. Demuestre que $\sum_{d = 1}^{n} \phi(d) * \lfloor \frac{n}{d} \rfloor = \frac{n*(n+1)}{2}$.
    \item Sean $a,n$ enteros con $(a, n) = 1$ y $ord_n(a) = \phi(n)$, entonces se dice que $a$ es una raíz primitiva de $n$. Demuestre que si $a$ es una raíz primitiva de $n$, entonces $a^1, a^2, ..., a^{\phi(n)}$ es un sistema residual reducido módulo $n$.
\end{enumerate}
\end{document}