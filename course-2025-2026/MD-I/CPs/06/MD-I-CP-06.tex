\documentclass{article}
\usepackage{enumitem}
\usepackage{tikz}
\usetikzlibrary{trees}
\usepackage{amsmath} 
\usepackage{amssymb}
\usepackage{nopageno}

\title{Clase pr\'actica 6}

\begin{document}

\maketitle
\begin{enumerate}
    \item  Si $f$ es una función multiplicativa y no es idénticamente nula entonces $f(1) = 1$.
    \item Si $n>1$ se escribe como $n = p_1^{e_1} *...* p_k^{e_k}$, entonces todos los divisores positivos de $n$ se escriben como $d = p_1^{a_1} *...* p_k^{a_k}$, con $0 \leq a_i \leq e_i$.
    \item Sean $n,m$ enteros positivos, tales que $(n, m) = 1$. Entonces todo divisor de $n*m$ se escribe como $d = d_1 * d_2$, en donde $d_1 | n$ y $d_2 | m$, además se cumple $(d_1, d_2) = 1$, y todos estos productos son diferentes.
    \item Sea $\tau(n)$ para $n$ positivo, la cantidad de divisores positivos de $n$. Demuestre que esta función es multiplicativa. Encuentre una expresión para calcularla.
    \item Sea $\sigma(n)$ para $n$ positivo, la suma de los divisores positivos de $n$. Demuestre que esta función es multiplicativa. Encuentre una expresión para calcularla.
    \item Sea $n>1$ demuestre que la multiplicación de todos los divisores positivos de $n$ es igual a $n^{\tau(n)/2}$.
    \item Demuestre que si $f$ es una función multiplicativa y $F(n) = \sum_{d | n} f(d)$, entonces $F$ es multiplicativa.
    \item Demuestre que para $n \geq 1$ se cumple que $\sum_{d|n}\phi(d) = n$.
    \item *********** Demuestre que $\phi(n) =  \sum_{d|n} \frac{n}{d} \mu(d)$.
\end{enumerate}
\end{document}