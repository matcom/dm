\documentclass{article}
\usepackage{enumitem}
\usepackage{tikz}
\usetikzlibrary{trees}
\usepackage{amsmath} 
\usepackage{nopageno}

\begin{document}

\title{Clase pr\'actica 1}
\maketitle

\begin{enumerate}
    \item Sea $k \in \mathbf{Z^+}$. Demuestre que $k$ divide a todo producto de $k$ enteros consecutivos. 
    \begin{enumerate}
        \item Demuestre que $k!$ divide al producto de $k$ enteros consecutivos.
    \end{enumerate}
    \item Un entero $n>1$ es especial si para todo $k \in \mathbf{Z^+}$, con $k \leq n$ se puede escribir como suma de divisores distintos de $n$. Demuestre que si $p$ y $q$ son especiales entonces $pq$ es especial.
    \item Determine el n\'umero de formas de descomponer a $n$ en sumandos donde el orden no es relevante y la diferencia modular de cualquier par de sumandos es a lo sumo $1$.
   
    \item Sea  $n \in \mathbf{Z^+}$. Demuestre que existen infinitos m\'ultiplos de $n$ que contienen a todos los d\'igitos decimales.
    \item Demuestre que con los valores 4 y 5 se puede obtener cualquier valor $n$, $n\geq 12$ como sumas de esos números, ($n = 4p + 5q$)
    
    \item Todo polígono simple con $n$ lados, $n\ge3$ puede ser triangulado en $n-2$ triángulos.
    \begin{itemize}
        \item Nota 1: Un polígono es simple si cualquiera de sus lados no consecutivos no se intersectan.
        \item Nota 2: El proceso de triangulación de un polígono se obtiene al dividirlo en triángulos agregando diagonales que no se intersectan.
        \item Nota 3: Se cumple que en cualquier polígono simple con al menos 4 lados, existe una diagonal interior.
    \end{itemize}

     \item Se tiene una matriz de $128 * 128$, demuestre que, si se quita una cuadr\'icula aleatoria entonces se puede completar la matriz utilizando cuadr\'iculas en forma de $L$ de tamaño 3.

      \item Se tiene el polinomio $p(x) = (x-a)(x-b)(x-c)...(x-y)(x-z)$, se cumple que $a+b+c+...+y+z=100$. Calcule $p(1024)$.
\end{enumerate}
\end{document}