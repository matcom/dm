\documentclass[a4paper,1pt]{report}
\usepackage[utf8]{inputenc}
\usepackage[spanish]{babel}
\usepackage{amsfonts}
\usepackage{amsthm}
\usepackage{amssymb}
\usepackage{amsmath}
\usepackage{graphicx}
\usepackage{subcaption}
\usepackage{float}
\usepackage[rightcaption]{sidecap}
\usepackage[pdf]{graphviz}

\newtheorem*{pbo}{Principio del Buen Ordenamiento}

\newtheorem*{pim}{Principio de Inducción Matemática}

\newtheorem*{teo}{Teorema}

\newtheorem*{cor}{Corolario}

\newtheorem*{dem}{Demostración}

\newtheorem*{dfn}{Definición}

\newtheorem*{lem}{Lema}

\newtheorem*{prp}{Propiedades}


% Title Page
\title{Conferencia 11 - Funciones Primitivo Recursivas}
\author{}

\begin{document}
\maketitle

\begin{dfn}
 Sean \\
 
 $K_0=\{0_1,0_2,dots\}$ la familia de funciones nulas donde $o_n(X)=0$ y $X=(x_1,x_2,\dots,x_n)$
 
 $K_1=\{U^1_1, U^2_1,U^2_2,U^3_1,U^3_3,U^3_3,\dots\}$~(proyección) donde $U^n_k(x_1,x_2,\dot,x_n)=x_k$ es la familia de funciones proyectivas
 
 $K_2-\{suc\}$, $suc:\mathbb{N}\rightarrow\mathbb{N}$, $suc(x)=x+1$, función sucesor\\
 
 entonces $K_1\cup K_2\cup K_3$ se conoce como la familia de funciones iniciales
\end{dfn}


\textbf{Esquemas de Recursión}

\begin{dfn}[Esquema de Composición]
Sean

$a_i:\mathbb{N}^n\rightarrow\mathbb{N}$  $1\leq i \leq R$

$b:\mathbb{N}^r\rightarrow\mathbb{N}$

$f:\mathbb{N}^n\rightarrow\mathbb{N}$\\

entonces $f(x)=(a_1(X),a_2(x)),\dots,a_R(X)$ donde $X=(x_1,x_2,\dots,x_n)$
 
\end{dfn}



\begin{dfn}[Esquema de Recursión]
Existen dos esquemas:

 \begin{enumerate}
  \item Sea $a$ constante y $b:\mathbb{N}^2\rightarrow\mathbb{N}$
  es posible definir entonces $b:\mathbb{N}\rightarrow\mathbb{N}$ tal que
  
  $h(0)=a$ y $h(y+1)=b(y,h(y))$
  \item Sea $n\geq 1$, $a:\mathbb{N}^n\rightarrow\mathbb{N}$ y $b:\mathbb{N}^{n+2}\rightarrow\mathbb{N}$
  es posible definir entonces $b:\mathbb{N}^{n+1}\rightarrow\mathbb{N}$ tal que
  
  $h(x,0)=a(X)$ y $h(x,y+1)=b(x,y,h(x,y))$ donde $X=(x_1,x_2,\dots,x_n)$
 \end{enumerate}

\end{dfn}


\begin{dfn}
 Una función se llama primitivo-recursiva si es una función inicial o si es posible definirla en términos de funciones iniciales por un número finito de aplicaciones de los esquemas composición o recursión
\end{dfn}

\textbf{Ejemplos}

\begin{enumerate}
 \item \textbf{Suma}
 
 $suma(x,y)$, $suma:\mathbb{N}^2\rightarrow\mathbb{N}$
 
 se tienen $a,b$ tales que $a:\mathbb{N}\rightarrow\mathbb{N}$ y $b:\mathbb{N}^3\rightarrow\mathbb{N}$
 
 se define entonces $suma(x,0)=a(x)=U^1_1(x)$
 
 y $suma(x,y+1)=b(x,y,suma(x,y))=suc(U^3_3(x,y,suma(x,y)))$
 
 \item \textbf{Producto}
 
 $prod(x,y)$, $prod:\mathbb{N}^2\rightarrow\mathbb{N}$
 
 se tienen $a,b$ tales que $a:\mathbb{N}\rightarrow\mathbb{N}$ y $b:\mathbb{N}^3\rightarrow\mathbb{N}$
 
 se define entonces $prod(x,0)=a(x)=O_1(x)$
 
 y $prod(x,y+1)=b(x,y,prod(x,y))=suma(U^3_1(x,y,z),U^3_3(x,y,z))$ donde $z=prod(x,y)$
 
 \item \textbf{Potencia}
 
 $pot(x,y)$, $pot:\mathbb{N}^2\rightarrow\mathbb{N}$
 
 se tienen $a,b$ tales que $a:\mathbb{N}\rightarrow\mathbb{N}$ y $b:\mathbb{N}^3\rightarrow\mathbb{N}$
 
 se define entonces $pot(x,0)=a(x)=suc(O_1(x))$
 
 y $pot(x,y+1)=b(x,y,pot(x,y))=prod(U^3_1(x,y,z),U^3_3(x,y,z))$ donde $z=pot(x,y)$
 
\end{enumerate}

\begin{teo}
La función constante $C^n_k(x_1,x_2,\dots,x_n)=k$ es primitivo-recursiva
\end{teo}

\begin{dem}
 Demostremos por inducción en $k$.
 
 \textbf{Caso base:} k=0
 
 $C^n_0(x_1,x_2,\dots,x_n)=O_n(x_1,x_2,\dots,x_n)=0$\\
 
 \textbf{Paso inductivo} Si se cumple para $C^n_k$ entonces se cumple para ¨$C^n_{k+1}$
 
  $C^n_{k+1}(x_1,x_2,\dots,x_n)=suc(C^n_k(x_1,x_2,\dots,x_n))=k+1$
\end{dem}

\begin{teo}
 Sean las funciones $f:\mathbb{N}^k\rightarrow\mathbb{N}$ y $h:\mathbb{N}^n\rightarrow\mathbb{N}$ y
 
 $h(x_1,x_2,\dots,x_n)=f(x_{i_1},x_{i_2},\dots,x_{i_k})$ donde $x_{i_1},x_{i_2},\dots,x_{i_k}$ es un secuencia de variables con posible repetición tomada de $x_1,x_2,\dots,x_n$, entonces si $f$ es primitivo recursiva tam bién lo es $h$
\end{teo}

\begin{dem}


\end{dem}
 $h(x_1,x_2,\dots,x_n)=f(U^n_{i_1}(x_1,\dots,x_n),U^n_{i_2}(x_1,\dots,x_n),\dots,U^n_{i_k}(x_1,\dots,x_n))$
 
 \begin{cor}
  Si $f(x,y)$ es primitivo-recursiva entonces lo es $h$ tal que:
  \begin{itemize}
   \item $h(x,y)=f(y,x)$
   \item $h(x)=f(x,x)$
   \item $h(x,y,z) = f(y,z)$
  \end{itemize}

 \end{cor}
 
 \begin{dfn}
  Sea $P$ un predicado numérico $k$-ario ($x\in\mathbb{N}^k$) ,
  
  la función
  
  \begin{equation}
C_P(x) = 
\begin{cases}
1 & \text{si } P(x) \text{se cumple} \\
0   & \text{si } P(x) \text{no se cumple} 
\end{cases}
\end{equation}

se llama función característica de $P$
 \end{dfn}
 
\begin{dfn}
 Se dice que $P$ es un predicado primitivo-recursivo si su función característica es primitivo-recursiva
\end{dfn}

\begin{teo}
 Si los predicados $P_1$ y $P_2$ son primitivo recursivos, entonces también lo son:
 \begin{itemize}
  \item $\neg P_1$
  \item $P_1\vee P_2$
  \item $P_1\wedge P_2$
  \item $P_1\Rightarrow P_2$
  \item $P_1\Leftrightarrow P_2$
 \end{itemize}

\end{teo}

\begin{teo}
 Sea $P_1,P_2,\dots,P_k$ predicados $n$-arios, considere la función
 \begin{equation}
f(x_1,x_2,\dots,x_n) = 
\begin{cases}
g_1(x_1,x_2,\dots,x_n) & \text{si } P_1(x_1,x_2,\dots,x_n) \\
g_2(x_1,x_2,\dots,x_n) & \text{si } P_2(x_1,x_2,\dots,x_n) \\
\dots\\
g_k(x_1,x_2,\dots,x_n) & \text{si } P_k(x_1,x_2,\dots,x_n) 
\end{cases}
\end{equation}
 si para todo $i$ $1\leq i \leq k$ $g_i$ es una función primitivo-recursiva y para cada $x_1,x_2,\dots,x_n$ se cumple exactamente uno de los predicados, entonces $f$ es primitivo recursiva.
\end{teo}

\begin{dfn}[Suma acotada]
 Sea $x=(x_1,x_2,\dots,x_n)$ y $f:\mathbb{N}^{n+1}\rightarrow\mathbb{N}$, se llama suma acotada a la función $h:\mathbb{N}^{n+1}\rightarrow\mathbb{N}$ tal que 
 
 $h(x,y)=\sum_{z<y}f(x,z)$ o sea
 \begin{equation}
h(z,y) = 
\begin{cases}
0 & y=0 \\
f(x,0) + f(x,1) + \dots + f(x,y-1) & y>0 
\end{cases}
\end{equation}
\end{dfn}

\begin{dfn}[Producto acotado]
 Sea $x=(x_1,x_2,\dots,x_n)$ y $f:\mathbb{N}^{n+1}\rightarrow\mathbb{N}$, se llama producto acotado a la función $h:\mathbb{N}^{n+1}\rightarrow\mathbb{N}$ tal que 
 
 $h(x,y)=\prod_{z<y}f(x,z)$ o sea
 \begin{equation}
h(z,y) = 
\begin{cases}
1 & y=0 \\
f(x,0)f(x,1)\dots f(x,y-1) & y>0 
\end{cases}
\end{equation}
\end{dfn}

\begin{teo}
 Si $f:\mathbb{N}^{n+1}\rightarrow\mathbb{N}$ es primitivo recursiva entonces la suma acotada de $f$ y el producto acotado de $f$ son primitivo recursivos
\end{teo}

\begin{teo}
 Sean $f:\mathbb{N}^{n+1}\rightarrow\mathbb{N}$ y $k:\mathbb{N}^{n+1}\rightarrow\mathbb{N}$ funciones primitivo recursivas entonces son primitivo recursivas las funciones 
 
 $\sum_{z<k(x,y)}f(x,z)$ y $\prod_{z<k(x,y)}f(x,z)$
\end{teo}

\begin{teo}
 Sean $f:\mathbb{N}^{n+1}\rightarrow\mathbb{N}$ una función primitivo recursiva y $g:\mathbb{N}^{n+1}\rightarrow\mathbb{N}$, entonces
 
 \begin{equation}
g(x,y) = \mu_{z<y}(f(x,z)=0) 
\begin{cases}
\text{el menor } z<y \text{ tal que }f(x,z)=0\\
y
\end{cases}
\end{equation}

es primitivo recursiva
 
\end{teo}


\begin{cor}
 Si  $f:\mathbb{N}^{n+1}\rightarrow\mathbb{N}$ y $k:\mathbb{N}^{n+1}\rightarrow\mathbb{N}$ son primitivo-recursivas entonces también lo es $\mu_{z<k(x,y)}(f(x,z)=0) $
\end{cor}

\begin{teo}
Sea $P(x,y)$ un predicado primitivo-recursivo, entonces la función 
\begin{enumerate}
 \item $f(x,y)=\mu_{z<y}(P(x,z))$
 \item $\forall(z) z<y, P(x,z)$ y $\exists(z)z<y, P(x,z)$ son predicados primitivo-recursivos
\end{enumerate}


\end{teo}

  
\end{document} 
