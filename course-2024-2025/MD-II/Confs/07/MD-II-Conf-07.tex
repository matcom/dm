\documentclass[a4paper,1pt]{report}
\usepackage[utf8]{inputenc}
\usepackage[spanish]{babel}
\usepackage{amsfonts}
\usepackage{amsthm}
\usepackage{amssymb}
\usepackage{amsmath}
\usepackage{graphicx}
\usepackage{subcaption}
\usepackage{float}
\usepackage[rightcaption]{sidecap}

\newtheorem*{pbo}{Principio del Buen Ordenamiento}

\newtheorem*{pim}{Principio de Inducción Matemática}

\newtheorem*{teo}{Teorema}

\newtheorem*{cor}{Corolario}

\newtheorem*{dem}{Demostración}

\newtheorem*{dfn}{Definición}

\newtheorem*{lem}{Lema}

\newtheorem*{prp}{Propiedades}


% Title Page
\title{Conferencia 7 - Emparejamiento}
\author{}

\begin{document}
\maketitle

\begin{dfn}
 Sea $G=<V,E>$ un grafo, dos aristas de G son independientes sin no tienen un vértice en común.
\end{dfn}

\begin{dfn}
 Un emparejamiento~(\textit{matching}) en G es un conjuntos de aristas independientes dos a dos.
\end{dfn}

\begin{dfn}
 Un emparejamiento M se dice que satura a un vértice v de G si v es un extremo de una arista en M.
\end{dfn}

\begin{dfn}
 Se dice que M satura a $X\subset V(G)$ si satura a todo v tal que $v\in X$
\end{dfn}

\begin{dfn}
 Si M satura a V(G) entonces se dice que M es un emparejamiento perfecto.
\end{dfn}

\begin{dfn}
 Un emparejamiento es maximal si no se puede adicionar ninguna arista a M que sea independiente con todas las demás posibles.
\end{dfn}

\begin{dfn}
 Un emparejamiento es máximo si tiene la mayor cardinalidad entre todos los emparejamientos
\end{dfn}

\begin{dfn}
 Un emparejamiento es  perfecto si todos los vértices del grafo están presentes en las aristas del emparejamiento.
\end{dfn}

\begin{dfn}
 Sea G un grafo y M un emparejamiento de G:
 \begin{itemize}
  \item Un camino M-alternativo es un camino simple cuyas aristas alternan entre aristas que están en M y aristas que no están en M
  \item Si el camino M-alternativo comienza y termina en vértices no saturados por M se dice que es un camino M-incremento
 \end{itemize}
\end{dfn}


\begin{lem}
 En un grafo G conexo tal que para todo vértice  v de G se tiene que el grado de v es menor o igual a 2 entonces G es un camino simple o un ciclo.
\end{lem}

\begin{dem}
 
\end{dem}


Tomemos el camino simple máximo de G, $c=<v_1,v_2,\dots,v_{k-1},v_k>$

Entonces para todo i, $2\leq i \leq k$ se tiene que $deg(v_i)=2$ por tanto no existe $w\in V(G)$ tal que  $w\neq v_{i-1}\wedge w\neq v_{i+1}\wedge \{v_i,w\}\in E(G)$ puesto que si existiera el grado de $v_i$ sería mayor que 2 lo que sería una contradicción.

Si existe $w\in V(G)$ tal que $w\neq c$ como G  es conexo entonces w sería adyacente a $v_1$ o a $v_k$ lo que también serían una contradicción porque c es maximal.

Luego, para todo vértice de G se tiene que este pertenece a c.

Además, como se vio que para todo i, $2\leq i \leq k$ se tiene que $deg(v_i)=2$, por  tanto las aristas involucradas en estos vértices están presentes en el camino.

Ahora, lo que más pudiera suceder es que exista una arista $\{v_1,v_k\}$ con lo que c sería un ciclo, o si no existe  en cuyo caso seguiría siendo un camnio simple.  Luego G es un camino simple o un ciclo $\blacksquare$.


\begin{teo}
 Sea $G=<V,E>$ un grafo, un emparejamiento M de G es máximo si y solo si no existen en G caminos M-incremento
\end{teo}



\begin{dem}
La demostración es equivalente a demostrar que M no es máximo si y solo si existe algún camino M-incremento en G. 
\end{dem}

Veamos en el sentido; si existe algún camino M-incremento entonces M no es máximo.

Si existe algún camino M-incremento este será de la forma:

\textit{no-si-no-si-...-no-si-no}

Las aristas no de este camino distintas a las de los extremos contienen vértices que están saturados por el emparejamiento.

Si en el emparejamiento se sustituyen las aristas \textit{si} por las \textit{no}, se obtiene uno nuevo que tiene mayor tamaño, puesto que las aristas \textit{no}  exceden en 1 a las \textit{si}.

Esto se puede realizar sin problemas pues ningún vértice del emparejamiento anterior está en el camino, lo que implica que las aristas no aparecían en él no se ven comprometidas.\\

En el otro senido, si M no es máximos entonces existe algún camino M-incremento.



En la demostración se usa también la operación diferencia simétrica:

$A\bigtriangleup B = (A\cup B) / (A\cap B)$

Entonces, volvamos a la demostración:

Como M no es un emparejamiento máximo entonces existe un emparejamiento P tal que $|P|>|M|$.

Tomemos el conjunto $P\bigtriangleup M$ en el cual hay al menos una arista pues \\
$|P|>|M|$. Entonces tomemos todas las componentes conexas $P\bigtriangleup M$ que contienen aristas. 

Note que todos los vértices de $P\bigtriangleup M$ tienen grado menor o igual a 2:
\begin{itemize}
 \item Puede aparecer en dos aristas distintas, una de P y una de M, en cuyo caso tendría grado 2
 \item Puede aparecer en una arista solamente de alguno entre P o M, de modo que dicha arista estaría en $P\bigtriangleup M$ y su grado sería 1
 \item Puede aparecer en una arista que es común en P y M, luego no estaría en el conjunto, y por tanto su grado sería 0.
\end{itemize}

Luego, cada componente conexa de las mencionadas es o bien un camino simple o un ciclo. 

Note también que en ambos casos las aristas se alternan entre una que pertenece a P y otra que pertenece a M.

Como $|P|>|M|$ entonces en $P\bigtriangleup M$ hay más aristas de P que de M, por tanto existe una componente conexa que tiene más aristas de P que de M. Esta componente no puede ser un ciclo pues tendría que tener dos aristas de P concatenadas lo que no puede ocurrir. Luego esta componente conexa sería un camino simple que empieza y termina en aristas de P que no pertenecen a M y va alternando entre aristas de M y aristas de P, que no están en M, luego este camino es M-incremento $\blacksquare$.

\begin{dfn}
 Sea S un subconjunto de V(G), entonces se denota $N(S)=\{v|\{u,v\} \in E(G) u\in S\}$
\end{dfn}

\begin{teo}
 \textbf{Teorema de Hall} Sea $G=<X\cup Y, E>$ un grafo bipartito con conjuntos X,Y entonces existe un emparejamiento que satura a X (emparejamiento completo) si y solo si para todo $S\subseteq X$ se cumple que $|N(S)|\geq |S|$
\end{teo}

\begin{dem}
 
\end{dem}


En el sentido directo

Tomemos S, un subconjunto cualquiera de X, como en G hay un emparejamiento que satura a X, entonces la menos para cada vértice v de S existe un vértice w en Y tal que $\{v,w\}$ pertenece al emparejamiento y por tanto a N(S), luego N(S) tendrá al menos la misma cantidad de vértices que S, por tanto $|N(S)|\geq |S|$\\


En el otro sentido:

Supongamos que $|N(S)|\geq |S|$ y que M es un emparejamiento máximo de G. Ahora, supongamos que existe un vértice $u\in X$ que no está saturado por M, entonces definimos un conjunto A formado por los vértices de G que pueden ser conectados con u a través de un camino alternado. Se tiene $S=A\cap X$ y $T=A\cap Y$. Como M es máximo no hay caminos M-incremento. 

Por otra parte, por definición de T, todos los vértices de T pueden ser conectados con u por caminos alternados que empiezan por tanto con una arista que no pertenece a M (puesto que u no está saturado por M). Entonces todos los vértices de T han de estar emparejados por M con un vértice de $S- \{u\}$ (ya que de lo contrario, el camino alternado que les uniría con u sería un camino M-incremento, el cual no puede existir).

De igual manera, por definición de S, todos los vértices de S distintos de u pueden ser conectados con u a través de un camino alternado.  Como $u\in S$ pero no está saturado por M, para que dichos caminos sean alternados, deben llegar a los vértices de S por aristas pertenecientes a M.

De aquí deducimos que todo vértice de S, a excepción de u, está emparejado por M con un vértice de T. Por tanto, por definición de los conjuntos A, S y T, se llega a establecer una biyección entre el conjunto T y $S-\{ u \}$, luego $|T| = |S| - 1$ (puesto que $u \in S$).

Veamos ahora que $N(S)\subseteq  T$.

Sea $b\in N(S)$, entonces existe $a\in S$ tal que $\{a,b\}\in E(G)$. Como $a\in S$ existe un camino alternado que une a con u. Sea $c\in T$ tal que la última arista de dicho camino es $\{a,c\}\in M$ (esto porque no hay camino M-incremento), pueden darse dos situaciones:
\begin{itemize}
 \item Si $b=c$ entonces $\{a,b\}\in M$ y  $b\in T$
 \item Si $b\neq c$ entonces $\{a,b\}\not \in M$~(ya que a está emparejado con c por M) y por tanto $b\in T$ porque podría ser conectado con u a través del camino alternado que
une a con u añadiéndole la arista $\{a,b\}\not \in M$
\end{itemize}

Entonces tenemos que $|N(S)| \leq |T| = |S| - 1 < |S|$, lo cual es una contradicción con
nuestra hipótesis de partida ($|N(S)| \geq |S|$). Por tanto, concluimos que ese vértice $u$ no
saturado por M no puede existir $\blacksquare$.


\begin{cor}
 Sea $G=<X\cup Y, E>$ un grafo bipartito con conjuntos X,Y, $|X|\leq|Y|$. Si existe un número natural k tal que para todo $x\in X$ y para todo $y\in Y$, $deg(x)\geq k$ y $deg(y)\leq k$ entonces en G existe un emparejamiento que satura a X.
\end{cor}

\newpage

\begin{dem}
 
\end{dem}


Sea A un subconjunto de X. Como por hipítesis $deg(x)\geq k$ para todo vértice
$x \in X$, y $A \subseteq X$, entonces hay al menos $k|A|$ aristas con un extremo en A. 
El otro extremo de estas aristas está en N(A). 

Como, además, por hipótesis $deg(y) \leq k$ para todo $y \in Y $,
entonces también se cumple que todo $y \in N(A)$ es incidente en como mucho k aristas (ya
que al ser G bipartito $N(A) \subseteq Y$ ). Se puede ver que el número de vértices de N(A) es al menos $\frac{k|A|}{k} = |A|$. Aplicando entonces el Teorema de Hall, G tiene un emparejamiento completo de X en Y $\blacksquare$.

\begin{cor}
 Sea $G=<X\cup Y, E>$ un grafo bipartito regular de grado $r\geq 1$  entonces G
tiene un emparejamiento perfecto.
\end{cor}


\begin{dem}
 
\end{dem}


Si G es bipartito y regular con $r\leq 1$, entonces $r|X| = r|Y|$ y por tanto $|X|=|Y|$.
Sea $S \subseteq X$ un subconjunto no vacío cualquiera de X. Sea $E_1$ el conjunto de las
las aristas incidentes en S y $E_2$ el conjunto de las aristas incidentes en N(S). Por definición de N(S) tenemos que $E_1\subseteq E_2$. Entonces, si $|E_1|=r|S|$ y 
$|E_2|=r|N(S)|$ tenemos que $r|N(S)|\geq r|S|$ por lo que $|N(S)|\geq |S|$ $\blacksquare$.
















\end{document} 
