\documentclass[a4paper,1pt]{report}
\usepackage[utf8]{inputenc}
\usepackage[spanish]{babel}
\usepackage{amsfonts}
\usepackage{amsthm}
\usepackage{amssymb}
\usepackage{amsmath}
\usepackage{graphicx}
\usepackage{subcaption}
\usepackage{float}
\usepackage[rightcaption]{sidecap}

\newtheorem*{pbo}{Principio del Buen Ordenamiento}

\newtheorem*{pim}{Principio de Inducción Matemática}

\newtheorem*{teo}{Teorema}

\newtheorem*{cor}{Corolario}

\newtheorem*{dem}{Demostración}

\newtheorem*{dfn}{Definición}

\newtheorem*{lem}{Lema}

\newtheorem*{prp}{Propiedades}


% Title Page
\title{Conferencia 4 - Grafo Hamiltoniano}
\author{}



\begin{document}
\maketitle

\begin{dfn}[Cadena Hamiltoniana] 
Una cadena en un grafo G se dice Hamiltoniana si contiene a
todos los vértices del grafo.
\end{dfn}

\begin{dfn}[Ciclo Hamiltoniano]
Un ciclo c en un grafo G se dice Hamiltoniano si contiene a todos los vértices del grafo.
\end{dfn}

\begin{dfn}[Grafo Hamiltoniano]
Un grafo G es un grafo Hamiltoniano si tiene un ciclo de Hamilton
\end{dfn}

\begin{teo}[Teorema de Dirac]
 Sea G un grafo con $|V(G)|=n,\, n\geq3$, si para todo $v,\, v\in V(G)$ se tiene que $deg(v)\geq \frac{n}{2}$ entonces G es Hamiltoniano.
\end{teo}

\begin{teo}[Teorema de Ore]
Sea G un grafo con $|V(G)|=n,\, n\geq3$, si para todo par $v,w$ de vértices no adayacentes se cumple que $deg(v)+deg(w)\geq n$ entonces G es Hamiltoniano.
\end{teo}
 
 \begin{dem}[Demostración del \textbf{Teorema de Dirac}] \end{dem}
 Vea que si $deg(v)\geq \frac{n}{2}$ para todo v de G, entonces\\ 
 
 $deg(v)+deg(w)\geq \frac{n}{2}+\frac{n}{2}=n$ \\
 
 y si se cumpliera el \textbf{Teorema de Ore}, como ya se tiene su condición, entonces sería Hamiltoniano. Luego, solo habría que demostrar el \textbf{Teorema de Ore}.\\
  
 Para demostrar el \textbf{Teorema de Ore} se utilizarán la definición de clausura y el \textbf{Teorema de Bondy-Chvátal}:
 
  \begin{dfn}
 Dado G, $|V(G)|=n$ se define inductivamente la secuencia $G_0,G_1,\dots,G_k$ de grafos donde
 $G_0=G$ y $G_{i+1=G_i+\{x,y\}}$ donde $x,y$ son vértices no adyacentes en $G_i$ tal $deg(x)+deg(y)\geq n$, entonces $G_k$ es la clausura ge $G$
 \end{dfn}
 
 O escrita de otra forma:
 
  \begin{dfn}
  Dado un grafo G con n vértices, la clausura de G es el grafo que tiene los mismos vértices que G y que aparece al agregar todas las aristas de la forma $\{u, v\}$ para cualquier par de vértices u y v que no sean adyacentes y cumplan que 
  $deg(v) + deg(u)\geq n$.
 \end{dfn}
 
 \begin{teo}[Teorema de Bondy-Chvátal]
  G es Hamiltoniano si y solo si su clausura es Hamiltoniana.
 \end{teo}

  \begin{dem}[Demostración del \textbf{Teorema de Bondy-Chvátal}] \end{dem}

En el sentido directo, G es Hamiltoniano $\Rightarrow$ su clausura es Hamiltoniana, la demostración es obvia. Si G es Hamiltoniano entonces cualquier ciclo Hamiltoniano sigue existiendo en la clausura de G porque las aristas que se añaden al grafo original no afectan el ciclo (solo conectan vértices no adyacentes).\\
  
En el otro sentido, si la clausura de G es Hamiltoniana $\Rightarrow$ G es Hamiltoniano, si se parte de $G=G_0$ hasta llegar a $G_k$, clausura de G, bastaría con demostrar que $G_i$ es Hamiltoniano ssi $G_{i+i}$ también lo es.

Luego, si $G_i$ es Hamiltoniano es obvio que $G_{i+1}$ también lo es.

Ahora, veamos que pasa si $G_{i+1}$ es Hamiltoniano.

Si en $G_{i+1}$ hay un ciclo de Hamilton que no contiene a la arista $\{x,y\}$ (la agregada que no estaba en $G_i$), entonces este ciclo también aparecía en $G_i$.

Suponga que $\{x,y\}$ si aparece en el ciclo, por tanto en el grafo $G_i$, como no está $\{x,y\}$ habrá un ciclo Hamiltoniano: 

$c=<v_0,v_1,\dots,v_{n-1}>$ donde $v_0=x$ y $v_{n-1}=y$

obviamente todos los vértices que son adaycentes a x o a y aparecen en el camino puesto que este contiene a todos los vértices del grafo.

Suponga que no existe un vértice $v_i$ de camino que sea adyacente a y, y que $v_{i+1}$ sea adyacente a x. 

Luego, se conoce que $deg(y)\leq n-1$ y, además, sabemos que y al menos no tiene de vértices adyacentes la misma cantidad de vértices que son adyacentes a x (por cada vértice en el camino a x se sabe que el anterior no es adyacente a y) por tanto $deg(y)\leq n-1 - deg(x)$ luego $deg(y) + deg(x)\leq n-1 $. Y esto es una contradicción!! 

(como se añadió $\{x,y\}$  a $G_{i+1}$ entonces $deg(x)+deg(y)\geq n$)

Entonces lo supuesto es falso, por tanto existe $v_i$ en c tal que $v_i$ es adyacente a y, y $v_{i+1}$ es adyacente a x. 

Por tanto se puede tomar el ciclo $<v_0=x,v_{i+1},\dots,v_{n-1}=y,v_i,v_{i-1},\dots,v_1,v_0=x>$ que es un ciclo Hamiltoniano.\\

Luego G es Hamiltoniano si y solo si su clausura lo es$\blacksquare$.

\begin{dem}[Demostración del \textbf{Teorema de Ore}]
 
\end{dem}

Note que si G cumple las condiciones de Ore entonces la clausura es $K_n$ y todo grafo completo es claramente Hamiltoniano, luego por el lema G es Hamiltoniano.
  
\begin{cor}[Corolario del \textbf{Teorema de Ore}]
  Si $G$ es un grafo conexo, simple y sin lazos con n vértices, con $n\geq 3$, en el cual $deg(u)+deg(v)\geq n - 1$ para todo par de vértices no adyacentes u, v, entonces $G$ posee un camino Hamiltoniano.
\end{cor}

\begin{dem}[Demostración del Corolario del \textbf{Teorema de Ore}]
 
\end{dem}

Como G es conexo, sin lazos, y con n vértices, entonces no contiene un ciclo Hamiltoniano. Ahora, si creamos el grafo $G'$ a partr de añadir el vértice $w$ y conectarlo con todos los vértices existentes se cumpliría ahora  que\\ $deg(u)+deg(v)\geq n - 1 + 2 = n + 1$ para todo par de vértices u, v no adyacentes,
luego $G'$ es Hamiltoniano por el \textbf{Teorema de Ore}.

Entonces existe un ciclo Hamiltoniano en $G'$ y este tiene que pasar por el vértice $w$, porque si no pasara por $w$ significaría que existían un ciclo en G y este, por definición, no lo tenía. Entonces como hay un ciclo Hamiltoniano en $G'$ que pasa por $w$, basta con eliminar este vértice y se tendría para G un camino Hamiltoniano $\blacksquare$.

\end{document} 
