\documentclass[a4paper,1pt]{report}
\usepackage[utf8]{inputenc}
\usepackage[spanish]{babel}
\usepackage{amsfonts}
\usepackage{amsthm}
\usepackage{amssymb}
\usepackage{amsmath}
\usepackage{graphicx}
\usepackage{subcaption}
\usepackage{float}
\usepackage[rightcaption]{sidecap}

\newtheorem*{pbo}{Principio del Buen Ordenamiento}

\newtheorem*{pim}{Principio de Inducción Matemática}

\newtheorem*{teo}{Teorema}

\newtheorem*{cor}{Corolario}

\newtheorem*{dem}{Demostración}

\newtheorem*{dfn}{Definición}

\newtheorem*{lem}{Lema}

\newtheorem*{prp}{Propiedades}


% Title Page
\title{Conferencia 8 - Grafos Dirigidos}
\author{}

\begin{document}
\maketitle

\begin{dfn}
 Un \textbf{grafo dirigido}~(digrafo) consiste en dos conjuntos V, el conjunto de los vértices, y E, el conjunto de aristas, formado ahora por pares ordenados del conjunto V.
\end{dfn}

\begin{dfn}
 Sean $v,w\in V(G)$ donde G es un digrafo, v,w son adayacentes si la arista $<v,w>\in E(G)$. Se dice que la arista $<v,w>$ es incidente desde v y que es incidente a w.
\end{dfn}

\begin{dfn}
 Sea G un digrafo y $v\in V(G)$:
 \begin{itemize}
  \item El grado exterior de v es el número de aristas incidentes desde v~(exdeg(v) o outdeg(x))
  \item El grado interior de v es el número de aristas incidentes sobre v~(indeg(v))
 \end{itemize}
\end{dfn}

\begin{teo}
 En todo digrafo se cumple que:
 \begin{itemize}
  \item $\sum_{v \in V(G)}exdeg(v)+indeg(v)=2|E|$
  \item $\sum_{v \in V(G)}exdeg(v)=\sum_{v \in V(G)}indeg(v)=|E|$
 \end{itemize}

\end{teo}

\begin{dfn}
 Un camino en un digrafo G es una secuencia de vértices de G $c=<v_1,v_2,\dots,v_k>$ tal que:
 \begin{enumerate}
  \item $k>1$
  \item $k>1$ implica que $<v_i,v_{i+1}>\in E(G)\, (1\leq i \leq k-1)$
 \end{enumerate}
Luego:
\begin{itemize}
 \item El camino es simple si no se repiten vértices
 \item Si $v_1=v_k$ el camino es cerrado
 \item Un ciclo es un camino cerrado donde solo se repiten el primer y últino vértice
\end{itemize}

\end{dfn}

\begin{dfn}
 El \textbf{grafo subyacente} es el multigrafo que resulta de quitar la orientación de las aristas de un digrafo
\end{dfn}

\begin{dfn}
 Un digrafo D es conexo si el grafo subyacente de D es conexo
\end{dfn}

\begin{dfn}
 Un digrafo es \textbf{fuertemente conexo} si todo par de vértices del digrafo es mutuamente accesible, o sea si hay camino de uno al otro, y viceversa
\end{dfn}

\begin{dfn}
 Un grafo es \textbf{orientable} si es posible orientar sus aristas de modo que el digrafo resultanto sea fuertemente conexo
\end{dfn}

\begin{teo}
 Sea G un grafo conexo, entonces G es orientable si y solo si no tiene puentes 
\end{teo}

\begin{dem}
 
\end{dem}



En el sentido directo, demostraremos que si G tiene puentes entonces G es no orientable~(contrarecíproco).

Sea e una arista puente tal que $e=<u,v>$.

Como e es un puente al removerla u y v quedan en componentes conexas distintas, lo que explica que no exista camino que los conecte, por tanto  G no es fuertemente conexo luego G no es orientable. \\

En el otro sentido.

Vamos a utilizar el siguiente lema.

\begin{lem}
 Una arista es puente~(arista de corte) si y solo si no participa en ningún ciclo
\end{lem}

\begin{dem}
Demostración del lema 
\end{dem}


Demostremos que si participa en algún ciclo no es puente.

Sea $e=\{u,v\}\in E(G)$ tal que $e\in c$ de manera que c es el ciclo $c=<u,v,v_1,v_2,\dots,v_k,u>$ entonces $c_1=<v,v_1,v_2,\dots,v_k,u>$ es un camino que contiene a los vértices u,v.
Entre todo par de vértices de G, si el camino que los une contiene a la arista $\{u,v\}$, esta puede ser reemplazada por el camino $c_1$, por tanto el grafo resultante de eliminar $\{u,v\}$ no varió en la cantidad de componentes conexas, entonces  E no es una arista puente.\\

En el otro sentido, si no es una arista puente entonces participa en algún ciclo.

Sea $e=\{m,v\}\in E(G)$ tal que no es arista puente, por tanto existe otro camino que conecta a m con v que no contiene a e, 

Sea este $<m,v_1,v_2,\dots,v_k,v>$  luego $c=<m,v_1,v_2,\dots,v_k,v,m>$ es un ciclo al que pertenece e.\\

Retornemos a la demostración del teorema.\\

Sea G' el mayor subgrafo orientable de G, suponga que hay vértices de G que no pertenecen a G'.

Sea entonces v tal que pertenece a G y no pertenece a G', note que existe u que pertenece a G' tal que que $\{u,v\}\in V(G)$ pues G es conexo.

Como G no tiene puentes, por el lema anterior entonces $\{u,v\}$ pertenece a algún ciclo. Sea este $c=<u,v,v_1,v_2,\dots,v_k,u>$, sea w el primer vértice de c luego de v tal que pertenece a V(G'). Este existe debido a que u pertenece a G' y es el último vértice de c.

Entonces se tomará el camino no dirigido de u hacia w de la siguiente forma c'$=<u,v,\dots,w>$ tal que las aristas se dirijan en ese sentido.

Note que para todo vértice que pertenece a G' se puede acceder a todo vértice de c'~(puesto que G' es orientable) y cada vértice de c' puede acceder a w, lo que implica que puede acceder a todo vértice de G' y por tanto G'+c' es un subgrafo de G que es orientable y es mayor que G' lo que es una contradicción, luego el mayor subgrafo de G que es orientable es el propio G.

\begin{dfn}
 Un \textbf{torneo} es un digrafo que tiene como grafo subyacente un grafo completo, o sea, un grafo completo orientado
\end{dfn}

\begin{teo}
 En todo torneo hay un camino de Hamilton.
\end{teo}

\begin{dem}
 Demostración por inducción el número de vértices
\end{dem}

\textbf{Caso base:} Si n=2 se cumple.\\

\textbf{Paso inductivo}

Demostremos que si se cumple para n se cumple para n+1

Sea T'=T-{v}

Note que T' es un torneo, pues tiene n vértices~(hipótesis de inducción), y por tanto existe un camino de Hamilton en T'. Sea este $c=<v_1,v_2,\dots,v_n>$.

Ahora:
\begin{itemize}
 \item Si $exdeg(v)=0$ entonces $indeg(v)=n$ luego el camino $<v_1,v_2,\dots,v_n,v>$ es de Hamilton
 \item Si $exdeg(v)=n$ entonces $indeg(v)=0$ luego el camino $<v,v_1,v_2,\dots,v_n>$ es de Hamilton
\end{itemize}

Sea $exdeg(v)=k$ tal que $i\leq k < n$, entonces existe $v_i$ tal que 

$<v,v_i>\in E(T)$ y $<v_{i- 1},v>\in E(T)$, luego $<v_1,v_2,\dots,v_{i-1},v,v_i,\dots,v_n>$ es un camino de Hamilton. Este i debe existir pues todas las aristas consecutivas dentro del camino tendrán el mismo sentido, en algún momento deben cambiar pues el grado exterior es k, y por tanto el interior es n-k.

\begin{teo}
 En un digrafo D se cumple que existe un camino de longitud $X(D)-1$
\end{teo}

\begin{dem}
 
\end{dem}


Sea A un conjunto minimal de arcos tal que cuando se elimine el grafo resultante D-A sea acíclico.

Sea k la longitud del camino simple más lardo de D-A.

aplíquese la siguiente función de coloración  para los vértices de D-A, $f(v)=p$ si $p-1$ es la longitud del camino más largo desde v.

Note que para todo vértice v de D-A si $<v,w>\in E(D-A)$ entonces $f(v)\neq f(w)$

Probemos esto, supongamos lo contrario que el camino más largo desde v es igual al camino más largo desde w, ambos igual a p.

Como existe $<v,w>$ entonces en el camino desde w no aparece v, pues D-A es acíclico.
Luego, el camino $<v,w,\dots,x>$ donde $<w,\dots,x>$ es camino de mayor longitud desde w, tiene longitud p, y es más largo que desde v que era de longitud p-1, lo que es una contradicción.

Ahora volvamos, igualmente, se demuestra que para todo vértice v que pertenece a un camino simple se tiene colores distintos.

Como el camino de longitud máxima es de tamño k entonces para colorear a a D-A bastan k+1 colores.

Note que añadir una de A necesariamente crea un ciclo, puesto que A era minimal, por tanto si $e=<e,y>$ entonces en D-A existía un camino desde x hasta a y, y por tanto sus colores son distintos.

Al añadir todas las aristas de A el grafo continúa siendo k+1 coloreable, entonces $X(D)\leq k+1$  luego $k\geq X(D)-1$
  
\end{document} 
