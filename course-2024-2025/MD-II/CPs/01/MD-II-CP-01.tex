\documentclass{article}
\usepackage{enumitem}
\usepackage{tikz}
\usetikzlibrary{trees}
\usepackage{amsmath} 
\usepackage{nopageno}

\begin{document}

\title{Clase pr\'actica 1}
\maketitle

\begin{enumerate}
    \item Sea $G$ un grafo y $v, w \in V(G)$, demuestre que, si existen dos caminos diferentes que conetan a $v$ y $w$ entonces existe un ciclo en $G$.
    \item Sea $G$ un grafo,  $|V(G)| = n$, si $\forall v,w \in V(G)$ tal que $v$ y $w$ no son adyacentes se cumple que $deg(v) + deg(w) \geq n-1$, entonces $G$ es conexo.
    \item Sea $G$ un grafo tal que $|V(G)| = n$ y $|E(G)|=m$ entonces el n\'umero de componentes conexas de $G$ es mayor o igual que $n-m$.
    \item Demuestre que si $|V(G)| \geq 9$ entonces $\alpha(G)\geq 4$ (n\'umero de independencia) o $\omega(G) \geq 3$ (n\'umero de clique).
    \item Sea $G$ un grafo, tal que $|V(G)| = n$, demuestre que si $$\sum_{v \in V(G)} \binom{deg(v)}{2} > \binom{n}{2}$$ entonces en $G$ hay un ciclo de longitud $4$.
    \item Demuestre que un grafo es bipartito $ \Leftrightarrow $ no tiene ciclos de longitud impar.
    \item Demuestre que todo grafo $G$ tiene un subgrafo en expansi\'on  $G'$ que es bipartito y que cumple que $|E(G')| \geq \frac{|E(G)|}{2}$.
  \end{enumerate}
\end{document}