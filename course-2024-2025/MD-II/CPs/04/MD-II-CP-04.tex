\documentclass{article}
\usepackage{enumitem}
\usepackage{amsmath}
\usepackage{nopageno}

\begin{document}
\title{Clase pr\'actica 4}
\maketitle

\begin{enumerate}
    \item Demuestre que $K_{n, n+1}$ no puede ser hamiltoniano.
    \item Sea $G$ un grafo con al menos $3$ v\'ertices. Si $G$ tiene al menos $\binom{n-1}{2} + 1$ aristas, entonces en $G$ hay un camino de Hamilton. Si $G$ tuviera al menos $\binom{n-1}{2} + 2$ entonces podemos decir que es un grafo hamiltoniano.
    \item Sea $G$ un grafo conexo tal que en $G^c$ no existen ciclos de longitud $3$. Demuestre que en $G$ hay un camino de Hamilton.
    \item Sea un grafo $G$ con $|V(G)|=n$, si para todo par de v\'ertices $u, v$ no adyacentes se cumple que $deg(u)+deg(v) \geq n+1$ entonces si tomamos dos v\'ertices cualesquiera $a, b$, demuestre que se puede hacer un camino hamiltoniano que comienza en $a$ y termina en $b$.
    \item Sea $G$ un grafo conexo de $n$ v\'ertices y $k$ un n\'umero entero positivo menor o igual que $n$. Si para todo par de v\'ertices no adyacentes $x, y$ se cumple que $deg(x)+deg(y) \geq k$, demuestre que en $G$ hay un camino simple de longitud $k$.
    \item Sea $G$ un subgrafo abarcador de $K_{n,n}$ con $n \geq 2$, cuyas particiones son $V_1$ y $V_2$. Sean $u,v$ v\'ertices no adyacentes tales que $u \in V_1$ y $v \in V_2$ con $deg(u) + deg(v) > n$. Pruebe que $G$ es hamiltoniano si y solo si $G + uv$ lo es. 
    \item Pruebe que el grafo de Pertersen es no hamiltoniano.
\end{enumerate}
\end{document}