\documentclass[a4paper,12pt]{report}
\usepackage[utf8]{inputenc}
\usepackage{amsfonts}
\usepackage{amsthm}
\usepackage{amsmath}
\usepackage{amssymb}

\newtheorem*{pbo}{Principio del Buen Ordenamiento}
\newtheorem*{pim}{Principio de Inducción Matemática}
\newtheorem*{psm}{Principio de la Suma}
\newtheorem*{ppr}{Principio del Producto}
\newtheorem*{pie}{Principio de Inclusión - Exclusión}
\newtheorem*{ppa}{Principio del Palomar}
\newtheorem*{pin}{Principio Inyectivo}
\newtheorem*{pso}{Principio Sobreyectivo}

\newtheorem*{teo}{Teorema}

\newtheorem*{cor}{Corolario}

\newtheorem*{dem}{Demostración}

\newtheorem*{dfn}{Definición}

\newtheorem*{lem}{Lema}

\newtheorem*{prp}{Propiedades}

\newtheorem*{pro}{Proposición}

% Title Page
\title{Conferencia 9 - Combinatoria}
\author{}



\begin{document}
\maketitle

%\begin{abstract}
%\end{abstract}

\begin{pie}
Permite calcular la cardinalidad de la unión de varios conjuntos\\
 $$|\bigcup^n_{i=1}A_i|=\sum^n_{k=1}(-1)^{k+1} \underset{\underset{|I|=k}{i\in I\subseteq\{1,2,\dots,n\}}}{|\cap A_i|}$$
\end{pie}

\textbf{Ejemplo}

De 200 estudiantes 50 toman el curso de matemáticas discretas, 140 el curso de economía y 24 ambos. Como ambos cursos programaron exámenes para el día siguiente, sólo los estudiantes que no est\'en en ninguno de estos curso podrán ir a la fiesta de la noche. Se quiere ver cuántos estudiantes iran a la fiesta.

Si $A_1$ es el conjunto que estudia discreta y $A_2$ el de los que estudian economía entonces $|A_1\cup A_2|=50+140-24=166$ por lo que irán a la fiesta $200-166=34$\\


\textbf{Demostración}

Sea un elemento $x$ del universo

Si $x$ no cumple ninguna propiedad entre $A_1,A_2,\dots,A_n$ entonces no se cuenta nunca en la suma.

Ahora, verifiquemos que si aparece entonces solamente se cuenta una vez.

Si $x$ cumple exactamente una propiedad, o sea, $\exists \, i$ tal que  $x\in A_i$ pero $\forall \, j$, $j\neq i$ implica que $x\not \in A_j$, $x$ se cuenta solo una vez

Si $x$ cumple $k$ propiedades al mismo tiempo entonces:

se cuenta $k$ en $\sum_{1\leq i \leq n}|A_i|$

se cuenta ${k}\choose{2}$ en $\sum_{1\leq i < j\leq n}|A_i\cap A_j|$

$\dots$

$\dots$

se cuenta ${k}\choose{k}$ en la intersección de $k$ propiedades

No tiene sentido verificar cuántas veces se cuenta $x$ en la intersección de más de $k$ propiedades porque solo aparece en $k$ conjuntos

Entonces $x$ se cuenta:

$k$ - $k\choose 2$ + $k\choose 3$ - $\dots$ + $(-1)^{k+1}$$k\choose k$

=$k\choose 0$ + [-$k\choose 0$ + $k\choose 1$ - $k\choose 2$ + $k\choose 3$ - $\dots$ + $(-1)^{k+1}$$k\choose k$]

=$k\choose 0$$=1$




\begin{dfn}
Sea un universo tal que existen $m$ posibles propiedades $P_1,P_2,\dots,P_m$. Si $\{P_{i_i},P_{i_2},\dots,P_{i_k}\}$ es un subconjunto de las $m$ posibles propiedades se denota 
$N_{i_i,i_2,\dots,i_k}$ a la cantidad de elementos que cumplen las propiedades 
$P_{i_1},P_{i_2},\dots,P_{i_k}$. 

$s_k$ es la cantidad de veces que se cumplen $k$ propiedades
 
Entonces se define $S_0=n$ donde $n=|U|$ y 
$S_k=\sum_{1\leq i_1 < i_2 < \dots i_k \leq m} N_{i_i,i_2,\dots,i_k}$ con $1\leq k \leq m$

(Esto último sería todas las maneras en las que pueden cumplirse $k$ propiedades de $m$)
\end{dfn}


Entonces el Principio de \textbf{Inclusión - Exclusión} puede escribirse:

$|A_1\cup A_2\cup \dots \cup A_m| = S_1 - S_2 + S_3 - \dots (-1)^{m-1}S_m$

$|A_1^c\cap A_2^c\cap \dots \cap A_m^c| = S_0 - S_1 + S_2 - \dots (-1)^{m}S_m$

\begin{teo}
 \textbf{Generalización del Principio de Exclusión}. Sea $U$ un conjunto finito de elementos y $P$ un conjunto arbitrario de $m$ propiedades, el número de elementos de $U$ que poseen exactamente $R$ propiedades y $0\leq R < m$ es
 $N(R) = \sum^{m-R}_{k=0}\, (-1)^k$${k+R}\choose{k}$$S_{k+R}$
\end{teo}

\textbf{Demostración}

Sea $k$ un elemento del universo que cumple $t$ propiedades

Si $t<R$ se cuenta 0 veces

Se puede notar que la fórmula comienza por $S_R$ lo que implica que $x$ no aparece en la interesección de más de $t$ elementos

Si $t=R$ se cuenta 1 vez, solo se cuenta en $S_R$

Si $t>R$ entonces $N(R) = \sum^{t-R}_{k=0}\, (-1)^k$${k+R}\choose{k}$$S_{k+R}$ porque no se va a contar en interesecciones de más de $t$

Ahora $k+R\leq t$ por tanto el elemento se cuenta $t\choose{k+R}$ veces en $S_{k+R}$ 

por tanto $x$ se cuenta en $N(R)$
 
$\sum^{t-R}_{k=0}\, (-1)^k$${k+R}\choose{k}$${t}\choose{k+R}$

ahora ${k+R}\choose{k}$${t}\choose{k+R}$$=$${t}\choose{R}$${t-R}\choose{k}$ esto se obtiene luego de desarrollar 

las combinaciones y multiplicar y dividir por $(t-R)!$

luego $\sum^{t-R}_{k=0}\, (-1)^k$${k+R}\choose{k}$${t}\choose{k+R}$$=$$\sum^{t-R}_{k=0}\, (-1)^k$${t}\choose{R}$${t-R}\choose{k}$

$=$${t}\choose{R}$$\sum^{t-R}_{k=0}\, (-1)^k$${t-R}\choose{k}$$=0$

\begin{teo}
  Sea $U$ un conjunto finito de elementos y $P$ un conjunto arbitrario de $m$ propiedades, el número de elementos de $U$ que satisfacen al menos $R$ propiedades y $0\leq R < m$ es
 $\bar{N}(R) = \sum^{m-R}_{k=0}\, (-1)^k$${k+R-1}\choose{R-1}$$S_{k+R}$
\end{teo}

\begin{pin}
 Sean $A$ y $B$ conjuntos tal que $B$ es finito. Entonces $A$ es finito y $|A|\leq|B|$ si y solo si existe una función inyectiva de $A$ en $B$.
\end{pin}

\textbf{Demostración}

En el sentido directo, se tiene que si $|A|=m$ y $|B|=n$ entonces se tiene se tiene que existen las funciones $f:A\rightarrow\mathbb{N}_m$ y $g:B\rightarrow\mathbb{N}_n$ biyectivas. Ahora, como $m\leq n$ se tiene que $\mathbb{N}_m\subseteq\mathbb{N}_n$, luego $g^{-1} \circ f:A\rightarrow B$ es inyectiva.

Se deja propuesta la demostración en el otro sentido. 

\begin{pso}
 Sean $A$ y $B$ conjuntos tal que $B$ es finito. Entonces $A$ es finito y $|A|\leq|B|$ si y solo si existe una función sobreyectiva de $B$ en $A$.
\end{pso}

\textbf{Demostración}

En el sentido directo, se tiene que si $|A|=m$ y $|B|=n$ entonces se tiene se tiene que existen las funciones $f:A\rightarrow\mathbb{N}_m$ y $g:B\rightarrow\mathbb{N}_n$ biyectivas, con $m\leq n$. Se puede definir la función $f':\mathbb{N}_m\rightarrow A$ ahora $f'(x)=f^{-1}(min(x,n))$ y como $f$ es biyectiva $f^{-1}$ lo es también, luego también es sobreyectiva y, por tanto, $f'$ es también sobreyectiva. Entonces $f'\circ g:B¨\rightarrow A$ es sobreyectiva (la composición de funciones sobreyectivas es sobreyectiva).

En el otro sentido, se puede definir $B={b_1,b_2,\dots,b_m}$ y $f:B\rightarrow A$ sobreyectiva, entonces está bien definida la función $h:A\rightarrow B$ definida como $h(a)=b_i$, donde $i$ es el mínimo índice tal que $f(b_i)=a$. Es fácil ver que $h$ es inyectiva, Luego, por la proposición anterior  $A$ es finito y $|A|\leq|B|$.

\begin{ppa}
 Si se tiene un palomar con $n$ agujeros o casillas y $m$, $m>n$, palomas, entonces existe una casilla que contiene más de una paloma. De manera general, existe una casilla que tiene al menos $\lceil {\frac{m}{n}} \rceil$ palomas.
\end{ppa}

\textbf{Ejemplo 1}

 En todo conjunto de $n$ números enteros es posible encontrar un subconjunto tal que la suma de los elementos del subconjunto es divisible por $n$.\\

 Tomando los subconjuntos $A_1,A_2,\dots,A_n$ tal que $A_1\subset A_2 \subset\dots\subset A_n$:\\
 $A_1=\{a_1\}$\\
 $A_2=\{a_1,a_2\}$\\  
 $A_3=\{a_1,a_2,a_3\}$\\
 $\dots$\\
 $\dots$\\
 $A_n=\{a_1,a_2,\dots,a_n\}$
 
 Si hay algún $A_i$ tal que la suma de sus elementos ($S(A_i)$) cumple que $S(A_i)\equiv 0 \, (n)$ entonces ya queda demostrado.
 
 Si esto no ocurre entonces para estos conjuntos en sus sumas hay $n-1$ posibles restos entonces, por el \textbf{Principio del Palomar}, hay dos que dejan el mismo resto, o sea,
 existen $i,j$ $i\neq j$ tal que $S(A_i)\equiv S(A_j) \, (n)$ luego $S(A_i) - S(A_j)\equiv 0 \, (n)$
 
 Entonces se toma el conjunto $A_i - A_j = A_{i-j}$ asumiendo, sin perdida de generalidad, que $|A_i|>|A_j|$, luego $S(A_{i-j})\equiv 0 \, (n)$
 
 
 \textbf{Ejemplo 2}

 Sea la relación de conocerse una relación mutua, entonces en un grupo de 6 personas hay al menos 3 personas que se conocen entre ellos o al menos 3 que no se conocen entre ellos.
 
 Si se toman n casillas, entendiendo que si alguien está en la misma casilla se conocen entre ellos y si están en casillas diferentes no se conoce entre si. Para $n > 2$ cualquier distribución significa que hay al menos tres que no se conocen entre si. Ahora si $1\leq n \leq 2$
 entonces, por el \textbf{Prinicipio del Palomar}, en una casilla como mínimo habrá 3 personas y entonces habría al menos 3 personas que se conocen entre si.
 
 

\end{document}          
