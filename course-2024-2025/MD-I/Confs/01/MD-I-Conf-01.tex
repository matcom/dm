\documentclass[a4paper,1pt]{report}
\usepackage[utf8]{inputenc}
\usepackage{amsfonts}
\usepackage{amsthm}
\usepackage{amssymb}

\newtheorem*{pbo}{Principio del Buen Ordenamiento}

\newtheorem*{pim}{Principio de Inducción Matemática}

\newtheorem*{teo}{Teorema}

\newtheorem*{cor}{Corolario}

\newtheorem*{dem}{Demostración}

\newtheorem*{dfn}{Definición}

\newtheorem*{lem}{Lema}

\newtheorem*{prp}{Propiedades}

% Title Page
\title{Conferencia 1 - Principios de la Teoría de Números}
\author{}



\begin{document}
\maketitle

%\begin{abstract}
%\end{abstract}

\begin{pbo}

Todo subconjunto no vacío de $\mathbb{Z}_{+}$ contiene un elemento mínimo. O sea,
$\exists(m)$ tal que $\forall(x) x\in A\wedge x\neq m$ se cumple que $m<x$
\end{pbo}

\begin{pim}
 Dada una proposición $P$, si se cumple $P(n_0)$ con $n_0\in \mathbb{Z}_{+}$ y, además, 
 $\forall(n)$ $n\geq n_0\wedge P(n) \Rightarrow P(n+1)$ entonces $\forall(n)$ $n\geq n_0 \wedge P(n)$
\end{pim}

\begin{teo}
 El Principio del Buen Ordenamiento es equivalente al Principio de Inducción Matemática
\end{teo}

\textbf{Demostración}

Sea $C$ el conjunto de los números naturales que no cumplen $P$ y asumamos que $P\neq \varnothing$. Entonces, por el \textbf{Principio del Buen Ordenamiento} existe $m\in C$ tal que $m$ es el mínimo elemento de $C$.

Ahora, asumamos a 1 como $n_0$, luego como $P(1)$ se cumple entonces $m>1$ por lo que $m-1\geq 1$.

Como $m-1<m$ entonces $m-1\notin C$ por lo que $P(m-1)$ se cumple. Por tanto, como para todo $n>1$ se tiene que $P(n)\Rightarrow P(n+1)$ entonces dado que $P(m-1)$ se cumple se tendría que $P(m)$ también se cumple ¡lo que es una contradicción!

\textbf{Ejemplo}
Demuestre, utilizando el \textbf{Principio del Buen Ordenamiento}, que para toda $n$, $n\in\mathbb{Z}$, $n\geq 1$ se cumple que $\sum^n_{k=1}(2k-1)=n^2$

Sea $C$ el conjunto de los números naturales que no cumplen $P$ y asumamos que $P\neq \varnothing$. Entonces, por el \textbf{Principio del Buen Ordenamiento} existe $m\in C$ tal que $m$ es el mínimo elemento de $C$.

$P(1)$ se cumple pues $\sum^1_{k=1}(2k-1)=2-1=1=1^2$, por tanto $m>1$ por lo que $m-1\geq1$. Ahora, como $m-1\geq m$ entonces $m-1\notin C$ por lo que $P(m-1)$ se cumple. Entonces $\sum^{m-1}_{k=1}(2k-1)=(m-1)^2$.

Ahora se tiene que\\ 
$\sum^m_{k=1}(2k-1)=\sum^{m-1}_{k=1}(2k-1)+(2m-1)$\\
$\sum^m_{k=1}(2k-1)=(m-1)^2+(2m-1)$\\
$\sum^m_{k=1}(2k-1)=(m^2-2m+1)+(2m-1)$\\
$\sum^m_{k=1}(2k-1)=m^2$\\
O sea, $P(m)$ se cumple, lo que es una ¡contradicción!

\begin{dfn}
 Sean $a,b$, $a\in\mathbb{Z}$, $b\in\mathbb{Z}$, $a\neq 0$, se dice que $a$ divide a $b$ o que $a$ es múltiplo de $b$, denotado $a|b$, si $\exists(q)$ $q\in\mathbb{Z}$ tal que $b=a*q$
\end{dfn}

\begin{lem}
 Todo número $a$, $a\in\mathbb{Z}$, es divisor de 0
\end{lem}

\begin{teo}
 Sean $a,b$, $a\in\mathbb{Z}$, $b\in\mathbb{Z}$, si $b|a$ y $a\neq 0$ entonces $|a| \geq |b|$
\end{teo}



\begin{teo}
 La relación \textbf{ser divisor de} es transitiva. O sea, si $a|b$ y $b|c$ entonces $a|c$
\end{teo}

\textbf{Demostración}

\begin{teo}
 \textbf{Algoritmo de la División}, sean $a,b$, $a\in\mathbb{Z}$, $b\in\mathbb{Z}$, $b > 0$, entonces existen $q,r$, $q\in\mathbb{Z}$, $r\in\mathbb{Z}$, únicos tales que $a = b*q+r$ donde $0\leq r < b$
\end{teo}

\textbf{Demostración}

\begin{dfn}
 Sea $a\in\mathbb{Z}$ tal que $n>1$, se dice que $n$ es un \textbf{número primo} si y solo sus únicos divisores positivos son $1$ y $n$, de lo contrario se dice que $n$ es un \textbf{número compuesto}
\end{dfn}

\begin{cor}
 $n$, $n\in\mathbb{Z}$, $n>1$, es un \textbf{número compuesto} si y solo si $n=a*b$ con $a\in\mathbb{Z}$, $b\in\mathbb{Z}$, $1<a\leq b < n$
\end{cor}

\begin{lem}
 Todo número entero mayor que 1 tiene un divisor primo
\end{lem}

\textbf{Demostración}

\begin{teo}
 Hay una infinita cantidad de números primos
\end{teo}

\textbf{Demostración}














\end{document}          
