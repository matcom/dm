\documentclass[a4paper,1pt]{report}
\usepackage[utf8]{inputenc}
\usepackage{amsfonts}
\usepackage{amsthm}
\usepackage{amssymb}

\newtheorem*{pbo}{Principio del Buen Ordenamiento}

\newtheorem*{pim}{Principio de Inducción Matemática}

\newtheorem*{teo}{Teorema}

\newtheorem*{cor}{Corolario}

\newtheorem*{dem}{Demostración}

\newtheorem*{dfn}{Definición}

\newtheorem*{lem}{Lema}

\newtheorem*{prp}{Propiedades}

% Title Page
\title{Conferencia 1 - Principios de la Teoría de Números}
\author{}



\begin{document}
\maketitle

%\begin{abstract}
%\end{abstract}

\begin{pbo}

Todo subconjunto no vacío de $\mathbb{Z}_{+}$ contiene un elemento mínimo. O sea,
$\exists(m)$ tal que $\forall(x) x\in A\wedge x\neq m$ se cumple que $m<x$
\end{pbo}

\begin{pim}
 Dada una proposición $P$, si se cumple $P(n_0)$ con $n_0\in \mathbb{Z}_{+}$ y, además, 
 $\forall(n)$ $n\geq n_0\wedge P(n) \Rightarrow P(n+1)$ entonces $\forall(n)$ $n\geq n_0 \wedge P(n)$
\end{pim}

\begin{teo}
 El Principio del Buen Ordenamiento es equivalente al Principio de Inducción Matemática
\end{teo}

\textbf{Demostración}

Sea $C$ el conjunto de los números naturales que no cumplen $P$ y asumamos que $P\neq \varnothing$. Entonces, por el \textbf{Principio del Buen Ordenamiento} existe $m\in C$ tal que $m$ es el mínimo elemento de $C$.

Ahora, asumamos a 1 como $n_0$, luego como $P(1)$ se cumple entonces $m>1$ por lo que $m-1\geq 1$.

Como $m-1<m$ entonces $m-1\notin C$ por lo que $P(m-1)$ se cumple. Por tanto, como para todo $n>1$ se tiene que $P(n)\Rightarrow P(n+1)$ entonces dado que $P(m-1)$ se cumple se tendría que $P(m)$ también se cumple ¡lo que es una contradicción!

\textbf{Ejemplo}
Demuestre, utilizando el \textbf{Principio del Buen Ordenamiento}, que para toda $n$, $n\in\mathbb{Z}$, $n\geq 1$ se cumple que $\sum^n_{k=1}(2k-1)=n^2$

Sea $C$ el conjunto de los números naturales que no cumplen $P$ y asumamos que $P\neq \varnothing$. Entonces, por el \textbf{Principio del Buen Ordenamiento} existe $m\in C$ tal que $m$ es el mínimo elemento de $C$.

$P(1)$ se cumple pues $\sum^1_{k=1}(2k-1)=2-1=1=1^2$, por tanto $m>1$ por lo que $m-1\geq1$. Ahora, como $m-1\geq m$ entonces $m-1\notin C$ por lo que $P(m-1)$ se cumple. Entonces $\sum^{m-1}_{k=1}(2k-1)=(m-1)^2$.

Ahora se tiene que\\ 
$\sum^m_{k=1}(2k-1)=\sum^{m-1}_{k=1}(2k-1)+(2m-1)$\\
$\sum^m_{k=1}(2k-1)=(m-1)^2+(2m-1)$\\
$\sum^m_{k=1}(2k-1)=(m^2-2m+1)+(2m-1)$\\
$\sum^m_{k=1}(2k-1)=m^2$\\
O sea, $P(m)$ se cumple, lo que es una ¡contradicción!

\begin{dfn}
 Sean $a,b$, $a\in\mathbb{Z}$, $b\in\mathbb{Z}$, $a\neq 0$, se dice que $a$ divide a $b$ o que $a$ es múltiplo de $b$, denotado $a|b$, si $\exists(q)$ $q\in\mathbb{Z}$ tal que $b=a*q$
\end{dfn}

\begin{lem}
 Todo número $a$, $a\in\mathbb{Z}$, es divisor de 0
\end{lem}

\begin{teo}
 Sean $a,b$, $a\in\mathbb{Z}$, $b\in\mathbb{Z}$, si $b|a$ y $a\neq 0$ entonces $|a| \geq |b|$
\end{teo}



\begin{teo}
 La relación \textbf{ser divisor de} es transitiva. O sea, si $a|b$ y $b|c$ entonces $a|c$
\end{teo}

\textbf{Demostración}

Se debe demostrar que si $a|b$ y $b|c$ entonces $a|c$

Como $a|b$ entonces existe $q_1, q_1\in\mathbb{Z}$ tal que $b=aq_1$
Del mismo modo, como $b|c$ existe $q_2, q_2\in\mathbb{Z}$ tal que $c=bq_2$

Ahora, como $c=bq_2=aq_1q_2$ entonces tomando $q=q_1q_2\in\mathbb{Z}$ se tiene entonces que $c=a*q$ y, por tanto, $a|c$ 

\begin{teo}
 \textbf{Algoritmo de la División}, sean $a,b$, $a\in\mathbb{Z}$, $b\in\mathbb{Z}$, $b > 0$, entonces existen $q,r$, $q\in\mathbb{Z}$, $r\in\mathbb{Z}$, únicos tales que $a = b*q+r$ donde $0\leq r < b$
\end{teo}

\textbf{Demostración}

Por una parte, si $b|a$ entonces existe $q\in\mathbb{Z}$ tal que $a=bq$, luego, para este caso con $r=0$ se cumple que $a=bq+r$

En el otro caso, si $b\nmid a$ entonces se puede construir el conjunto 

$S=\{a-sb|a-sb>0, s\in\mathbb{Z}\}$, noten que este es el conjunto los posibles $r$.

Ahora se debe demostrar que $S$ no es vacío.

Veamos para $a>0$, entonces para este caso se toma $s=0$ y es evidente aquí que el conjunto posee al menos al elemeno $a$.

Para $a<0$ tomamos a $s=a-1$ y por tanto

$a-sb=a-(a-1)b$

$a-sb=a-ab-b$

$a-sb=a(1-b)+b$

Como $a<0$ y $1-b<0$~(pues $b>0$) entonces $a(1-b)$ es mayor que 0 y, por tanto, $a(1-b)+b$ también lo es.

Luego, sea $r$ el elemento mínimo de $S$ y sea $s=q$ se tiene que $a-bq=r$ entonces $a=bq+r$

Ahora se debe demostrar que $0\leq r < b$.

Se sabe que $r=a-sb>0$

Supongamos que $r>b$ por tanto

$r-b>0$ y como $r=a-bq$ entonces $r-b=a-qb-b>0$ y estos es lo mismo que $r-b=a-q(b+1)>0$, luego $r-b\in\mathbb{Z}$ y como $r>r-b$ esto es una contradicción pues $r$ era el elemento mínimo de $S$.

Ahora se debe demostrar que $q$ y $r$ son únicos.

Supongamos que existen $q_1,r_1$ tal que $q_1\neq q$ o $r_1\neq r$, y $a=bq_1+r_1=bq+r$

Entonces $b(q-q1)=r_1-r$

y como se cumple que $0\leq r < b$ y $0\leq r_1 < b$

se tiene que $-b<r-r_1<b$ y, por tanto,

$-b<b(q-q_1)<b$

$-1<q-q_1<1$

Como $q-q_1\in\mathbb{Z}$ ello implica que $q-q_1=0$ y $q=q_1$ por tanto $r = r_1$ y esto es una contradicción, luego $q$ y $r$ son únicos.



\begin{dfn}
 Sea $a\in\mathbb{Z}$ tal que $n>1$, se dice que $n$ es un \textbf{número primo} si y solo sus únicos divisores positivos son $1$ y $n$, de lo contrario se dice que $n$ es un \textbf{número compuesto}
\end{dfn}

\begin{cor}
 $n$, $n\in\mathbb{Z}$, $n>1$, es un \textbf{número compuesto} si y solo si $n=a*b$ con $a\in\mathbb{Z}$, $b\in\mathbb{Z}$, $1<a\leq b < n$
\end{cor}

\begin{lem}
 Todo número entero mayor que 1 tiene un divisor primo
\end{lem}

\textbf{Demostración}

\textit{Demostración 1}

Para $n>1$

Si $n$ es primo ya está demostrado.

Si $n$ no es primo es compuesto, entonces $n=ab$, $1<a,b<n$

Si $a$ es primo o $b$ es primo ya queda demostrado.

Sino $a$ es compuesto y es de la forma $a=a_1b_1$, $1<a_1,b_1<n$

\dots

\dots

Como no existe descenso infinito para números positivos, este proceso debe terminar encontrando un número $a_i$ primo que por transitividad divide a $n$.

\textit{Demostración 2}

Para $n=3$ se cumple.

Luego hasta $n-1$, entonces si $n$ es primo ya, sino $n=ab$, $1<a,b<n$.

Si $a$ es primo se cumple sino $a$ es compuesto y como $a<n$ entonces tiene divisores primos los que, por transitividad, también lo son de $n$.

\textit{Demostración 3}

Si $n$ es primo, ya está demostrado.
Sino, se tiene $D=\{d|\, d|n, 1<d<n\}$ y sea $m$ el mínimo elemento de $D$. 

Supongamos que $m$ es compuesto, luego existe $p$ primo tal que $p|m$, entonces por transitividad $p|n$ y $p<n$, y esto es un contradicción. Luego $m$ es primo.

\begin{teo}
 Hay una infinita cantidad de números primos
\end{teo}

\textbf{Demostración}

Si tenemos el conjunto de $k$ números primos distintos,$A=\{p_1,_2,\dots ,p_k\}$
entonces tomemos $m=p_1p_2\dots p_k+1$

Ahora, si $p_i|m (1\leq i\leq k)$ como $p_i|p_1p_2\dots p_k$  entonces $p_i|1$ lo que es una contradicción.

Luego, existe $q$ primo tal que $q|m$ y $q\not\in A$











\end{document}          
