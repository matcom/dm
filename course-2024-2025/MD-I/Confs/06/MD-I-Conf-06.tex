\documentclass[a4paper,12pt]{report}
\usepackage[utf8]{inputenc}
\usepackage{amsfonts}
\usepackage{amsthm}
\usepackage{amssymb}
\usepackage{amsmath}

\newtheorem*{pbo}{Principio del Buen Ordenamiento}

\newtheorem*{pim}{Principio de Inducción Matemática}

\newtheorem*{teo}{Teorema}

\newtheorem*{cor}{Corolario}

\newtheorem*{dem}{Demostración}

\newtheorem*{dfn}{Definición}

\newtheorem*{lem}{Lema}

\newtheorem*{prp}{Propiedades}

\newtheorem*{pro}{Proposición}

% Title Page
\title{Conferencia 6 - Función de Euler}
\author{}



\begin{document}
\maketitle

%\begin{abstract}
%\end{abstract}

 
 \begin{dfn}
  Sea $n\in \mathbb{Z}_+$, la \textbf{Función de Euler} que se denota $\varphi(n)$ representa el número de enteros positivos menores o iguales que $n$ primos relativos con $n$. O sea $\varphi(n)=|\{d\, |1\leq d \leq n,(d,n)=1\}|$. 
  
  Para 1 se define $\varphi(1)=1$
 \end{dfn}
 
 \begin{dfn}
  Se llama \textbf{Sistema Residual Reducido módulo n}~(SRR(n)) a un conjunto de $\varphi(n)$ enteros positivos incongruentes módulo $n$ que son primos relativos con $n$
 \end{dfn}
 
 O sea, dado un natural positivo $n$, se dice que un conjunto SRR es un \textbf{Sistema Residual Reducido módulo n} si cumple lo siguiente:
 
 \begin{enumerate}
  \item SRR posee $\varphi(n)$ elementos
  \item para cada $a\in SRR$ se cumple $(a,n)=1$
  \item los elementos de SRR son incongruentes módulo de $n$ entre si. O lo que es lo mismo,  si $a,b\in SRR$ y $a\neq b$ entonces $a\not\equiv b \, (n)$
 \end{enumerate}

 
\begin{teo}
 Sean $n\in\mathbb{Z}_+$, $k\in\mathbb{Z}$, $(k,n)=1$ y $\{a_1,a_2,\dots,a_{\varphi(n)}\}$ un sistema residual reducido módulo $n$, entonces $\{ka_1,ka_2,\dots,ka_{\varphi(n)}\}$ es también un sistema residual reducido módulo $n$.
\end{teo}

\textbf{Demostración}

%Se cumple que $\forall (i)\, 1\leq i \leq \varphi(n) \, (ka_i,n)=1$

Supongamos que $\{ka_1,ka_2,\dots,ka_{\varphi(n)}\}$ no es un SRR(n)

entonces existen $i$,$j$ tales que $ka_i\equiv ka_j \, (n)$

como $(k,n)=1$ entonces $a_i\equiv a_j \, (n)$

luego $\{a_1,a_2,\dots,a_{\varphi(n)}\}$ tampoco es un SRR(n), 

por tanto, por contrarecíproco, si $\{a_1,a_2,\dots,a_{\varphi(n)}\}$ es un SRR(n) 

entonces $\{ka_1,ka_2,\dots,ka_{\varphi(n)}\}$ también lo es

\begin{teo}
 $\varphi(p)=p-1$ si y solo si $p$ es primo
\end{teo}

\textbf{Demostración}

Si $p$ es primo entonces significa que todo entero positivo menor que él es primo relativo con él, por tanto $\varphi(p)=p-1$

Ahora, si $\varphi(p)=p-1$ y $p$ es compuesto entonces $p=qr$ con $1<q\leq r <p$ por lo qué habría al menos dos números enteros positivo, además del propio $p$, que no serían primos relativos con $p$ por lo que $\varphi(p)\leq p-3$, lo que es una contradicción y, por tanto, $p$ es primo

\begin{teo}
 Si $p$ es primo y $k>0$ entonces $\varphi(p^k)=p^k - p^{k-1}$ 
\end{teo}

\textbf{Demostración}

Para cualquier número $n$ se tiene que $(n,p^k)=1$ si y solo si $p\nmid n$. Ahora, entre 1 y $p^k$ hay $p^{k-1}$ enteros que son divisibles por $p$ y que, por tanto, no son primos relativos con $p^k$. Estos serían $p,2p,3p,\dots,p^{k-1}p$. Luego, el conjunto $\{1,2,3,\dots,p^k\}$ contendría $p^k-p^{k-1}$ enteros que son primos relativos con $p^k$ y, por tanto, por definición, $\varphi(p^k)=p^k - p^{k-1}$ 

\begin{teo}
 \textbf{Teorema de Euler} Sean $a,n\in\mathbb{Z}$, $n>0$ y $(a,n)=1$ entonces $a^{\varphi(n)}\equiv 1\, (n)$
\end{teo}

\textbf{Demostración}

Sea $\{n_1,n_2,\dots,n_{\varphi(n)}\}$ un SRR(n), entonces $\{an_1,an_2,\dots,an_{\varphi(n)}\}$ 

con $a\in\mathbb{Z}$ tal que $(a,n)=1$ es también un SRR(n), luego se cumple

$(an_1)(an_2)\dots(an_{\varphi(n)})\equiv n_1 n_2 \dots n_{\varphi(n)} \, (n)$

$a^{\varphi(n)}n_1 n_2\dots n_{\varphi(n)}\equiv n_1 n_2 \dots n_{\varphi(n)} \, (n)$

pero como $\forall(i) 1\leq i\leq \varphi(n)$ $(n_i,n)=1$

entonces $a^{\varphi(n)}\equiv 1\, (n)$


\begin{dfn}
 Una función se denomina \textbf{aritmética} si está definida en los enteros positivos.
\end{dfn}

\begin{dfn}
 Una función aritmética  $f$ se denomica \textbf{multiplicativa} si para cualquier $m$ y $n$ tales que $(m,n)=1$ se cumple que $f(mn)=f(m)f(n)$
\end{dfn}

\begin{teo}
 Si $f$ es una función multiplicativa y $n$ se descompone en primos de la siguiente forma $n=p_1^{e_1}p_2^{e_2}\dots p_k^{e_k}$ entonces $f(n)=f(p_1^{e_1})f(p_2^{e_2})\dots f(p_k^{e_k})$ 
\end{teo}


\textbf{Demostración}

Si $n$ tiene exactamente dos divisores primos distintos, entonces 

$k=2$ implica que $n=p_1^{e_1}p_2^{e_2}$ por lo que $f(n)=f(p_1^{e_1}p_2^{e_2})$

ahora como $(p_1,p_2)=1$ entonces $(p_1^{e_1},p_2^{e_2})=1$ 

y como f es multiplicativa entonces $f(n)=f(p_1^{e_1}p_2^{e_2})=f(p_1^{e_1})f(p_2^{e_2})$


Ahora, supongamos que se cumple para $k=m$, 

probemos entonces que se cumple para $k=m+1$

se tiene que $n=p_1^{e_1}p_2^{e_2}\dots p_{m}^{e_m}p_{m+1}^{e_m+1}$ y como todos los $p_i$, $1\leq i \leq m+1$, 

son primos relativos 2 a 2, se cumple que $(p_1^{e_1}p_2^{e_2}\dots p_{m}^{e_m},p_{m+1}^{e_m+1})=1$

y como $f$ es multiplicativa entonces $f(n)=f(p_1^{e_1}p_2^{e_2}\dots p_{m}^{e_m})f(p_{m+1}^{e_m+1})$

pero como se cumple hasta $m$ entonces $f(n)=f(p_1^{e_1})f(p_2^{e_2})\dots f(p_m^{e_m})f(p_{m+1}^{e_{m+1}})$


\begin{teo}
 $\varphi(n)$ es multiplicativa
\end{teo}
\newpage

\textbf{Demostración}

Si tenemos la matriz

\begin{equation}
\begin{pmatrix}
1 & m+1 & 2m+1& \dots & (n-1)m+1\\
2 & m+2 & 2m+2& \dots & (n-1)m+2\\
\vdots & \vdots & \vdots& \dots & \vdots\\
m & 2m & 3m& \dots & nm
\end{pmatrix}
\end{equation}

Note que los que los elementos de una fila son de la forma $cm + r$,  $0 \leq c \leq n-1$ y una $r$ fija, $1\leq r \leq m$,

Ahora, si $(r,m)>1$ entonces no hay ningún número es esa fila tal que sea primo relativo com $m$. Entonces eliminamos todas las filas de la matriz tales que $(r,m)>1$.

Luego, si $(r,m)=1$ todos lo números de esa fila son primos relativos con $m$. Por tanto hay $\varphi(m)$ filas que me interesan.

Se puede notar que cada una de estas filas es un $SRC(n)$ por lo que hay $\varphi(n)$ números por cada fila que son primos relativos con $n$.

Entonces, como $\varphi(nm)$ es el número de enteros de la matriz que son primos relativos a $nm$, y como para que sea multiplicativa, por definición, se tiene que $(n,m)=1$, entonces como hay $\varphi(m)$ columnas con enteros primos relativos a $m$ y en cada una de ellas $\varphi(n)$ primos relativos con $n$, se tiene entonces que $\varphi(nm)=\varphi(m)\varphi(n)$

\begin{teo}
 Sea $n\in\mathbb{Z}$ con $n>1$ tal que $n=p_1^{e_1}p_2^{e_2}\dots p_k^{e_k}$ entonces 
 
 $\varphi(n)=n(1-\frac{1}{p_1})(1-\frac{2}{p_2})\dots(1-\frac{1}{p_k})$
\end{teo}

\textbf{Demostración}

$\varphi(n)=\varphi(p_1^{e_1})\varphi(p_3^{e_3})\dots\varphi(p_k^{e_k})$

$\varphi(n)=(p_1^{e_1}-p_1^{e_1-1})(p_2^{e_2}-p_2^{e_2-1})\dots(p_k^{e_k}-p_k^{e_k-1})$

$\varphi(n)=p_1^{e_1}p_2^{e_2}\dots p_k^{e_k}(1-\frac{1}{p_1})(1-\frac{2}{p_2})\dots(1-\frac{1}{p_k})$

$\varphi(n)=n(1-\frac{1}{p_1})(1-\frac{2}{p_2})\dots(1-\frac{1}{p_k})$

\begin{teo}
 Sea $n\in\mathbb{Z}$ tal que $n>2$ entonces $\varphi(n)$ es par
\end{teo}


\textbf{Demostración}

Si $n=2^k$ con $k>1$ se tiene que $\varphi(2^k)=2^k-2^{k-1}$ 

y la resta de dos números pares es par.

Si $n$ no es una potencia de 2 entonces lo divide un primo impar $p$ 

tal que $n=p^dm$ con $(p^d,m)=1$ por lo que $\varphi(n)=\varphi(p^d)\varphi(m)$ 

y como $\varphi(p^d)=p^d-p^{d-1}=p^{d-1}(p-1)$, por tanto, $p-1$ es par, 

luego  $\varphi(n)$ es par.


\begin{dfn}
Sean $a,n\in\mathbb{Z}_+$, $(a,n)=1$, el menor entero positivo tal que $a^k \equiv 1 \, (n)$ se denomina
orden de $a$ módulo de $n$ y se denota $ord_na$
\end{dfn}

Note que $ord_na\leq\varphi(n)$

\end{document}          
