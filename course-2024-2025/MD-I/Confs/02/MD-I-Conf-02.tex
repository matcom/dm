\documentclass[a4paper,1pt]{report}
\usepackage[utf8]{inputenc}
\usepackage{amsfonts}
\usepackage{amsthm}
\usepackage{amssymb}

\newtheorem*{pbo}{Principio del Buen Ordenamiento}

\newtheorem*{pim}{Principio de Inducción Matemática}

\newtheorem*{teo}{Teorema}

\newtheorem*{cor}{Corolario}

\newtheorem*{dem}{Demostración}

\newtheorem*{dfn}{Definición}

\newtheorem*{lem}{Lema}

\newtheorem*{prp}{Propiedades}

% Title Page
\title{Conferencia 2 - Principios de la Teoría de Números}
\author{}



\begin{document}
\maketitle

%\begin{abstract}
%\end{abstract}




\begin{dfn}
 Sean $a,b$, $a\in\mathbb{Z}$, $b\in\mathbb{Z}$, $a\neq 0$ o $b\neq 0$, se denota\\ $mcd(a,b)=max\{d|   d\in\mathbb{Z}\wedge d|a\wedge d|b\}$ como el máximo común divisor de a y b.
\end{dfn}

El $mcd(a,b)$ también suele denotarse $(a,b)$ .



\begin{prp}
 $mcd(a,b)=mcd(-a,b)=mcd(a,-b)=mcd(-a,-b)$
\end{prp}

\begin{teo}
 Sean $a,b$, $a\in\mathbb{Z}$, $b\in\mathbb{Z}$, si $a|b$ entonces $mcd(a,b)=|a|$
\end{teo}

El $mcd(a,0)=|a|$ $(a\neq0)$.


\begin{dfn}
 Sean $a,b$, $a\in\mathbb{Z}$, $b\in\mathbb{Z}$, si el $mcd(a,b)=1$ entonces a y b son \textbf{primos relativos}
\end{dfn}

\begin{dfn}
 Un entero $c$ es combinación lineal de los enteros $a_1,a_2,\dots,a_n$ si existen enteros $b_1,b_2,\dots,b_n$ tales que $c=a_1*b_1+a_2*b_2+\dots+a_b*b_n$.
\end{dfn}

\begin{teo}
 El máximo común divisor de  $a_1,a_2,\dots,a_n$, números enteros, no todos iguales a 0, $mcd(a_1,a_2,\dots,a_n)$ es el menor entero positivo que puede ser expresado como combinación lineal de $a_1,a_2,\dots,a_n$.
\end{teo}

\textbf{Demostración}

Partamos de $a_1x_1+a_2x_2+\dots+a_nx_n$.

Tomando $x_i=a_i$ se tiene que $\sum^n_{i=1}a_ix_i=\sum^n_{i=1}a_i^2\geq0$

Como existe $a_k\neq 0 \, (1\leq k \leq n)$ entonces $\sum^n_{i=1}a_i^2>0$

Por tanto existe al menos una combinación lineal positiva.

Sea $S=\{d|d>0,\, d = a_1x_1+a_2x_2+\dots+a_nx_n,\, \forall (i)\, 1\leq i \leq n\, x_i\in\mathbb{Z}\}$

$S\neq \emptyset$

Por el \textbf{Principio del Buen Ordenamiento~(PBO)} tomemos 

$s=a_1s_1+a_2s_2+\dots+a_ns_n$ como el menor elemento de $S$.

Probemos que $s|a_1$

Por el \textbf{Algoritmo de la División}

$a_1=sq+r$, $0\leq r < s$ 

Supongamos que $r>0$

$r=a_1-sq$

$r=a_1 - (a_1s_1+a_2s_2+\dots+a_ns_n)q$

$r=a_1(1-s_1q)+a_2(-s_2q)+\dots+a_n(-s_nq)$

Por tanto $r$ es una combinación lineal positiva de los $a_i$ tal que $r<s$ pero $s$ es la menor de las combinaciones lineales positivas. Y esto es una contradicción! 

Luego $r=0$ y, por tanto, $s|a_1$

Análogamente, se puede demostrar que $s|a_i,\, 1\leq i \leq n$

Entonces $s$ es divisor común de $a_i,\, 1\leq i \leq n$

Ahora, sea $d$ el mayor de los divisores comúnes de $a_i$, entonces $s\leq d$

Por otra parte, $d$ divide a cualquier combinación lineal de $a_i$, entonces $d|s$ y, por tanto, $d\leq s$

Entonces, como $d\leq s \leq d$ se tiene que $s=d$

\begin{teo}
 Sean $a,b$, $a\in\mathbb{Z}$, $b\in\mathbb{Z}$, el conjunto de los divisores comunes de $a$ y $b$ coincide con el conjunto de los divisores del $mcd(a,b)$
\end{teo}

\begin{cor}
 Si $a_1,a_2,\dots,a_n$ son números enteros no todos iguales a 0 entonces $mcd(a_1,a_2,\dots,a_n)=mcd(a_1,mcd(a_2,a_3,\dots,a_n))$
\end{cor}

\begin{cor}
 Sean $a,b$, $a\in\mathbb{Z}$, $b\in\mathbb{Z}$, no simultáneamente nulos, entonces $\frac{a}{mcd(a.b)}$ y  $\frac{b}{mcd(a.b)}$ son \textbf{primos relativos}. O sea, $mcd(\frac{a}{(a.b)},\frac{b}{(a.b)})=1$
\end{cor}

\begin{teo}
 Sea $a$, $a\in\mathbb{Z}$, $a\neq 0$, $b_i\in\mathbb{Z}$, $1\leq i \leq n$, si $a|b_1*b_2*\dots *b_n$ y para  todo $j$, $1\leq j \leq n-1$, se cumple que $mcd(a,b_j)=1$ entonces $a|b_n$
\end{teo}

\begin{cor}
 Sean $a,b,q,r$ tales que $a\in\mathbb{Z}$, $b\in\mathbb{Z}$, $q\in\mathbb{Z}$, $r\in\mathbb{Z}$, $b\neq 0$, y $a=q*b+r$ entonces $mcd(a,b)=mcd(b,r)$
\end{cor}

\textbf{Demostración}



Si $x|a$ y $x|b$ entonces se tiene que $a=xy_1$ y $b=xy_2$, luego 

$a=qb+r$

$xy_1 = qxy_2 + r$

$r = xy_1 - qxy_2$

$r = x(y_1-qy_2)$

Por lo que $x|r$

De igual modo, si $x|b$ y $x|r$ entonces se tiene que $b=xy_2$ y $r=xy_3$, luego

$a=qb+r$

$a=qxy_2+xy_3$

$a=x(qy_2+y_3)$

Por lo que $x|a$

Como los divisores comunes de $a$ y $b$ coinciden con los de $b$ y $r$, entonces tendrán el mismo máximo común divisor.

\begin{dfn}
 Sean $a,b,c$, $a\in\mathbb{Z}$, $b\in\mathbb{Z}$, $c\in\mathbb{Z}$, $a\neq 0$, $b\neq 0$ se dice que $ax+by=c$ es una ecuación lineal diofantina si esta es resuelta con $x\in\mathbb{Z}$ y $y\in\mathbb{Z}$
\end{dfn}

\begin{teo}
 La ecuación lineal diofantina $ax+by=c$ tiene solución si y solo si $mcd(a,b)|c$
\end{teo}

\textbf{Demostración}

Se debe demostrar en ambos sentidos.
\\

Demostremos que si $ax+by=c$ tiene solución entonces $mcd(a,b)|c$.

Como $ax+by=c$ tiene solución tomemos $d=mcd(a,b)$, luego se sabe que

$d|ax+by$ y, por tanto, $d|c$.
\\

Demostremos ahora que si $mcd(a,b)|c$ entonces $ax+by=c$ tiene solución.

Si $mcd(a,b)|c$ entonces existe $k,\, k\in\mathbb{Z}$ tal que $c=k(a,b)$.

Ahora, sabemos que existe $x_0,y_o\in\mathbb{Z}$ tal que

$ax_0+by_0=(a,b)$

por lo que

$akx_0+bky_0=k(a,b)$

Entonces, si se toma $x=kx_0$ y $y=ky_0$ se cumple que existe $x,y\in\mathbb{Z}$ tal que 

$ax+by=c$


\newpage
\begin{teo}
 \textbf{Algoritmo de Euclides}. Sean $a,b$, $a\in\mathbb{Z}$, $b\in\mathbb{Z}$, $a>b$, si se realizan los siguientes cálculos:\\
 $a=q_1*b+r_1$ $0\leq r_1<b$\\
 $b=q_2*r_1+r_2$  $0\leq r_2<r_1$\\
 $r_1=q_3*r_2+r_3$ $0\leq r_3<r_2$\\
 $r_2=q_4*r_3+r_4$ $0\leq r_4<r_3$\\
 \dots\\
 \dots\\
 \dots\\
 $r_{k-2}=q_k*r_{k-1}+r_k$ $0\leq r_k<r_{k-1}$\\
 $r_{k-1}=q_{k+1}*r_{k}$ $0=r_{k+1}$\\
 donde $r_k$ es el último resto diferente de 0, entonces $r_k=mcd(a,b)$
\end{teo}

\textbf{Ejemplo}
Para calcular el máximo común divisor de 3088 y 456:\\
$3088=6*456+352$\\
$456=1*352+104$\\
$352=3*104+40$\\
$104=2*40+24$\\
$40=1*24+16$\\
$24=1*16+8$\\
$16=2*8 + 0$\\
Entonces 8 es el último resto distinto de 0. Por tanto $mcd(3088,456)=8$

A partir del \textbf{Algoritmo de Euclides} también se puede calcular la combinación lineal de la siguiente forma:\\
$A_1=1$ $B_1=-q_k$\\
$A_2=B_1$ $B_2=A_1-q_{k-1}*B_1$\\
\dots\\
$A_{i+1}=B_i$ $B_{i+1}=A_i-q_{k-i}*B_i$\\
\dots\\
$A_{k-1}=B_{k-2}$ $B_{k-1}=A_{k-2}-q_2*B_{k-2}$\\
$A_{k}=B_{k-1}$ $B_{k}=A_{k-1}-q_1*B_{k-1}$\\
Luego $r_k=a*A_k+b*B_k$ y, por lo tanto, $r_k=a*A_k+b*B_k=mcd(a,b)$

\textbf{Ejemplo}
Para calcular la combinación lineal de 3088 y 456 con la que se obtiene su $mcd$ se tiene:\\
$3088=6*456+352$ $A_1=1$ $B_1=-1$\\
$456=1*352+104$ $A_2=-1$ $B_2=1-1*(-1)=2$\\
$352=3*104+40$ $A_3=2$ $B_3=-1-2*2=-5$\\
$104=2*40+24$ $A_4=-5$ $B_4=2-3*(-5)=17$\\
$40=1*24+16$ $A_5=17$ $B_5=-5-1*17=-22$\\
$24=1*16+8$ $A_6=-22$ $B_6=17-6*(-22)=149$\\
$16=2*8 + 0$\\
Por tanto $8=mcd(3088,456)=3088*(-22)+456*149$

\begin{teo}
 Si $x_0,y_0$ son una solución de la ecuación diofantina $ax+by=c$ entonces 
 $x=x_0+k\frac{b}{(a,b)}$ y $y=y_0-k\frac{a}{(a,b)}$ con $k\in\mathbb{Z}$ es la solución general de la ecuación diofantina.
\end{teo}

\newpage
\textbf{Demostración}

Se debe demostrar, primero que es solución y luego que toda solución es de esa forma.
\\

La demostración de lo primero es trivial, basta sustituir en la ecuación original.
\\

Para demostrar lo segundo, asumamos que $x_1$,$y_1$ son otra solución de la ecuación, luego

$ax_0+by_0=ax_1+by_1$

$a(x_0-x_1)=b(y_1-y_0)$

$\frac{a}{(a,b)}(x_0-x_1)=\frac{b}{(a,b)}(y_1-y_0)$

Esto implica que

$\frac{b}{(a,b)}|\frac{a}{(a,b)}(x_0-x_1)$

pero como se sabe que $(\frac{a}{(a,b)},\frac{b}{(a,b)})=1$ entoces $\frac{b}{(a,b)}|x_0-x_1$

por tanto existe $k\in\mathbb{Z}$ tal que $x_1=x_0 + k\frac{b}{(a,b)}$ luego

$\frac{a}{(a,b)}(x_0-x_1)=\frac{b}{(a,b)}(y_1-y_0)$


$\frac{a}{(a,b)}(x_0- x_0 - k\frac{b}{(a,b)} )=\frac{b}{(a,b)}(y_1-y_0)$

$\frac{a}{(a,b)}(- k\frac{b}{(a,b)} )=\frac{b}{(a,b)}(y_1-y_0)$

$-k\frac{a}{(a,b)}=y_1-y_0$

$y_1=y_0-k\frac{a}{(a,b)}$

Entonces $x=x_0+k\frac{b}{(a,b)}$ y $y=y_0-k\frac{a}{(a,b)}$ son solución general de la ecuación.

\begin{dfn}
 Sean $a,b,c$, $a\in\mathbb{Z}$, $b\in\mathbb{Z}$, $c\in\mathbb{Z}$, los tres distintos de 0, se dice que $c$ es múltiplo común de $a$ y $b$ si $c$ es múltiplo de $a$ y $c$ es múltiplo de $b$. Se dice que $c$ es el mínimo común múltiplo de $a$ y $b$, si es el menor entero positivo múltiplo común de de $a$ y $b$, lo que se denota $mcm(a,b)$.
\end{dfn}

El $mcm(a,b)$ también suele denotarse $[a,b]$.



\begin{teo}
 Sean $a,b$, $a\in\mathbb{Z}_+$, $b\in\mathbb{Z}_+$, todo múltiplo común de $a$ y $b$ se expresa como $k\frac{a*b}{(a,b)}$ donde $k\in\mathbb{Z}$
\end{teo}

\textbf{Demostración}


Sea $m$ múltiplo común de $a$ y $b$, entonces

$m=k_1a=k_2b$

$k_1\frac{a}{(a,b)}=k_2\frac{b}{(a,b)}$

Por tanto

$\frac{b}{(a,b)}|k_1\frac{a}{(a,b)}$ pero como $(\frac{a}{(a,b)},\frac{b}{(a,b)})=1$

$\frac{b}{(a,b)}|k_1$ luego existe $k\in\mathbb{Z}$ tal que $k_1=k\frac{b}{(a,b)}$

y por tanto $m=k_1a=k\frac{a*b}{(a,b)}$

\begin{cor}
 El $[a,b]=\frac{|a*b|}{(a,b)}$, lo que es lo mismo $(a,b)=\frac{|a*b|}{[a,b]}$
\end{cor}

\begin{cor}
 Todo múltiplo común de $a$ y $b$ es múltiplo común de $[a,b]$
\end{cor}














\end{document}          
