\documentclass[a4paper,12pt]{report}
\usepackage[utf8]{inputenc}
\usepackage{amsfonts}
\usepackage{amsthm}
\usepackage{amssymb}
\usepackage{amsmath}

\newtheorem*{pbo}{Principio del Buen Ordenamiento}

\newtheorem*{pim}{Principio de Inducción Matemática}

\newtheorem*{teo}{Teorema}

\newtheorem*{cor}{Corolario}

\newtheorem*{dem}{Demostración}

\newtheorem*{dfn}{Definición}

\newtheorem*{lem}{Lema}

\newtheorem*{prp}{Propiedades}

\newtheorem*{pro}{Proposición}

% Title Page
\title{Conferencia 7 - Raíces Primitivas}
\author{}



\begin{document}
\maketitle

%\begin{abstract}
%\end{abstract}


\begin{dfn}
Sean $a,n\in\mathbb{Z}_+$, $(a,n)=1$, el menor entero positivo tal que $a^k \equiv 1 \, (n)$ se denomina
\textbf{orden} de $a$ módulo de $n$ y se denota $ord_na$
\end{dfn}

Note que $ord_na\leq\varphi(n)$

\begin{teo}
 Sean $a,n\in\mathbb{Z}_+$, $(a,n)=1$, y $ord_na=e$ entonces $a^t \equiv 1 \, (n)$ si y solo si $e|t$
\end{teo}

\textbf{Demostración}

Si $e|t$ entonces $t=eq$, además $a^e \equiv 1 \, (n)$, por definición de orden

luego $(a^e)^q \equiv 1 \, (n)$ y, por tanto, $a^{eq}\equiv\, a^t  \equiv 1 \, (n)$ 
\\

En el otro sentido, asumamor que $t=eq+r$ donde 

si $r\neq 0$ entonces $0<r<e$

entonces $a^t=a^{eq+r}=a^{eq}a^r$ pero como  $a^t \equiv 1 \, (n)$ 

se tendría que $a^{eq} \equiv 1 \, (n)$ y $a^r \equiv 1 \, (n)$   y cómo $r<e$ 

esto sería una contradicción por la definición de orden

luego $r=0$ y $t=eq$ por lo que $e|t$

\begin{cor}
 Sean $a,n\in\mathbb{Z}_+$, $(a,n)=1$ entonces $ord_na|\varphi(n)$
\end{cor}


\begin{teo}
 Sean $a,n\in\mathbb{Z}_+$, $(a,n)=1$, y $ord_na=e$ entonces $a^i \equiv a^j \, (n)$ si y solo si
 $i \equiv j \, (e)$
\end{teo}

\textbf{Demostración}

Si $a^i \equiv a^j \, (n)$ entonces $a^{i-j} \equiv 1 \, (n)$ luego, por definición de orden,

$e|i-j$ y, por tanto,  $i \equiv j \, (e)$
\\

En el otro sentido, como $i \equiv j \, (e)$ se tiene que $e|i-j$

luego, por defición de orden, $a^{i-j} \equiv 1 \, (n)$, entonces $a^{i-j}a^j \equiv a^j \, (n)$,

por tanto $a^i \equiv a^j \, (n)$


\begin{dfn}
 Sean $a,n\in\mathbb{Z}_+$, $(a,n)=1$, $a$ es \textbf{raíz primitiva} módulo $n$ si $ord_na=\varphi(n)$ 
\end{dfn}

\begin{teo}
 Sean $a,n\in\mathbb{Z}_+$, $(a,n)=1$ y $ord_na=e$ entonces $ord_na^k=\frac{e}{(e,k)}$
\end{teo}

\textbf{Demostración}

Sea $m=ord_na^k$  y $d=(e,k)$

entonces $k=dk_1$ y $e=de_1$ tal que $(k_1,e_1)=1$

luego  $a^e \equiv 1 \, (n)$ y  $(a^k)^m \equiv (a^{dk_1})^m \,\equiv a^{dk_1m} \, \equiv 1 \, (n)$

entonces $e|dk_1m$ luego $e_1d|dk_1m$ por lo que $e_1|k_1m$ 

como $(e_1,k_1)=1$ entonces $e_1|m$

Por otra parte, $(a^k)^{e_1} \equiv a^{dk_1e_1} \, \equiv (a^e)^{k_1} \,\equiv 1 \, (n)$

por lo que $m|e_1$ y como  $m|e_1$ y $e_1|m$ entonces $m=e_1$ 

por tanto $m=e_1=\frac{e}{(e,k)}$

\begin{teo}
 Sean $a,n\in\mathbb{Z}_+$, $(a,n)=1$ y $a$ raíz primitiva módulo $n$ entonces
 $\{a,a^2,a^3,\dots,a^{\varphi(n)}\}$ es un SRR(n)
\end{teo}

\textbf{Demostración}

Para todo $a^i$, $1\leq i \leq \varphi(n)$, se tiene que $(a^i,n)=1$

Supongamos que existen $i\neq j$ tales que $a^i \equiv a^j \, (n)$

pero también se tiene que $ord_na=\varphi(n)$, pues es $a$ raíz primitiva

y como $a^i \equiv a^j \, (n)$ entonces $i \equiv j \, (\varphi(n))$ luego $\varphi(n)|i-j$

pero como $i-j<\varphi(n)$ entonces $i-j=0$ por lo que $i=j$ 

entonces  $\{a,a^2,a^3,\dots,a^{\varphi(n)}\}$ es un SRR(n)


\begin{teo}
 Sea $n\in\mathbb{Z}_+$, si $n$ tiene raíces primitivas entonces $n$ tiene $\varphi(\varphi(n))$ raíces primitivas
\end{teo}

\textbf{Demostración}

Si $n$ tiene raíces primitivas, y sea $a$ una de esas raíces

entonces $\{a,a^2,a^3,\dots,a^{\varphi(n)}\}$ es un SRR(n)

Supongamos que $b$ es otra raíz primitiva de $n$, luego $(b,n)=1$

por tanto existe un $i$ entero, $1\leq i \leq \varphi(n)$  tal que $b \equiv \, a^i (n)$

por lo que $ord_nb=ord_na^i$ 

luego $\varphi(n)=ord_nb=ord_na^i=\frac{ord_na}{(ord_na,i)}=\frac{\varphi(n)}{(\varphi(n),i)}$ 

entonces $(\varphi(n),i) =1$

Por tanto, para toda raíz primitiva $b$ de $n$ existe $i$ tal que $b \equiv \, a^i (n)$ 

pero como $(\varphi(n),i) =1$, por consiguiente, hay $\varphi(\varphi(n))$ elementos en

$\{a,a^2,a^3,\dots,a^{\varphi(n)}\}$ que son primos relativos con $\varphi(n)$

por tanto, hay $\varphi(\varphi(n))$ raíces primitivas de $n$



\end{document}       



