\documentclass[a4paper,12pt]{report}
\usepackage[utf8]{inputenc}
\usepackage{amsfonts}
\usepackage{amsthm}
\usepackage{amssymb}

\newtheorem*{pbo}{Principio del Buen Ordenamiento}

\newtheorem*{pim}{Principio de Inducción Matemática}

\newtheorem*{teo}{Teorema}

\newtheorem*{cor}{Corolario}

\newtheorem*{dem}{Demostración}

\newtheorem*{dfn}{Definición}

\newtheorem*{lem}{Lema}

\newtheorem*{prp}{Propiedades}

\newtheorem*{pro}{Proposición}

% Title Page
\title{Conferencia 4 - Congruencia}
\author{}



\begin{document}
\maketitle

%\begin{abstract}
%\end{abstract}

\begin{dfn}
 Sea $n\in \mathbb{Z}_+$, $a\in \mathbb{Z}$, $b\in \mathbb{Z}$, se dice que $a$ es congruente con $b$ módulo $n$ si $a$ y $b$ tienen el mismo resto al ser divididos por $n$ y esto se denota $ a\equiv b \ (\textrm{mod}\ n)$ o $a\equiv b\, (n)$
\end{dfn}

\begin{teo}
 Sea $n\in \mathbb{Z}_+$, $a\in \mathbb{Z}$, $b\in \mathbb{Z}$, se dice que $a \equiv b \, (n)$ si y solo si $n|a-b$
\end{teo}

\textbf{Demostración}

Demostremos que si $a\equiv b\, (n)$ entonces $n|a-b$

Como $a\equiv b\, (n)$, por definición, $a=kn+r$ y $b=qn+r$  luego 

$a-kn=b-qn$ 

$a-b=kn-qn$

$a-b=(k-q)n$

por lo que $n|a-b$
\\

Demostremos ahora que si $n|a-b$ entonces $a\equiv b\, (n)$

Tenemos que 

$a=nq_1+r_1$ y $b=nq_2+r_2$

luego $a-b=n(q_1-q_2) + r_1 - r_2$

Ahora, como $n|a-b$ y $n|n(q_1-q_2)$

entonces $n|r_1-r_2$ y $n||r_1-r_2|$

pero $|r_1-r_2|<n$ por lo que $r_1-r_2=0$

entonces $r_1=r_2$ y, por tanto, $a\equiv b\, (n)$

\begin{teo}
 La relación de congruencia módulo $n$ es una relación de equivalencia
\end{teo}


\textbf{Propiedades básicas de la congruencia}
\begin{enumerate}
 \item Para todo $a$, $a \equiv a \, (n)$
 
 \textbf{Demostración}
 
 Como $a-a = 0 = n*0$ entonces $n|a-a$
 
 \item Si $a \equiv b \, (n)$ si y solo si $b \equiv a \, (n)$
 
 \textbf{Demostración}
 
 Si $a-b=kn$ para algún $k$ luego $b-a=-kn$
 
 \item Si $a \equiv b \, (n)$ y $b \equiv c \, (n)$ entonces $a \equiv c \, (n)$
 
 \textbf{Demostración}
 
 Si $a-b=kn$ y $b-c=ln$ para $k,l$ enteros entonces $a-c=(k+l)n$
 
 \item Si $a \equiv b \, (n)$ y $c \equiv d \, (n)$ entonces $a\pm c \equiv b\pm d \, (n)$
 
 \textbf{Demostración}
 
 Si $a-b=kn$ y $c-d=ln$ para $k,l$ enteros entonces 
 
 $(a+c)-(b+d)=(k+l)n$ y $(a-c)-(b-d)=(k-l)n$
 
 \item Si $a \equiv b \, (n)$ y $k\in\mathbb{Z}_+$ entonces  $ak \equiv bk \, (n)$
 
 \textbf{Demostración}
 
 Se suma $k$ veces $a \equiv b \, (n)$
 
 \item Si $a \equiv b \, (n)$ y $c \equiv d \, (n)$ entonces $ac \equiv bd \, (n)$
 
 \textbf{Demostración}
 
 Para ello se debe demostrar que $ac - bd$ es múltiplo de $n$. Entonces 
 
 $ac - bd = ac - bc + bc - bd = c(a-b) + b(c-d)$
 
 y como $a-b$ y $c-d$ son múltiplos de $n$ entonces $ac - bd$ también lo es
 
 
 \item Si $a \equiv b \, (n)$ y $k\in\mathbb{Z}_+$ entonces $a^k \equiv b^k \, (n)$
 
  \textbf{Demostración}
 
 Se multiplica $k$ veces $a \equiv b \, (n)$
 
 \item Si $a \equiv b \, (n)$ entonces $a+c \equiv b+c \, (n)$
 
 \textbf{Demostración}
 
Se tiene que $a \equiv b \, (n)$ y también que $c \equiv c \, (n)$ luego $a+c \equiv b+c \, (n)$

 
 \item Si $c$ es divisor común de $a,b,n$ luego, si $a \equiv b \, (n)$ entonces $\frac{a}{c} \equiv \frac{b}{c} \, (\frac{n}{c})$
 
\textbf{Demostración}
 
 Como $c$ es divisor común de $a,b,n$ entonces para $a_1,b_1,n_1$ enteros se tiene que $a=ca_1$, $b=cb_1$ y $n=cn_1$ y, entonces, $ca_1 \equiv cb_1 \, (cn_1)$ luego $ca_1-cb_1=kcn_1$ para $k$ entero, lo que es lo mismo que $a_1-b_1=kn_1$, por lo que $a_1 \equiv b_1 \, (n_1)$ por tanto $\frac{a}{c} \equiv \frac{b}{c} \, (\frac{n}{c})$
 
 \item Si $c|n$ y $a \equiv b \, (n)$ entonces $a \equiv b \, (c)$

 \textbf{Demostración}
 
 Como $c|n$ entonces $n=qc$ con $q$ entero y como $a \equiv b \, (n)$ entonces $a-b=kn$ con $k$ entero, luego $a-b=kqc$ y como $kq$ es un entero entonces $a \equiv b \, (c)$
 
 
\end{enumerate}


\begin{teo}
 Si $ca \equiv cb \, (n)$ entonces $a \equiv b \, (\frac{n}{d})$ donde $d=mcd(c,n)$ 
\end{teo}

\textbf{Demostración}

Como $ca \equiv cb \, (n)$ entonces $ca-cb=c(a-b)=kn$, ahora si se tiene que $d=mcd(c,n)$  
entonces existen $s$ y $r$, $(r,s)=1$ tales que $c=dr$ y $n=ds$, si se sustituye en la igualdad previa se tiene que $dr(a-b)=kds$ y si se simplifica queda que $r(a-b)=ks$. 

A partir de esto se tiene que $s|r(a-b)$ y, como $r$ y $s$ son primos relativos, entonces  $s|a-b$ y, por tanto $a\equiv b \, (s)$ lo que es lo mismo que $a\equiv b \, (\frac{n}{d})$


\begin{cor}
 Si $ca \equiv cb \, (n)$ y $mcd(c,n)=1$ entonces $a \equiv b \, (n)$
\end{cor}

\textbf{Demostración}

Para $d=1$ se tiene entonces, a partir del teorema anterior que $a\equiv b \, (\frac{n}{1})$

\begin{cor}
 Si $ca \equiv cb \, (p)$ y $p\nmid c$ donde $p$ es primo, entonces $a \equiv b \, (p)$
\end{cor}

\textbf{Demostración}

Como $p\nmid c$ y $p$ es primo, entonces $mcd(c,p)=1$, y entonces se tienen el corolario anterior


\textbf{Propiedades fuertes de la congruencia}
\begin{enumerate}
 \item Si $a \equiv b \, (m)$ y $a \equiv b \, (n)$ entonces $a \equiv b \, (mcm(m,n))$
 
 \textbf{Demostración}
 
 Por el \textbf{Teorema Fundamental de la Aritmética} 
 
 $m=\prod_pp^{m_p}$, $n=\prod_pp^{n_p}$ y $mcm(m,n)=\prod_pp^{max(m_p,n_p)}$
 
 donde $p$ son números primos, $m_p\geq 0$ y $n_p\geq 0$.
 
 Entonces, si $a \equiv b \, (m)$ y $a \equiv b \, (n)$ esto significa que $p^{m_p}|a-b$ y que $p^{n_p}|a-b$ y, por tanto, $p^{max(m_p,n_p)}|a-b$.
 
 Ahora como $a-b=\prod_pp^{c_p}$ donde $c_p\geq max(m_p,n_p)$ entonces 
 
 $mcm(m,n)|a-b$ y, por tanto, 
 $a\equiv b \, (mcm(m,n))$
 
 \item Si $a \equiv b \, (n)$ entonces $mcd(a,n)=mcd(b,n)$
 
  \textbf{Demostración}
  
  Si $a \equiv b \, (n)$ entonces $a-b=kn$ para $k$ entero, luego $a=k*n+b$ y por tanto $mcd(a,n)=mcd(n,b)$
 
 \item Si $ac \equiv bd \, (n)$, $c \equiv d \, (n)$ y $mcd(c,n)=1$ entonces $a \equiv b \, (n)$
 
 \textbf{Demostración}
 
 Como $c \equiv d \, (n)$ entonces $mcd(d,n)=mcd(c,n)=1$ y  $c-d=qn$, lo que es lo mismo que $c=qn+d$
 
 Ahora, como $ac \equiv bd \, (n)$ entonces $ac-bd=kn$ y sustituyendo $c$ se tiene que 
 
 $a(qn+d)-bd=kn$
 
 $aqn+ad-bd=kn$
 
 $ad-bd=kn-aqn$
 
 $ad-bd=(k-aq)n$
 
 entonces $da\equiv db \, (n)$ y como $mcd(d,n)=1$ entonces $a\equiv b\, (n)$
 
 \item Sea $f(x)$ un polinomio con coeficientes enteros $a \equiv b \, (n)$ entonces 
 $f(a) \equiv f(b) \, (n)$
 
  \textbf{Demostración}
  
  $f(x)$ de manera general se puede definir como $f(x)=\sum^{m}_{k=0}c_kx^k$
  
  Como $a \equiv b \, (n)$ entonces se puede tener $a^k \equiv b^k \, (n)$ luego se pueden multiplicar por un entero tal que $c_ka^k \equiv c_kb^k \, (n)$ y estas se pueden sumar varias veces de modo que $\sum^{m}_{k=0}c_ka^k \equiv \sum^{m}_{k=0}c_kb^k \, (n)$ y como $f(a)=\sum^{m}_{k=0}c_ka^k$ y $f(b)=\sum^{m}_{k=0}c_kb^k$ entonces $f(a)\equiv f(b) \, (n)$
  
  
\end{enumerate}

\begin{dfn}
 Si $f(x)$ es un polinomio con coeficientes enteros se dice que $a$ es solución de $f(x)\equiv 0 \, (n)$ si $f(a)\equiv 0 \, (n)$
\end{dfn}


\begin{teo}
 Sea $f(x)$ un polinomio con coeficientes enteros tal que $a$ es solución de  $f(x)\equiv 0 \, (n)$  y $a \equiv b \, (n)$, entonces $b$ también es solución
\end{teo}

\textbf{Demostración}

Por el teorema anterior se tiene que $f(a)\equiv f(b) \, (n)$ entonces se tiene que $f(b)\equiv f(a) \equiv 0\, (n)$ y, por tanto, $b$ es solución




\begin{teo}
 \textbf{Pequeño Teorema de Fermat}. Sea $p$ primo y $a\in\mathbb{Z}$, luego si $p\nmid a$ entonces $a^{p-1} \equiv 1 \, (p)$
\end{teo}

\textbf{Demostración}

Si se tienen los primeros $p-1$ múltiplos positivos de $a$, que serían $a,2a,3a,\dots,(p-1)a$, ninguno de ellos es congruente con otro módulo $p$ pues si eso pasara entonces se tendría $ra\equiv sa \, (p)$, $1\leq r < s \leq p-1$, lo que resultaría en que $r\equiv s \, (p)$ lo que es falso.

Entonces el conjunto de múltiplos debe ser cogruente módulo $p$ con $1,2,3,\dots,p-1$, en algún orden. Luego, si múltiplicamos todas estas congruencias se tiene:

$a*2a*3a*\dots *(p-1)a \equiv 1*2*3*\dots *(p-1) \, (p)$

Lo que es lo mismo que:

$(p-1)!a^{p-1} \equiv (p-1)!\, (p)$

luego $a^{p-1} \equiv 1\, (p)$


% \begin{pro}
%  Sea $n\in \mathbb{Z}$, si $a$ es primo relativo con $n$ (o sea, $mcd(a,n)=1$) entonces existe un entero $b$ tal que $ab \equiv 1\, (n)$ (Se dice que $a$ es \textbf{invertible} y que $b$ es un \textbf{inverso} de $a$ módulo $n$). Recíprocamente, si $a$ y $b$ son enteros tales que $a \equiv 1 \, (n)$ entonces $a$ y $n$ no tienen factores en común (o sea, $mcd(a,n)=1$)
% \end{pro}
% 
% 
% 
% \begin{teo}
%  \textbf{Teorema de Wilson}. Sea $p$ entero mayor que 1, $p$ es primo si y solo si $(p-1)!\equiv -1\, (p)$
% \end{teo}













\end{document}          
