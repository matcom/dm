\documentclass[a4paper,12pt]{report}
\usepackage[utf8]{inputenc}
\usepackage{amsfonts}
\usepackage{amsthm}
\usepackage{amsmath}
\usepackage{amssymb}

\newtheorem*{pbo}{Principio del Buen Ordenamiento}
\newtheorem*{pim}{Principio de Inducción Matemática}
\newtheorem*{psm}{Principio de la Suma}
\newtheorem*{ppr}{Principio del Producto}
\newtheorem*{pie}{Principio de Inclusión - Exclusión}
\newtheorem*{ppa}{Principio del Palomar}
\newtheorem*{pin}{Principio Inyectivo}
\newtheorem*{pso}{Principio Sobreyectivo}

\newtheorem*{teo}{Teorema}

\newtheorem*{cor}{Corolario}

\newtheorem*{dem}{Demostración}

\newtheorem*{dfn}{Definición}

\newtheorem*{lem}{Lema}

\newtheorem*{prp}{Propiedades}

\newtheorem*{pro}{Proposición}

% Title Page
\title{Conferencia 10 - Relaciones de Recurrencia}
\author{}



\begin{document}
\maketitle

%\begin{abstract}
%\end{abstract}


\begin{dfn}
 Una relación de recurrencia para la sucesión $a_1, a_2,\dots, a_n$ es una ecuación que relaciona el término n-ésimo con alguno(s) de sus predecesores
\end{dfn}

\textbf{Ejemplos}

\begin{enumerate}
 \item Progresión Artmética: $a_n=a_{n-1}+d$
 \item Progresión Geométrica: $a_n=q*a_{n-1}$
 \item Torres de Hanoi: $a_n = 2a_{n-1}+1,a_0=0$
\end{enumerate}

\textbf{Ejercicio}

Se tienen $n$ líneas que se pondrán en el plano. No hay rectas paralelas no
hay un punto del plano donde se corten 3 o más líneas.

¿En cuántas regiones queda dividido el plano luego de pintar todas las líneas?

$a_0=1$

$a_1=2$

$a_2=4$

$\dots$

$a_n=a_{n-1}+n$

Cuando se pinta la primera línea el plano se divide en 2 regiones.

Cuando se pinta la línea n-ésima el plano está dividido en $a_{n-1}$ regiones ya, y la nueva línea corta a las $n-1$ rectas ya existentes~(no hay 2 rectas paralelas), por tanto por cada l\'inea que pasa la divide en 2 nuevas regiones y pasa por $(n-1)+1$ regiones (por el final pasa cuando entra y cuando sale), luego $a_n=a_{n-1}+n$.\\


\textbf{Ejercicio}

¿Cómo repartir $n$ objetos iguales en $k$ categorías distintas?

Denotemos $a(n,k)$ como el número de formas de hacerlo.

Suponga que las categorías son cajas, luego, si en la primera caja no se pone ningún objeto entonces se resuelve en $a(n,k-1)$, que es lo mismo que repartir $n$ objetos en $k-1$ cajas.

Ahora, si en la primera caja se pone al menos un elemento entonces se resuelve como $a(n-1,k)$

Entonces $a(n,k)=a(n,k-1)+a(n-1,k)$ con $n\geq 0$,$k\geq 1$ donde $a(n,1)=1$ y $a(0,k)=1$.\\

\textbf{Ejercicio}

Una compañía tiene 2 oficinas, A y B, y su negocio es rentar carros para recorridos locales. Al final de cada mes la mitad de los carros de la oficina A terminan en B y un tercio de los carros de B terminan en A. Si se conocen los valores iniciales de los carros que hay en A($c_A$) y en B($c_B$), al transcurrir $n$ meses cuántos carros hay en A y en B respectivamente.

$$A_n=\frac{A_{n-1}}{2}+\frac{B_{n-1}}{3}$$

$$B_n=\frac{2B_{n-1}}{3}+\frac{A_{n-1}}{2}$$

donde $A_1=c_A$ y $B_1=c_B$.\\

\textbf{Ejercicio-Problema de Josephus}

Hay gente de pie en un círculo a la espera de ser ejecutada. La cuenta empieza en un punto y dirección específica del círculo. Después de que se haya saltado un número específico de personas, la siguiente persona es ejecutada. El procedimiento se repite con las personas restantes, a partir de la siguiente persona, que va en la misma dirección y omitiendo el mismo número de personas, hasta que solo una persona permanece y se libra de la ejecución.

Si hay $n$ personas y se ejecutan alternadamente ¿cuál sobrevive?

Tomemos $a_n$ como la persona sobreviviente.

$n$ es par, si el número inicial de personas es par, entonces la persona en la posición $x$ durante la segunda vuelta alrededor del círculo estaba en la posición $2x-1$~(para cualquier valor de $x$). Entonces, si $n=2j$ la persona en $a_j$ que ahora sobrevive, inicialmente estaba en la posición $2a_j-1$. Luego tiene como recurrencia a $a_{2n}=2a_{n}-1$

Si el número de personas inicial es impar, entonces pensamos que las personas en la primera posición morirán al final de la primera vuelta alrededor del círculo. En este caso, la persona en la posición $x$ estaba originalmente en la posición $2x+1$. Esto da la recurrencia $a_{2n+1}=2a_n+1$.

El caso base es siempre $a_1=1$.\\

\textbf{Clasificación de la Relaciones de Recurrencia}
 \\

 \textbf{De orden K}
 
 Ejemplo: $a_n=a_{n-1}+a_{n-2}+\dots+a_{n-k}$\\
 
  \textbf{Con coeficientes constantes}
  
  Ejemplo: $a_n=2a_{n-1}+3a_{n-2}$\\

  \textbf{Con coeficientes variables}
  
  Ejemplo: $a_n=(n-1)a_{n-1}$\\
  
  \textbf{Lineales}
  
  Ejemplo: $a_n=2a_{n-1}$\\
  
  \textbf{No Lineales}
  
  Ejemplo: $a_n=2a_{n-1}^2$\\
  
  \textbf{Homogéneas}
 
 Ejemplo: $a_n=a_{n-1}+a_{n-2}$\\
 
 \textbf{No Homogéneas}
 
 Ejemplo: $a_n=a_{n-1}+a_{n-2} + n$\\
 
 
 Si se generaliza para: 
 
 $a_n=c_1a_{n-1}+c_2a_{n-2}+\dots+c_ka_{n-k}+f(n)$
 
 si $f(n)=0$ es homogénea
 
 si $f(n)\neq 0$ no es homogénea
 
 
 \begin{dfn}
  Resolver una relación de recurrencia es expresarla en su forma cerrada, es decir, expresarla en una fórmula que solo necesite el valor de $n$ para computarla.
 \end{dfn}

 
 \begin{teo}
  Sea $f$ tal que $f:\mathbb{R}^k\rightarrow \mathbb{R}$ y sean $c_0,c_1,c_2,\dots,c_{k-1}\in\mathbb{R}$ valores dados, entonces existe una y solo una secuencia $a_0,a_1,\dots,a_n,\dots$ que satisface que $a_n=f(a_{n-1},a_{n-2},\dots,a_{n-k})$ con $n\geq k$ y donde $a_0=c_0$, $a_1=c_1$, $a_2=c_2$, $\dots$, $a_{k-1}=c_{k-1}$
 \end{teo}

 \begin{teo}
  Sea $q\in \mathbb{R}$, $q>0$, la sucesión $\{q^n\}$ satisface la ecuación de recurrencia $a_n=c_1a_{n-1}+c_2a_{n-2}$ si y solo si $q$ es raíz de la ecuación 
  
  $x^2-c_1x-c_2=0$
 \end{teo}
 
  $x^2-c_1x-c_2=0$ se conoce como polinomio o ecuación característica de $a_n=c_1a_{n-1}+c_2a_{n-2}$ \\

  
   \textbf{Demostración}

   Como $\{q^n\}$ satisface $a_n=c_1a_{n-1}+c_2a_{n-2}$ entonces $q^n=c_1q^{n-1}+c_2q^{n-2}$, por tanto
   $q^2=c_1q+c_2$ luego $q^2-c_1q-c_2=0$ entonces $p(x)=x^2-c_1x-c_2=0$ tiene como raíz a $q$.\\
   
   Por otra parte, si $p(x)=0$ se tiene que $p(q)=q^2-c_1q-c_2=0$ por lo que $q^2-c_1q-c_2=0$, luego, multiplicando por $q^{n-2}$ y despejando, se tiene que 
   $q^n=c_1q^{n-1}+c_2q^{n-2}$ y, por tanto, $\{q^n\}$ satisface $a_n=c_1a_{n-1}+c_2a_{n-2}$
   
\begin{teo}
 Sean $\{x_n\}$ y $\{y_n\}$ soluciones de la relación de recurrencia\\ 
 $a_n=c_1a_{n-1}+c_2a_{n-2}$, entonces $Ax_n + By_n$  $(A,B\in\mathbb{R})$ es solución de la ecuación de recurrencia dada.
\end{teo}

\textbf{Demostración}

Como $\{x_n\}$ es solución entonces $x_n=c_1x_{n-1}+c_2x_{n-2}$ y

$Ax_n=Ac_1x_{n-1}+Ac_2x_{n-2}$

Como $\{y_n\}$ es solución entonces $y_n=c_1y_{n-1}+c_2y_{n-2}$ y

$By_n=Bc_1y_{n-1}+Bc_2y_{n-2}$

Entonces $Ax_n+By_n=c_1(Ax_{n-1}+By_{n-1})+c_2(Ax_{n-2}+By_{n-2})$

luego $Ax_n+By_n$ es solución

\begin{teo}
 Sean $q_1$ y $q_2$ soluciones de la ecuación $x^2-c_1x-c2=0$ tal que $q_1\neq q_2 \neq 0$, entonces $x_n$ es solución de la relación $a_n=c_1a_{n-1}+c_2a_{n-2}$ donde $x_n=Aq^n_1+Bq^n_2$
\end{teo}

\textbf{Demostración}

Por el Teorema previo se tiene entonces que las sucesiones de la forma $x_n$ son solución

Ahora se debe demostrar que cualquier solución es de esta forma, que es equivalente a demostrar que el sistema siguiente tiene una única solución

$Aq_1+Bq_2=a_1$

$Aq^2_1+Bq^2_2=a_2$

para ello el determinante debe ser distinto de 0 y se cumple pues

$q_1q^2_2-q_2q^2_1=q_1q_2(q_2-q_1)\neq0$




\begin{teo}
 Sea $q\in\mathbb{R}$, $q\neq 0$ única raíz de la ecuación $x^2-c_1x-c_2=0$, entonces la sucesión $nq^n$  es solución de la relación de recurrencia $a_n=c_1a_{n-1}+c_2a_{n-2}$
\end{teo}

\textbf{Demostración}

$q$ es raíz de $x^2-c_1x-c_2$ luego $q^2-c_1q-c_2=0$

como $q$ es raíz única por Vieta se tiene que $2q=c_1$ y $q^2=-c_2$

entonces $a_n=2qa_{n-1}-q^2a_{n-2}$

Luego 

$nq^n=2q(n-1)q^{n-1}-q^2(n-2)q^{n-2}$

$nq^n=2(n-1)q^{n}-(n-2)q^{n}$

$nq^n=(2n-2-n+2)q^{n}$

$nq^n=nq^n$
   
\end{document}          
