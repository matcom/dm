\documentclass[a4paper,12pt]{report}
\usepackage[utf8]{inputenc}
\usepackage{amsfonts}
\usepackage{amsthm}
\usepackage{amssymb}

\newtheorem*{pbo}{Principio del Buen Ordenamiento}

\newtheorem*{pim}{Principio de Inducción Matemática}

\newtheorem*{teo}{Teorema}

\newtheorem*{cor}{Corolario}

\newtheorem*{dem}{Demostración}

\newtheorem*{dfn}{Definición}

\newtheorem*{lem}{Lema}

\newtheorem*{prp}{Propiedades}

\newtheorem*{pro}{Proposición}

% Title Page
\title{Conferencia 5 - Sistemas Residuales}
\author{}



\begin{document}
\maketitle

%\begin{abstract}
%\end{abstract}



\begin{dfn}
  Un \textbf{Sistema Residual Completo} módulo $n$, SRC(n), con $n\in\mathbb{Z}_+$, es un conjunto de $n$ enteros incongruentes módulo $n$
\end{dfn}



\begin{teo}
 Sean $n \in \mathbb{Z}_+$, $k\in\mathbb{Z}$, $(k,n)=1$ y $\{a_1,a_2,\dots,a_n\}$ un sistema residual completo módulo $n$, entonces $\{ka_1,ka_2,\dots,ka_n\}$ es también un sistema residual completo módulo $n$.
\end{teo}

\textbf{Demostración}

Supongamos que $\{ka_1,ka_2,\dots,ka_n\}$ no es un SRC(n)

entonces existen $i$,$j$ tales que $ka_i\equiv ka_j \, (n)$

como $(k,n)=1$ entonces $a_i\equiv a_j \, (n)$

luego $\{a_1,a_2,\dots,a_n\}$ tampoco es un SRC(n), 

por tanto, por contrarecíproco, si $\{a_1,a_2,\dots,a_n\}$ es un SRC(n) 

entonces $\{ka_1,ka_2,\dots,ka_n\}$ también lo es

\begin{dfn}
 Una ecuación de la forma $ax\equiv b \, (n)$ con $a,b\in\mathbb{Z}$ y $n\in\mathbb{Z}_+$ es una 
 ecuación lineal congruencial si se trata de resolver en enteros. Dos soluciones se consideran distintas si son incongruentes módulo $n$.
\end{dfn}

\begin{teo}
 La ecuación lineal congruencial $ax\equiv b \, (n)$ es soluble si y solo si $(a,n)|b$
\end{teo}

\textbf{Demostración}

$ax\equiv b \, (n)$ tiene solución si existe $x_0$ tal que $ax_0\equiv b \, (n)$

entonces $n|ax_0-b$ por lo que existe $y_0$ tal que $ax_0-b=ny_0$

entonces como $ax_0-ny_0=b$ 

esta ecuación tiene solución si y solo si $(a,n)|b$\\


Note que si $x_0$ es solución de  $ax\equiv b \, (n)$ y $x_1\equiv x_0\, (\frac{n}{mcd(a,n)})$ entonces $x_1$ es también solución.\\

\textbf{Ejemplo}

$3x\equiv 9\, (7)$

$3x\equiv 2\, (7)$

y se cumple que $mcd(3,7)|2$

por tanto $3x-7q=2$ y $x=3$ y $q=1$ son solución

por lo que $x\equiv 3\, (7)$



\begin{teo}
 La ecuación lineal congruencial $ax\equiv b \, (n)$ donde $d=(a,n)$ y $d|b$ tiene exactamente $d$ soluciones\\
\end{teo}



\textbf{Demostración}

Ya se observó que la ecuación de congruencia lineal es equivalente a la ecuación lineal Diofantina $ax-ny=b$ y esta ecuación se resuelve si $(a,n)|b$ y como $d=(a,n)$ entonces $d|b$.

Esta ecuación tiene entonces las soluciones $x=x_0 + \frac{n}{d}t$ $y=y_0 + \frac{a}{d}t$ donde $x_0$ y $y_0$ es una solución de la ecuación Diofantina.

Si se considera $t=0,1,2,\dots,d-1$ entonces 

$x_0$, $x_0 + \frac{n}{d}$, $x_0 + \frac{2n}{d}$,\dots, $x_0 + \frac{(d-1)n}{d}$ son soluciones.

Ahora hay que verificar que estas $d$ soluciones son incongruentes entre ellas y cualquier otra fuera de ellas es congruente con alguna de ellas.

Verifiquemos lo primero, si asumimos que no se cumple entonces 

$x_0 + \frac{t_1n}{d} \equiv x_0 + \frac{t_2n}{d} \, (n)$ con $0\leq t_1 < t_2 \leq d-1$

entonces se tiene que $\frac{t_1n}{d} \equiv \frac{t_2n}{d} \, (n)$

como se tiene que $(\frac{n}{d},n)=\frac{n}{d}$ luego se llega a que $t_1\equiv t_2 \, (d)$

y esto implica que $d|t_2-t_1$ pero esto es una contradicción pues se cumple que $0<t_2-t_1<d$

Ahora hay que demostrar que cualquier otra solución $x_0 + \frac{n}{d}t$ 

es congruente módulo $n$ con una de las soluciones $x_0$, $x_0 + \frac{n}{d}$,\dots, $x_0 + \frac{(d-1)n}{d}$

Por el Algoritmo de la División  $t=qd+r$ donde $0 \leq r \leq d-1$

entonces $x_0 + \frac{n}{d}t = x_0 + \frac{n}{d}(qd+r) =  x_0 + nq + \frac{n}{d}r$

por tanto $x_0 + \frac{n}{d}t \equiv x_0 + nq + \frac{n}{d}r \equiv x_0 + \frac{n}{d}r \, (n)$

y $x_0 + \frac{n}{d}r$ es una de las soluciones de referencia\\

\textbf{Ejemplo}

$18x\equiv 30 (42)$ como $(18,42)=6$ y $6|30$ entonces la ecuación 

tiene exactamente 6 soluciones inconguentes entre ellas.

Como una solución de la ecuación es 4 entonces las 6 soluciones 

son  de la forma $x\equiv 4 + t\frac{42}{6} \equiv 4 +7t \, (42)$ con $t=0,1,\dots,5$

lo que es $x\equiv 4, 11, 18, 25, 32, 39 \, (42)$



\begin{cor}
 Si $mcd(a,n)=1$ entonces la ecuación lineal congruencial $ax\equiv b \, (n)$ tiene una única solución módulo $n$
\end{cor}

\begin{teo}
 \textbf{Teorema Chino del Resto} Sean $n_1,n_2,\dots,n_k$ enteros positivos primos relativos 2 a 2, entonces el sistema de ecuaciones de congruencia lineal:\\
 $x\equiv a_1 \, (n_1)$\\
 $x\equiv a_2 \, (n_2)$\\
 $\dots\dots$\\
 $\dots\dots$\\
 $x\equiv a_k \, (n_k)$\\
 tiene una única solución módulo $(n_1 * n_2 * \dots *n_k)$
\end{teo}

\textbf{Demostración}

Se tiene $p=n_1*n_2*\dots *n_k$ y $p_j=\frac{p}{n_j}$ con $1\leq j \leq k$

como los $n_j$ son primos relativos 2 a 2 entonces $(n_j,p_j)=1$

por tanto existen $r_j$ y $s_j$ tales que $r_j n_j + s_j p_j=1$ luego $s_j p_j=-r_j n_j +1$

y con ello $p_jx\equiv 1\, (n_j)$ tiene solución única y si llamamos $s_j$ a esa solución 

se tiene que $p_j s_j\equiv 1\, (n_j)$

pero también se sabe que para $i\neq j$ se tiene que  $p_js_j\equiv 0\, (n_i)$

entonces si se conforma $A=\sum_{i=1}^k a_ip_is_i$ se tiene que $A\equiv a_i \, (n_i)$


Ahora hay que probar la unicidad de la solución, 

o sea que todas las soluciones son congruentes entre ellas. 

Asumamos que hay dos soluciones $x$ y $y$ diferentes,  

entonces se debe cumplir que

$x\equiv a_i\, (n_i)$

$y\equiv a_i\, (n_i)$

esto implica que $x-y\equiv 0\, (n_i)$

ahora como todos los $n_i$ son primos relativos entonces $n_1 n_2\dots n_k | x-y$

luego $x\equiv y\, (n_1 n_2\dots n_k)$ 

por tanto las soluciones son congruentes entre ellas, como\\

\textbf{Ejemplo}

Encuentra un número que deja resto 2,3,2 cuando se divide por 3, 5 y 7 respectivamente.

Se tiene el sistema:

$x\equiv 2 \, (3)$

$x\equiv 3 \, (5)$

$x\equiv 2 \, (7)$

Entonces se tiene $p=3*5*7=105$

Luego $p_1=105/3=35$, $p_2=105/5=21$ y $p_3=105/7=15$

A partir de esto se tienen las ecuaciones de congruencias lineal

$35x_1\equiv 1 \, (3)$ donde $x_1=2$ es solución

$21x_2\equiv 1 \, (5)$ donde $x_2=1$ es solución

$15x_3\equiv 1 \, (7)$ donde $x_3=1$ es solución

Luego $A=a_1 p_1 x_1 + a_2 p_2 x_2 + a_3 p_3 x_3 = 2*35*2 + 3*21*1 + 2*15*1 = 233$

Entonces $A=233\equiv 23 \, (105)$


\begin{dfn}
 Si $a\in\mathbb{Z}$ tal que $(a,n)=1$ entonces la solución de la ecuación de congruencia lineal $ax\equiv 1 \, (n)$ se llama inverso de $a$ módulo $n$ y se denota  \={a}  y se dice que $a$ es inversible módulo $n$
\end{dfn}

\textbf{Ejemplo}

$7x\equiv 1\, (31)$

$x\equiv 9\, (31)$

\={7}$\equiv 9\, (31)$\\



Note que se cumple que $\frac{a}{b}\equiv a$\={b}$\, (n)$

\textbf{Demostración}

Si $b$\={b}$\equiv 1 \, (n)$ entonces $ab$\={b}$\equiv a \, (n)$

ahora como $(b,n)=1$ entonces $a$\={b}$\equiv \frac{a}{b} \, (n)$ que es lo mismo que $\frac{a}{b}\equiv a$\={b}$  \, (n)$\\


Note también que el inverso módulo $n$ es único 

\textbf{Demostración}

Como $a$\={a}$\equiv 1 \, (n)$  entonces  $a$\={a}$b\equiv b \, (n)$

luego $x=$\={a}$b$ es solución de la ecuación $ax\equiv b \, (n)$ tal que $(a,n)=1$ 

y, por tanto, esta solución es única\\

Por otra parte, si se hace $n=p$ con $p$ primo, $a\in\mathbb{Z}$, $(a,p)=1$

entonces, por el \textbf{Pequeño Teorema de Fermat}, $a^{p-1}\equiv 1 \, (p)$

por lo que \={a}$=a^{p-2}$ pues $aa^{p-2}=a^{p-1}\equiv 1 \, (p)$ 

luego como $ax\equiv b \, (p)$ entonces $x\equiv a^{p-2}b \, (p)$

 \begin{pro}
  Sea $n\in \mathbb{Z}$, si $a$ es primo relativo con $n$ (o sea, $mcd(a,n)=1$) entonces existe un entero $b$ tal que $ab \equiv 1\, (n)$.
  %(Se dice que $a$ es \textbf{invertible} y que $b$ es un \textbf{inverso} de $a$ módulo $n$). 
  Recíprocamente, si $a$ y $b$ son enteros tales que $ab \equiv 1 \, (n)$ entonces $a$ y $n$ no tienen factores en común (o sea, $mcd(a,n)=1$)
 \end{pro}
 
 
 
 \begin{teo}
  \textbf{Teorema de Wilson}. Sea $p$ entero mayor que 1, $p$ es primo si y solo si $(p-1)!\equiv -1\, (p)$
 \end{teo}
 
 \textbf{Demostración}
 
 Demostremos primero que si $p|(p-1)! + 1$ entonces $p$ es primo
 
 Asumamos que existe $d$ tal que $d|p$~(o sea, $p$ no es primo) con $1<d<p$
 
 por tanto $d\leq p-1$ por lo que $d|(p-1)!$
 
 pero como $d|(p-1)!+1$ entonces $d|1$ por lo que $d=1$ 
 
 lo que contradice a $1<d<p$ y, por tanto, $p$ debe ser primo\\
 
 Demostremos ahora que si $p$ es primo entonces $p|(p-1)! + 1$
 
 Es fácil verificar que el teorema se cumple para $p=2,3$ 
 
 entonces tomemos $p>3$
 
 Un $SRC(p)=\{0,1,2,\dots,p-1\}$ y si se tiene $a\in SRC(p)$ 
 
 entonces si $(a,p)=1$  se tendría que $ax\equiv 1\, (p)$ tiene solución y es única
 
 Luego, con excepción del 0, para todo elemento de SRC(p) se tiene que hay un número del propio 
 conjunto que ambos multiplicados dejan resto 1.
 
 Ahora, si $a$ es una solución de $ax\equiv 1\, (p)$ se tendría que $p|a^2-1$ 
 
 o lo que es lo mismo $p|(a-1)(a+1)$ y como $a\in SRC(p)$  
 
 entonces $a$ es 1 o $a$ es $p-1$
 
 Entonces para el conjunto $S=\{2,\dots,p-2\}$ si $b$ es solución de $ax\equiv 1\, (p)$
 
 tal que $a\neq b$ y $a,b\in S$ luegp $2*3*\dots * (p-2)=(p-2)!\equiv 1 \, (p)$
 
 y esto es lo mismo que $(p-1)!\equiv p-1 \, (p)$ y como $p-1\equiv -1 \, (p)$
 
 entonces $(p-1)!\equiv -1 \, (p)$

\end{document}          
