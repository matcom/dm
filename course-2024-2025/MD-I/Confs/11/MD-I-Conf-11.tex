\documentclass[a4paper,12pt]{report}
\usepackage[utf8]{inputenc}
\usepackage{amsfonts}
\usepackage{amsthm}
\usepackage{amssymb}
\usepackage{xcolor}

\newtheorem*{pbo}{Principio del Buen Ordenamiento}
\newtheorem*{pim}{Principio de Inducción Matemática}
\newtheorem*{psm}{Principio de la Suma}
\newtheorem*{ppr}{Principio del Producto}
\newtheorem*{pie}{Principio de Inclusión - Exclusión}
\newtheorem*{ppa}{Principio del Palomar}

\newtheorem*{teo}{Teorema}

\newtheorem*{cor}{Corolario}

\newtheorem*{dem}{Demostración}

\newtheorem*{dfn}{Definición}

\newtheorem*{lem}{Lema}

\newtheorem*{prp}{Propiedades}

\newtheorem*{pro}{Proposición}

%\color{blue!60!black}

% Title Page
\title{Conferencia 11 - Relaciones de Recurrencia}
\author{}



\begin{document}
\maketitle



%\begin{abstract}
%\end{abstract}


\begin{teo}
 Sea $q\in\mathbb{R}$ raíz de multiplicidad 2 de la ecuación $x^2-c_1x-c2=0$, entonces $x_n$ es solución de la relación de recurrencia $a_n=c_1a_{n-1}+c_2a_{n-2}$ si y solo si $x_n=Aq^n + Bnq^n$ 
\end{teo}

\textbf{Demostración}
Ya se vio que las sucesiones de la forma $x_n$ son solución

Ahora se debe demostrar que cualquier solución es de esta forma, que es equivalente a demostrar que el sistema siguiente tiene una única solución

$Aq+Bq=a_1$

$Aq^2+2Bq^2=a2$

para ello el determinante debe ser distinto de 0 y se cumple pues

$2q^3-q^3=q^3\neq0$

\begin{dfn}
 La ecuación característica de la relación de recurrencia\\ 
 $a_n=c_1a_{n-1}+c_2a_{n-2}+\dots+c_ka_{n-k}$ es de la forma\\ 
 $p(x)=x^k-c_1x^{k-1}+c_2x^{k-2}+\dots+c_k=0$
\end{dfn}

\begin{teo}
 Si la ecuación característica de la relación de recurrencia homogénea $a_n=c_1a_{n-1}+c_2a_{n-2}+\dots+c_ka_{n-k}$ tiene $k$ raíces distintas, entonces 
 $A_1q^n_1+A_2q^n_2+\dots+A_kq^n_k$ es solución de la relación, donde $q_i\, 1\leq i\leq k$ son raíces de la ecuación característica ($p(x)$). 
\end{teo}

\begin{teo}
 Sea $q\in\mathbb{R}$ raíz de multiplicidad $t$, $t\geq1$, de la ecuación característica de la relación de recurrencia $a_n=c_1a_{n-1}+c_2a_{n-2}+\dots+c_ka_{n-k}$, entonces $q^n,\, nq^n,\, n^2q^n,\,\dots,n^{t-1}q^n $ son soluciones de la relación de recurrencia.
\end{teo}

\begin{teo}
 Si la ecuación característica de la relación de recurrencia\\ 
 $a_n=c_1a_{n-1}+c_2a_{n-2}+\dots+c_ka_{n-k}$ tiene raíces $q_1,q_2,\dots,q_t$ con multiplicidades $m_1, m_2,\dots, m_t$, entonces la relación de recurrencia tiene como solución
 $P_1(n)q^n_1+P_2(n)q^n_2+\dots+P_t(n)q^n_t$ donde $P_i$ es un polinomio en $n$ de grado $m_i $
\end{teo}

\begin{dfn}
 Una solución particular de una relación de recurrencia es una sucesión que cumple la recurrencia aunque no satisfaga las condiciones iniciales
\end{dfn}

\begin{teo}
 Sea $P_n$ una solución particular de la relación de recurrencia\\ 
 $a_n=c_1a_{n-1}+c_2a_{n-2}+\dots+c_ka_{n-k}+f(n)$ entonces la solución general de la misma es
 $P_n+H_n$, donde $H_n$ es la solución de la relación homogénea asociada
\end{teo}

\textbf{Solución Particular}

Una solución particular $P_n$ se puede encontrar en algunos casos:

\begin{enumerate}
 \item Si $f(n)=T_k(n)$~(polinomio de grado $k$) entonces $P_n=Q_k(n)$~(polinomio de grado $k$), excepto si 1 es raíz característica con multiplicidad $s$, en cuyo caso $P_n=n^sQ_k(n)$
 \item Si $f(n)=ca^n$, $c\in\mathbb{R}$, entonces $P_n=qa^n$, $q\in\mathbb{R}$, excepto si $a$ es raíz característica con multiplicidad $s$, en cuyo caso $P_n=n^sqa^n$
 \item Si $f(n)=a^nT_k(n)$ entonces $P_n=a^nQ_k(n)$ excepto si $a$ es raíz característica con multiplicidad $s$, en cuyo caso $P_n=n^sa^nQ_k(n)$

\end{enumerate}


\textbf{Solución General}

La solución general se obtiene de la siguiente forma:

\begin{enumerate}
 \item Se calcula la solución general de la ecuación homogénea\\ 
 $a_n=c_1a_{n-1}+a-2c_{n-2}+\dots+c_ka_{n-k}$ 
 \item Se calcula una soluciín particular $P_n$ de la ecuación\\ 
 $a_n=c_1a_{n-1}+a-2c_{n-2}+\dots+c_ka_{n-k}+f(n)$ 
 \item La suma de ambas soluciones es una solución general de la ecuación\\
 $a_n=c_1a_{n-1}+a-2c_{n-2}+\dots+c_ka_{n-k}+f(n)$ 
 \item Se obtiene la solución correspondiente a las condiciones iniciales
 \end{enumerate}
 
\textbf{Ejemplo 1}

Sea la recurrencia $a_0=5,\, a_1=1,\, a_n=a_{n-1}+6a_{n-2}-6n^2+26n-25$\\

La relación homogénea tiene como polinomio característico a

$P(x)=x^2-x-6$  cuyas raíces son $q_1=-2\, q-2=3$ por lo que la solución 

general de la ecuación es  $a_n=A(-2)^n+B3^n$\\

Como $f(n)=-6n^2+26n-25$ es un polinomio de grado 2 entonces 

$P_n$ es de grado 2 por lo que se prueba una solución particular de la forma 

$P_n=an^2+bn+c$ que sustituida en la relación de recurrencia da una 

solución $a=1\, b=0\, c=0$ por lo que una solución particular es $P_n=n^2$\\

Entonces la solución general de la no homogénea es $a_n=A(-2)^n+B3^n+n^2$

Cómo las condiciones iniciales son $a_0=5=A+B$ y $a_1=1=-2A+3B+1$ 

entonces $A=3$ y $B=2$ y la solución de la recurrencia es 

$a_n=3(-2)^n+2*3^n+n^2$\\

\textbf{Ejemplo 2}

Sea la recurrencia lineal $a_0=0,\, a_1=1,\, a_n=a_{n-1}+6a_{n-2}+2^n$\\

La homogénea tiene como polinomio característico $P(x)=x^2-x-6$ 

cuyas raíces son $q_1=-2\, q-2=3$ por lo que la solución 

general de la ecuación es  $a_n=A(-2)^n+B3^n$\\

Como $f(n)=2^n$ y  $b=2$ no es raíz del polinomio característico entonces 

se puede probar una solución particular de la forma $P_n=c2^n$

que cuando se sustituye en la recurrencia da como solución $c=-1$

luego una solución particular de la recurrencia es $P_n=-2^n$\\

Entonces la solución general de la no homogénea es 

$a_n=A(-2)^n+B3^n-2^n$ y como las condiciones iniciales son 

$a_0=0=A+B-1$ y $A_1=1=-2A+3B-2$ entonces $A=0$ y $B=1$

por lo que la solución de la recurrencia es $a_n=3^n-2^n$\\

\textbf{Ejemplo 3}

Sea la recurrencia lineal $a_0=0,\, a_1=1,\, a_n=a_{n-1}+6a_{n-2}+3^n$\\

La homogénea tiene como polinomio característico $P(x)=x^2-x-6$ 

cuyas raíces son $q_1=-2\, q-2=3$ por lo que la solución 

general de la ecuación es  $a_n=A(-2)^n+B3^n$\\

cuyas raíces son $q_1=-2\, q-2=3$ por lo que la solución 

general de la ecuación es  $a_n=A(-2)^n+B3^n$\\

Como $f(n)=3^n$ y  $b=3$ es raíz del polinomio característico  

con  multiplicidad 1 entonces se prueba con una solución particular 

de la forma $P_n=cn3^n$ que cuando se sustituye en la recurrencia 

da $c=3/5$ entonces la solución particular queda $P_n=\frac{n3^{n+1}}{5}$\\

Entonces la solución general de la no homogénea queda 

$a_n=A(-2)^n+B3^n+\frac{n3^{n+1}}{5}$ que, evaluando en las condiciones iniciales,

queda $a_0=0=A+B$ y $a_1=1=-2A+(\frac{3}{5}+B)3$ luego 

$A=\frac{4}{25}$ y $B=-\frac{4}{25}$ por tanto la solución de la recurrencia es

$a_n=\frac{4}{25}(-2)^n+(\frac{15n-4}{25})3^n$\\

\textbf{Ejemplo 4}

Sea la recurrencia $a_1=1.\, a_n=2A_{n-1}+1$~(Torres de Hanoi)\\

La relación homogénea ($a_n=2A_{n-1}$) tiene como polinomio característico

$P(x)=x-2$ cuya raíz es $q=2$ luego la solución general de la homogénea 

es  $P_n=A2^n$\\

Entonces se prueba una solución particular de tipo $P_n=c$ que sustituida 

en la recurrencia da $c=-1$ luego la solución particular es $P_n=-1$

entonces la solución general de la homogénea es $a_n=A2^n-1$

como la conidición inicial es $a_1=1=2A-1$ entonces $A=1$

por lo que la solución de la recurrencia es $a_n=2^n-1$


\begin{teo}
 Si el término no homogéneo de la relación de recurrencia de orden $k$ es de la forma $P_1(n)S^n_1+P_2(n)S^n_2+\dots+P_n(n)S^n_n$ entonces hay una solución particular $f_1(n)+f_2(n)+\dots+f_n(n)$ donde $f_i(n)$ es solución particular de la recurrencia de orden $k$  $a_n=c_1a_{n-1}+c_2a_{n-2}+\dots+c_ka_{n-k}+P_i(n)S^n_i$
\end{teo}



 








 
 

\end{document}          
