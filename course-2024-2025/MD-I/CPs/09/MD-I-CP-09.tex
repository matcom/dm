\documentclass{article}
\usepackage{enumitem}
\usepackage{tikz}
\usetikzlibrary{trees}
\usepackage{amsmath} 
\usepackage{amssymb}
\usepackage{nopageno}
% \usepackage{listings}
% \usepackage{xcolor}

% \lstset{
% 	basicstyle=\ttfamily,
%     keywordstyle=\color{blue},
%     commentstyle=\color{green},
%     stringstyle=\color{red},
%     numbers=left,
%     numberstyle=\tiny,
%     stepnumber=1,
%     numbersep=5pt,
%     backgroundcolor=\color{gray!10},
%     frame=single,
%     tabsize=4,
%     breaklines=true
% }

\title{Clase pr\'actica 9}

\begin{document}

\maketitle
\begin{enumerate}
	\item Sean $A, B$ dos conjuntos tal que $|A| = n$ y $|B| = m$. Calcule el n\'umero de funciones totales sobreyectivas de $A$ en $B$.
	\item Sea $n$ un n\'umero entero positivo tal que $( n, 10 ) = 1$. Prueba que $\forall d \in \{1,2,3,4,5,6,7,8,9\}$ existen infinitos m\'ultiplos de $n$ que est\'an compuestos \'unicamente por el d\'igito $d$.
	\item Sea $A = \{1,2, \dots , 2n\}$ y $S$ un subconjunto de $A$ de tama\~no $n+1$. Prueba que existen dos elementos $a,b \in S$ tal que $a$ divide a $b$.
	\item Calcule el n\'umero de permutaciones del conjunto $\{1,2,3, \dots ,n\}$ donde ning\'un elemento est\'a en su posici\'on inicial.
	\item Determine el n\'umero de soluciones enteras de $x_1 + x_2 + x_3 + x_4 = 21$ con:
	\item[] 
	\begin{itemize}
		\item $2 \leq x_1 \leq 5$
		\item $3 \leq x_2 \leq 7$
		\item $0 \leq x_3 \leq 6$
		\item $2 \leq x_4 \leq 10$
	\end{itemize}
	\item Una compa\~n\'ia de baile tiene 11 semanas para prepararse para una competencia y decide practicar una vez al d\'ia pero no m\'as de 12 veces por semana. Prueba que existe un intervalo de d\'ias en que la compa\~n\'ia practica exactamente 21 veces.
	\item Deduzca la expresi\'on de $\phi (n)$ a partir del principio de inclusiones-exclusiones.
\end{enumerate}
\end{document}
