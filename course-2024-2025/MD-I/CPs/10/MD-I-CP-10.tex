\documentclass{article}
\usepackage{enumitem}
\usepackage{tikz}
\usetikzlibrary{trees}
\usepackage{amsmath} 
\usepackage{amssymb}
\usepackage{nopageno}
% \usepackage{listings}
% \usepackage{xcolor}

% \lstset{
% 	basicstyle=\ttfamily,
%     keywordstyle=\color{blue},
%     commentstyle=\color{green},
%     stringstyle=\color{red},
%     numbers=left,
%     numberstyle=\tiny,
%     stepnumber=1,
%     numbersep=5pt,
%     backgroundcolor=\color{gray!10},
%     frame=single,
%     tabsize=4,
%     breaklines=true
% }

\title{Clase pr\'actica 10}

\begin{document}

\maketitle
\date{}
\begin{enumerate}
	\item Encuentre una relaci\'on de recurrencia y resuelva (hallar su forma cerrada) en caso de que se pueda con los m\'etodos estudiados:
        \begin{enumerate}
            \item Cantidad de palabras de longitud $n$ que no contengan dos letras $a$ juntas (asuma que el alfabeto tiene $26$ letras).
            \item Cantidad de cadenas ternarias de longitud $n$ con una cantidad par de $0$s. 
            \item Cantidad de formas de descomponer a $n$ en sumandos positivos donde el orden de los sumandos es importante.
            \item Cantidad de formas de descomponer a $n$ en sumandos positivos donde el orden de los sumandos es importante y cada sumando es mayor que $1$.
            \item Una permutaci\'on se dice especial si $\forall i: 1 \leq i \leq n-1, \exists j: j > i$ tal que $|p_i - p_j| = 1$. Calcule el n\'umero de permutaciones especiales del conjunto $\{1,2,...,n\}$.
            \item Cantidad de cadenas ternarias de longitud $n$ que no tienen dos $0$ juntos ni dos $1$ juntos.
            \item Cantidad de desarreglos de tamaño $n$.
            \item Cantidad de cadenas de longitud $n$ sobre el alfabeto $\{a,b,c,d\}$ tal que todas las $a$ aparezcan antes que todas las $b$.
        \end{enumerate}
    \item Exprese de forma cerrada las siguientes recurrencias:
        \begin{enumerate}
            \item $a_n=5a_{n-1}+8-6a_{n-2}$ con $a_0 = 1$, $a_1=2$
            \item $a_n = 2a_{n-1} - a_{n-2} + 4*3^n + 4$ con $a_0=10$, $a_1 = 35$. 
        \end{enumerate}
    \item Consideremos una matriz de $n*n$, incialmente nos encontramos en la esquina inferior izquierda y queremos llegar a la esquina superior derecha, solo podemos viajar un segmento a la derecha o un segmento hacia arriba en cada paso.
        \begin{enumerate}
            \item De cu\'antas formas se puede llegar.
            \item De cu\'antas formas se puede llegar si solo se permite tocar pero no ir por arriba de una recta diagonal que va desde la esquina inferior izquierda hacian la esquina superior derecha. 
            \item Plantee una relaci\'on de recurrencia para el \'ultimo problema
            \item Note que relaci\'on hay entre este problema y la cantidad de cadenas de par\'entesis balanceados de tamaño $n$.
        \end{enumerate}
\end{enumerate}
\end{document}
