\documentclass{article}
\usepackage{enumitem}
\usepackage{tikz}
\usetikzlibrary{trees}
\usepackage{amsmath} 
\usepackage{nopageno}
\usepackage{hyperref}

\begin{document}

\title{Clase pr\'actica 1- Respuestas}
\maketitle

\begin{enumerate}
    \item Sea $k \in \mathbf{Z^+}$. Demuestre que $k$ divide a todo producto de $k$ enteros consecutivos. 
    \item[]  R(1) Aquí va la respuesta  \\

        Usa cuaciones
        \begin{equation}
            k \in \mathbf{Z^+}
        \end{equation}\label{eq:1}

        referencia las equaciones: como se dijo en la \href{1}{Proposicion 1}
    \begin{enumerate}
        \item Demuestre que $k!$ divide al producto de $k$ enteros consecutivos.
    \end{enumerate}
    \item Un entero $n>1$ es especial si para todo $k \in \mathbf{Z^+}$, con $k \leq n$ se puede escribir como suma de divisores distintos de $n$. Demuestre que si $p$ y $q$ son especiales entonces $pq$ es especial.
    \item Determine el n\'umero de formas de descomponer a $n$ en sumandos donde el orden no es relevante y la diferencia modular de cualquier par de sumandos es a lo sumo $1$.
    \item Demuestre que si $n \in \mathbf{Z^+}$ entonces $2^{2^n} - 1$ tiene al menos $n$ divisores distintos.
    \item Demuestre que $\sqrt{2}$ es irracional.
    \item Sea  $n \in \mathbf{Z^+}$. Demuestre que existen infinitos m\'ultiplos de $n$ que contienen a todos los d\'igitos decimales.
    \item Demuestre que si $p$ y $p^2+2$ son primos entonces $p^3+2$ es primo.
\end{enumerate}
\end{document}