\documentclass{article}
\usepackage{enumitem}
\usepackage{tikz}
\usetikzlibrary{trees}
\usepackage{amsmath} 
\usepackage{amssymb}
\usepackage{nopageno}

\title{Clase pr\'actica 5}

\begin{document}

\maketitle
\begin{enumerate}
    \item Sea $a \in \mathbb{Z}$. Demuestra que existen infinitos m\'ultiplos de $a$ que terminan en cualquier secuencia de d\'igitos, si $a$ es coprimo con $10$.
    \item Una mujer va al mercado con un bolso de huevos. Un caballo estacionado all\'i se los rompe. El due\~no del animal, apenado por la situaci\'on, ofrece retribuir lo perdido. Pregunta cu\'antos huevos tra\'ia. La mujer no recuerda el n\'umero exacto, pero recuerda que cuando los agrupaba en grupos de a dos, quedaba uno afuera; lo mismo suced\'ia con grupos de tres, cuatro, cinco y seis. Sin embargo, en grupos de siete, no sobraba ninguno. ?`Cu\'al es el menor n\'umero de huevos que debe pagar?
    \item Determine el menor $n$ impar, $n > 3$, tal que $3|n$, $5|n+2$ y $7|n+4$.
    \item Sean $p$ y $q$ dos n\'umeros primos diferentes, demuestre que $\exists k \in \mathbb{Z}$, tal que $pn^{q} + qn^{p} + kn$ es divisible por $pq$ $\forall n \in \mathbb{Z}$.
    \item Demuestra que dado $k \in \mathbb{Z}^{+}$, es posible encontrar una secuencia de $k$ enteros consecutivos, cada uno divisible por un cubo mayor que 1.
    \item Sea $p$ un n\'umero primo, $p > 5$, demuestre que $(p-1)! + 1$ tiene al menos dos divisores primos distintos.
    \item Demuestre que $61! + 1$ es divisible por $71$.
\end{enumerate}
\end{document}