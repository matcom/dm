\documentclass{article}
\usepackage{enumitem}
\usepackage{tikz}
\usetikzlibrary{trees}
\usepackage{amsmath} 
\usepackage{amssymb}
\usepackage{nopageno}

\title{Clase pr\'actica 6}

\begin{document}

\maketitle
\begin{enumerate}
	\item Demuestre que si $n>1$, entonces $$n^{\frac{\tau(n)}{2}} = \prod_{d | n} d$$.
	\item Demuestre que $\phi(n)$ es multiplicativa usando el teroema chino del resto.
	\item Demuestre que hay infinitos n\'umeros primos utilizando que para $n>2$, $\phi(n)$ es par.
	\item Sea $n \in \mathbb{Z}$ tal que $(n, 10) = 1$, entonces $n$ divide a un entero cuyos d\'igitos son todos iguales a 1.
	\item Demuestre que para $n \geq 1$
	$$n = \sum_{d|n} \phi(d)$$
	\item Demuestre que para $n>1$, la suma de los enteros positivos menores que n y coprimos con n, es $\frac{1}{2}n\phi(n)$.
	\item Demuestre que si $n$ es compuesto, entonces se cumple que $\phi(n) \leq n - \sqrt{n}$.
\end{enumerate}
\end{document}