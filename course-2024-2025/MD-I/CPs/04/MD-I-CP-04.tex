\documentclass{article}
\usepackage{enumitem}
\usepackage{tikz}
\usetikzlibrary{trees}
\usepackage{amsmath} 
\usepackage{amssymb}
\usepackage{nopageno}

\title{Clase pr\'actica 4}

\begin{document}

\maketitle
\begin{enumerate}
    \item Clasifique en verdadero o falso las siguientes afirmaciones:
    \begin{itemize} 
        \item $37621 + 2^{30} * 471 + 59603 * 25$ es divisible por 12.
        \item $375121  * 4^{105} - 35^{91}$ es coprimo (primo relativo) con 6 y $9^{1684} - 7^{52688}$ es divisible por 10.
        \item $2^{70} + 3^{70}$ es divisible por 13 y $3^{47}$ deja resto 4 cuando se divide por 23.
    \end{itemize}
    \item Demuestre que es finita la cantidad de valores de n, para los cuales la suma desde k=1 hasta n de k! es un cuadrado ($\left[\sum_{k=1}^{n} k! = x^{2}\right]$).
    \item Demuestre que las siguientes ecuaciones no tienen soluci\'on en enteros ($\mathbb{Z}$):
        \begin{itemize}
            \item $3x^{2} + 5 + 9xy = y^{2}$
            \item $x^{2} + y^{2} - 8z = 6$
        \end{itemize}
    \item Determine el n\'umero de ternas (a, b, c) que satisfagan: $2^{a} + 2^{b} = c!$.
    \item Sea P un n\'umero primo tal que si $P \equiv 5(8)$ y $P| \left( a^{4} + b^{4} \right) $ entonces $P|a$ y $P|b$.
    \item Demuestre que dados tres n\'umeros enteros cualesquiera, siempre es posible seleccionar dos de ellos, sean estos $a$ y $b$ tales que el n\'umero $a^{3}b - ab^{3}$ sea divisible por 10.
    \item Se le llama `n\'umero de Fermat' a aquellos que pueden ser escritos de la forma $2^{2^{n}}+ 1$, $n \geq 0$. Demuestre que los n\'umeros de Fermat son coprimos dos a dos.\\
    a) Utilizando este resultado, demuestre que existen infinitos n\'umeros primos.    
\end{enumerate}


\end{document}
