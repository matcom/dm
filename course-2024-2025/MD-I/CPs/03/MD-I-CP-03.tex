\documentclass{article}
\usepackage{enumitem}
\usepackage{tikz}
\usetikzlibrary{trees}
\usepackage{amsmath} 
\usepackage{nopageno}

\begin{document}

\title{Clase pr\'actica 3}
\maketitle

\begin{enumerate}
    \item Sean $a$ y $n$ enteros mayores que $1$.
    \begin{itemize}
        \item Si $a^n - 1$ es primo $\Rightarrow$ $a=2$ y $n$ es primo.
        \item Si $a^n + 1$ es pimo $\Rightarrow$ $a$ es par y $n$ es una potencia de 2
    \end{itemize}
    \item Demustra que existe un bloque de 2022 enteros consecutivos donde exactamente 15 de ellos son primos.
    \item Sean $n,m$ enteros positivos, con $n>1$. Demustre que $(a^n -1, a^m -1)= a^{(n,m)} -1$
    \item Sea $n \in \mathbf{Z} ~ n>4$. Demuestra que $n|(n-1)!$ si y solo si $n$ es compuesto.
    \item Se define $p_1p_2...p_n$, $p_1 =2$ y $\forall n>1$ $p_n$ es el mayor primo que divide a $p_1p_2...p_n +1$. Demuestra que $5$ no existe en la seccuencia.
    \item Determine los valores enteros positivos de n, para los cuales $n^4 + 4$ es primo. 
\end{enumerate}
\end{document}