\documentclass{article}
\usepackage{enumitem}
\usepackage{tikz}
\usetikzlibrary{trees}
\usepackage{amsmath} 
\usepackage{nopageno}

\begin{document}

\title{Clase pr\'actica introductoria}
\maketitle

\begin{enumerate}
    \item Demuestre que $\sum_{i=1}^{n} i = \frac{n(n+1)}{2}$.
    \item Deduzca el valor de $\sum_{i=1}^{n} 2*i - 1$.
    \item En una fiesta se encuentran $n$ personas, si cada persona saluda a todas las dem\'as, cu\'antos saludos se dieron?.
    \item Se tiene una matriz de $128 * 128$, demuestre que, si se quita una cuadr\'icula aleatoria entonces se puede completar la matriz utilizando cuadr\'iculas en forma de $L$ de tamaño 3.
    \item Se tiene una matriz de $n * m$, de cu\'antas formas se puede llegar desde la posici\'on $(1, 1)$ hasta la posici\'on $(n, m)$, dado una posici\'on $(i, j)$ se puede mover para $(i+1, j)$ o $(i, j+1)$. Considere tambi\'en una soluci\'on con recursividad.
    \item Manzano y Alejandra est\'an obstinados de la vida y deciden ponerse a jugar con piedras. Tienen un total de $2024$ piedras, Alejandra va primero, luego Kevin .... Kevin es muy mala cabeza y quiere saber la cantidad de $K$ para las cuales \'el siempre tiene una estrategia ganadora. Dado una $K$, los movimientos posibles para cualquier jugador en su turno son retirar desde $1$ hasta $K$ piedras. Un jugador pierde cuando no puede jugar en su turno.
    \item Se tiene el polinomio $p(x) = (x-a)(x-b)(x-c)...(x-y)(x-z)$, se cumple que $a+b+c+...+y+z=100$. Calcule $p(1024)$.
    \item En cualquier grupo de $6$ personas, dado la relaci\'on sim\'etrica de conocerse, demuestre que siempre hay $3$ personas que se conocen m\'utuamente o que se desconocen m\'utuamente.
\end{enumerate}
\end{document}