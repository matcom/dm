\documentclass{article}
\usepackage{enumitem}
\usepackage{tikz}
\usetikzlibrary{trees}
\usepackage{amsmath} 
\usepackage{amssymb}
\usepackage{nopageno}

\title{Clase pr\'actica 7}

\begin{document}

\maketitle
\begin{enumerate}
	\item Determine el n\'umero de subconjuntos de tama\~no $k$ que se pueden formar del conjunto $A = \{1, 2, \dots, n \} $ tal que ninguno de esos subconjuntos contenga dos elementos consecutivos.
	\item Calcule el n\'umero de enteros de 5 d\'igitos divisibles por 3 que contienen al 9.
	\item Sean $n, k \in \mathbb{Z}_+$ y $A$ un conjunto de tama\~no $n$. Calcule el n\'umero de k-uplas $<A_1, A_2, \dots, A_K>$ de subconjuntos de $A$ que cumplen que:
	\item[] 
	\begin{itemize}
		\item $A_1 \subseteq A_2 \subseteq \dots \subseteq A_k$.
		\item $A_1 \cap A_2 \cap \dots \cap A_k = \emptyset$.
	\end{itemize}
	\item Sea $E$ un conjunto de cardinalidad n. Calcule el n\'umero de pares no ordenados de subconjuntos de $E$ no nulos $A,B$ tales que tengan intersecci\'on nula.
	\item Una permutaci\'on de n n\'umeros es casi creciente si solo existe un \'unico $k, k < n$ tal que $a_k > a_{k+1}$. Calcule el n\'umero de permutaciones casi crecientes que hay en el conjunto $\{1, 2, \dots, n\}$.
	\item Es conocido que la serie $\sum_{k=1}^{\infty} \frac{1}{k}$ diverge. ¿Qu\'e ocurre si se extraen todos los n\'umeros que contienen al menos un $2$ entre sus cifras?
\end{enumerate}
\end{document}